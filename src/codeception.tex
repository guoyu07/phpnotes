\part{Codeception}


\chapter{Overview}




Codeception是在PHPUnit基础上开发的BDD风格的测试框架,同时Codeception定义了自己的领域测试语言来实现可描述的测试。


\begin{lstlisting}[language=bash]
$ composer require "codeception/codeception" --dev
$ ./vendor/bin/codecept
Codeception 2.3.6

Usage:
  command [options] [arguments]
\end{lstlisting}

如果以Phar形式运行,则执行:

\begin{lstlisting}[language=bash]
$ wget http://codeception.com/codecept.phar
$ php codecept.phar
Codeception 2.3.6

Usage:
  command [options] [arguments]
\end{lstlisting}

如果需要全局安装codecept,则可以执行:

\begin{lstlisting}[language=bash]
$ sudo curl -LsS http://codeception.com/codecept.phar -o /usr/local/bin/codecept
$ sudo chmod a+x /usr/local/bin/codecept
$ codecept
Codeception 2.3.6

Usage:
  command [options] [arguments]
\end{lstlisting}



\section{验收测试}


设想这样一个场景:当技术人员开发完毕, 其客户, 产品经理, 或者是测试人员, 他们怎么确定产品的可用性? 

一般情况下, 他们都是执行以下几个步骤进行测试:

\begin{compactenum}
\item 打开浏览器;
\item 输入 url;
\item 看到一些信息, 并确定了这个页面是可用的;
\item 点击某个 url;
\item 填写表单, 并提交表单, 看到了某些信息, 并确定此功能是可用的
\end{compactenum}

和手动测试(或人工测试)相反,Codeception自动化测试的Acceptance Tests会利用浏览器的编程接口来执行上述的人工测试,涉及到的步骤完全自动化。


\begin{lstlisting}[language=PHP]
<?php
$I = new AcceptanceTester($scenario);
$I->amOnPage('/');
$I->click('Sign Up');
$I->submitForm('#signup', array('username' => 'MilesDavis', 'email' => 'miles@davis.com'));
$I->see('Thank you for Signing Up!');
\end{lstlisting}

验收测试的优点如下:

\begin{compactitem}
\item 可用来测试任何网站;
\item 完全基于浏览器, 可以测试 Javascript 甚至是 ajax 请求;
\item 可以把运行状态给 产品经理 或者 客户看, 让人信服;
\item 不需要多余的配置, 对 App 源码修改要求最少, 代码适应性好, 可以当成整个应用来测试, 不在乎内部实现.
\end{compactitem}

验收测试的缺点:

\begin{compactitem}
\item 测试速度缓慢, 因为需要运行在浏览器和真实的数据库上;
\item 相比单元测试, 做不到完全的测试, 有些细微的逻辑可能会错过;
\item 在运行的时候有时候会发生不可控的事情, 因为浏览器的渲染, javascript 的运行, 有时候会有意想不到的情况发生.
\end{compactitem}



\section{功能测试}


功能测试可以模拟一个Web请求(模拟\$\_GET和\$\_POST等变量),发送给API接口并接收返回结果,这样在功能测试过程中就可以进行分析和断言来判定返回的数据,或者检查数据是否正常存储到数据库。

功能测试需要预先提供一个测试环境,例如Laravel提供了Package来集成测试环境。

以下是一个简单的功能测试:

\begin{lstlisting}[language=PHP]
<?php
$I = new FunctionalTester($scenario);
$I->amOnPage('/');
$I->click('Sign Up');
$I->submitForm('#signup', array('username' => 'MilesDavis', 'email' => 'miles@davis.com'));
$I->see('Thank you for Signing Up!');
$I->seeEmailSent('miles@davis.com', 'Thank you for registration');
$I->seeInDatabase('users', array('email' => 'miles@davis.com'));
\end{lstlisting}

和Acceptance Tests(验收测试)的语法类似,在测试环境中可以检查 email和数据库。

功能测试的优点如下:

\begin{compactitem}
\item 跟 Acceptance tests 类似, 但是少了打开浏览器来渲染, 速度快多了;
\item 能提供更详细的分析, 如数据库或者 email;
\item 可读性很强, 虽然没法让测试人员看到打开浏览器模拟人工测试, 但是还是可以让别人信服;
\item 比较稳定, 只有当大规模的代码变更, 或者把代码从一个框架转移到另一个框架的时候, 才会有影响.
\end{compactitem}

功能测试的缺点如下:

\begin{compactitem}
\item 无法测试 javascript 和 ajax;
\item 因为使用代码相对简单的模拟一个浏览器请求, 测试的可行度, 或者说完整性, 会相对较差;
\item 需要一个框架的支持。
\end{compactitem}

\section{单元测试}


单元测试(又称为模块测试, Unit Testing)是针对程序模块(软件设计的最小单位)来进行正确性检验的测试工作, 当 functional 或者 acceptance 测试都检查不到 最小单位 的逻辑时, 还能通过 单元测试 确认深藏在代码里面的某些功能仍然可用, 单元测试能消除程序单元的不可靠性。


Codeception 的单元测试功能是基于 PHPUnit 之上的,因此Codeception可以运行PHPUnit的测试代码。

Codeception 在 PHPUnit 的基础上提供了一系列工具能让单元测试更加简单, 代码可读性更高. 单元测试是最复杂最繁琐的测试, 并且是会跟着业务逻辑代码的改变而改变, 在实际开发中技术人员会经常因为需求、业务的变更而修改单元测试, 提高其可读性和易用性可以帮助相关人员更加快速的跟上一切变化。


以下是一个简单的 Unit test (单元测试)


\begin{lstlisting}[language=PHP]
<?php
function testSavingUser()
{
    $user = new User();
    $user->setName('Miles');
    $user->setSurname('Davis');
    $user->save();
    $this->assertEquals('Miles Davis', $user->getFullName());
    $this->unitTester->seeInDatabase('users',array('name' => 'Miles', 'surname' => 'Davis'));
}
\end{lstlisting}

单元测试的优点如下:

\begin{compactitem}
\item 最快的测试, 当然, 在上面的示例代码中, 触碰到了数据库, 还是有点延迟;
\item 能把测试覆盖到特别刁钻的程序逻辑上, 这是 functional 或者 acceptance 所做不到的;
\item 允许你测试最核心代码, 确定核心代码的健壮性。
\end{compactitem}

单元测试的缺点如下:

\begin{compactitem}
\item 因为是单元测试, 会把代码分为多个小单元单独测试, 但是各个单元之间的对接测试不到;
\item 对代码的修改非常敏感, 很多项目的 test 最后没用上就是因为测试跟不上业务逻辑代码的修改.
\end{compactitem}


\chapter{Bootstrap}







在安装完成Codeception之后,执行\texttt{codecept bootstrap}初始化配置来生成codeception.yml配置文件和tests目录及其默认测试集(其中包含了验收测试/功能测试和单元测试等)。



\begin{lstlisting}[language=bash]
$ codecept bootstrap

codecept bootstrap

==== Redirecting to Composer-installed version in vendor/codeception ====
 Bootstrapping Codeception 

File codeception.yml created       <- global configuration
> Unit helper has been created in tests/_support/Helper
> UnitTester actor has been created in tests/_support
> Actions have been loaded
tests/unit created                 <- unit tests
tests/unit.suite.yml written       <- unit tests suite configuration
> Functional helper has been created in tests/_support/Helper
> FunctionalTester actor has been created in tests/_support
> Actions have been loaded
tests/functional created           <- functional tests
tests/functional.suite.yml written <- functional tests suite configuration
> Acceptance helper has been created in tests/_support/Helper
> AcceptanceTester actor has been created in tests/_support
> Actions have been loaded
tests/acceptance created           <- acceptance tests
tests/acceptance.suite.yml written <- acceptance tests suite configuration
 --- 

 Codeception is installed for acceptance, functional, and unit testing 
\end{lstlisting}


接下来的步骤:

\begin{compactenum}
\item Edit tests/acceptance.suite.yml to set url of your application. Change PhpBrowser to WebDriver to enable browser testing
\item Edit tests/functional.suite.yml to enable a framework module. Remove this file if you don't use a framework
\item Create your first acceptance tests using \texttt{codecept g:cest acceptance First}
\item Write first test in tests/acceptance/FirstCest.php
\item Run tests using: \texttt{codecept run}
\end{compactenum}

Codeception支持验收测试、功能测试和单元测试。

\begin{compactitem}
\item Acceptance Tests(验收测试)
\item Functional Tests(功能测试)
\item Unit Tests(单元测试)
\end{compactitem}

在上面的示例中继续创建验收测试来模拟真实用户访问网站,例如:


\begin{lstlisting}[language=bash]
$ codecept generate:cest acceptance First

==== Redirecting to Composer-installed version in vendor/codeception ====
Test was created in tests/acceptance/FirstCest.php
\end{lstlisting}

上述的Codeception测试场景称为Cepts。

在创建好Cepts后需要配置Codeception配置文件(例如tests/acceptance.suite.yml),例如:

\begin{lstlisting}[language=bash]
actor: AcceptanceTester
modules:
      enabled:
              - PhpBrowser:
                    url: http://dev.hitour.cc
              - \Helper\Acceptance
\end{lstlisting}

接下来开始第一个测试(例如tests/acceptance/FirstCest.php):

\begin{lstlisting}[language=PHP]
<?php
class FirstCest 
{
    public function frontpageWorks(AcceptanceTester $I)
    {
        $I->amOnPage('/');
        $I->see('Home');  
    }
}
\end{lstlisting}


上面测试将验证首页是否包含“Home”字符串,Codeception会使用PhpBrowser来请求和检查HTML内容、链接、表单并提交POST或GET请求。

\begin{lstlisting}[language=PHP]
$ codecept run --steps
\end{lstlisting}

更复杂的测试则需要使用包含WebDriver模块的Selenium的浏览器。


\chapter{Template}

除了使用默认的Boostrap命令来生成测试代码之外,Codeception也提供了预定义的模板来代替Bootstrap。

\begin{compactitem}
\item Acceptance Testing

\begin{lstlisting}[language=PHP]
$ codecept init acceptance
\end{lstlisting}

\item Rest API Testing


\begin{lstlisting}[language=PHP]
$ codecept init api
\end{lstlisting}


\item Unit Testing


\begin{lstlisting}[language=PHP]
$ codecept init unit
\end{lstlisting}


\end{compactitem}

