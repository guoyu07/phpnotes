\part{Zend Opcache}

\chapter{Overview}


PHP 5.5.0 及后续版本中默认绑定的OPcache扩展通过将 PHP 脚本预编译的字节码存储到共享内存中来提升 PHP 的性能, 存储预编译字节码的好处是不需要每次加载和解析PHP脚本,PHP5.4及之前的版本未内置Opcache扩展。

Opcache默认支持file://和phar://流处理,不支持自定义数据流。

如果需要将Xdebug扩展和OPcache一起使用,必须在Xdebug扩展之前加载 OPcache 扩展。

\section{Requirement}

构建Opcache扩展不需要其他扩展。

\section{Resource Type}


Opcache扩展没有定义资源类型。


\section{Runtime Configure}

OPcache 只能编译为共享扩展。 如果使用\texttt{-\/-disable-all}参数 禁用了默认扩展的构建, 那么必须使用\texttt{-\/-enable-opcache}选项来开启 OPcache。


\begin{compactitem}
\item \texttt{zend\_extension=/full/path/to/opcache.so}
\item \texttt{zend\_extension=/full/path/to/php\_opcache.dll}
\end{compactitem}

zend\_extension 指令用来将 OPcache 扩展加载到 PHP 中,使用下列推荐设置来获得较好的性能:

\begin{lstlisting}[language=bash]
opcache.memory_consumption=128
opcache.interned_strings_buffer=8
opcache.max_accelerated_files=4000
opcache.revalidate_freq=60
opcache.fast_shutdown=1
opcache.enable_cli=1
\end{lstlisting}

可以禁用 opcache.save\_comments 并且启用 opcache.enable\_file\_override。

在生产环境中使用上述配置之前,必须经过严格测试。 因为上述配置存在一个已知问题,它会引发一些框架和应用的异常, 尤其是在文档使用了注解的情况下。

Opcache提供的函数的行为受php.ini中的对应设置的影响。

\begin{longtable}{|m{150pt}|m{40pt}|m{150pt}|}
%head
\multicolumn{3}{r}{}
\tabularnewline\hline
配置项&默认值&可修改范围
\endhead
%endhead

%firsthead
\caption{Opcache配置选项}\\
\hline
配置项&默认值&可修改范围
\endfirsthead
%endfirsthead

%foot
\multicolumn{3}{r}{}
\endfoot
%endfoot

%lastfoot
\endlastfoot
%endlastfoot
\hline
opcache.enable						&\texttt{"1"}			&PHP\_INI\_ALL	 \\
\hline
opcache.enable\_cli					&\texttt{"1"}			&PHP\_INI\_SYSTEM\\
\hline
opcache.memory\_consumption			&\texttt{"64"}			&PHP\_INI\_SYSTEM\\
\hline	 
opcache.interned\_strings\_buffer		&\texttt{"4"}			&PHP\_INI\_SYSTEM\\
\hline	 
opcache.max\_accelerated\_files			&\texttt{"2000"}		&PHP\_INI\_SYSTEM\\
\hline	 
opcache.max\_wasted\_percentage		&\texttt{"5"}			&PHP\_INI\_SYSTEM\\
\hline	 
opcache.use\_cwd						&\texttt{"1"}			&PHP\_INI\_SYSTEM\\
\hline	 
opcache.validate\_timestamps			&\texttt{"1"}			&PHP\_INI\_ALL	 \\
\hline
opcache.revalidate\_freq				&\texttt{"2"}			&PHP\_INI\_ALL	 \\
\hline
opcache.revalidate\_path				&\texttt{"0"}			&PHP\_INI\_ALL	 \\
\hline
opcache.save\_comments				&\texttt{"1"}			&PHP\_INI\_SYSTEM\\
\hline	 
opcache.load\_comments				&\texttt{"1"}			&PHP\_INI\_ALL	 \\
\hline
opcache.fast\_shutdown				&\texttt{"0"}			&PHP\_INI\_SYSTEM\\
\hline	 
opcache.enable\_file\_override			&\texttt{"0"}			&PHP\_INI\_SYSTEM\\
\hline	 
opcache.optimization\_level			&\texttt{"0xffffffff"}	&PHP\_INI\_SYSTEM\\
\hline	 
opcache.inherited\_hack				&\texttt{"1"}			&PHP\_INI\_SYSTEM \\
\hline
opcache.dups\_fix						&\texttt{"0"}			&PHP\_INI\_ALL	 \\
\hline
opcache.blacklist\_filename				&\texttt{""}			&PHP\_INI\_SYSTEM \\
\hline
opcache.max\_file\_size				&\texttt{"0"}			&PHP\_INI\_SYSTEM \\
\hline
opcache.consistency\_checks			&\texttt{"0"}			&PHP\_INI\_ALL	 \\
\hline
opcache.force\_restart\_timeout			&\texttt{"180"}		&PHP\_INI\_SYSTEM \\
\hline
opcache.error\_log					&\texttt{""}			&PHP\_INI\_SYSTEM \\
\hline
opcache.log\_verbosity\_level			&\texttt{"1"}			&PHP\_INI\_SYSTEM \\
\hline
opcache.preferred\_memory\_model		&\texttt{""}			&PHP\_INI\_SYSTEM \\
\hline
opcache.protect\_memory				&\texttt{"0"}			&PHP\_INI\_SYSTEM \\
\hline
opcache.mmap\_base					&\texttt{NULL}		&PHP\_INI\_SYSTEM \\
\hline
opcache.restrict\_api					&\texttt{""}			&PHP\_INI\_SYSTEM \\
\hline
opcache.file\_cache					&\texttt{NULL}		&PHP\_INI\_SYSTEM\\
\hline
opcache.file\_cache\_only				&\texttt{"0"}			&PHP\_INI\_SYSTEM\\
\hline
opcache.file\_cache\_consistency\_checks	&\texttt{"1"}			&PHP\_INI\_SYSTEM\\
\hline
opcache.file\_cache\_fallback			&\texttt{"1"}			&PHP\_INI\_SYSTEM\\
\hline
opcache.validate\_permission			&\texttt{"0"}			&PHP\_INI\_SYSTEM\\
\hline
opcache.validate\_root					&\texttt{"0"}			&PHP\_INI\_SYSTEM\\
\hline
\end{longtable}


\section{Predefined Constants}



\chapter{Opcache Function}


\section{opcache\_compile\_file()}


无需运行,即可编译并缓存 PHP 脚本


\section{opcache\_get\_configuration()}

获取缓存的配置信息


\section{opcache\_get\_status()}

获取缓存的状态信息


\section{opcache\_invalidate()}

清除脚本缓存

\section{opcache\_is\_script\_cached()}

查询某个脚本是否缓存在Opcache中

\section{opcache\_reset()}

重置字节码缓存的内容






