\part{Option/Info}

\chapter{Overview}


PHP的选项/信息函数允许用户获取关于PHP自身的参数(例如运行时的配置,被加载的扩展和版本等),而且用户也可以使用这些函数来设置运行中的PHP的选项。


\section{Requirement}


PHP的选项/信息函数是PHP核心的一部分,构建它们不需要其他扩展。

\section{Runtime Configuration}

PHP的选项/信息函数受到php.ini中的设置的影响。


\begin{longtable}{|m{150pt}|m{30pt}|m{100pt}|}
%head
\multicolumn{3}{r}{}
\tabularnewline\hline
名字&默认&可修改范围
\endhead
%endhead

%firsthead
\caption{PHP选项/Inf配置选项}\\
\hline
名字&默认&可修改范围
\endfirsthead
%endfirsthead

%foot
\multicolumn{3}{r}{}
\endfoot
%endfoot

%lastfoot
\endlastfoot
%endlastfoot
\hline
assert.active				&\texttt{"1"}		&\texttt{PHP\_INI\_ALL}\\
\hline	 
assert.bail				&\texttt{"0"}		&\texttt{PHP\_INI\_ALL}\\
\hline	 
assert.warning			&\texttt{"1"}		&\texttt{PHP\_INI\_ALL}\\
\hline	 
assert.callback			&\texttt{NULL}	&\texttt{PHP\_INI\_ALL}\\
\hline	 
assert.quiet\_eval			&\texttt{"0"}		&\texttt{PHP\_INI\_ALL}\\
\hline	 
assert.exception			&\texttt{"0"}		&\texttt{PHP\_INI\_ALL}\\
\hline
enable\_dl				&\texttt{"1"}		&\texttt{PHP\_INI\_SYSTEM}\\
\hline
max\_execution\_time		&\texttt{"30"}		&\texttt{PHP\_INI\_ALL} \\
\hline
max\_input\_time			&\texttt{"-1"}		&\texttt{PHP\_INI\_PERDIR}\\
\hline
max\_input\_nesting\_level	&\texttt{"64"}		&\texttt{PHP\_INI\_PERDIR}\\
\hline
max\_input\_vars			&\texttt{1000}	&\texttt{PHP\_INI\_PERDIR}\\
\hline
magic\_quotes\_gpc		&\texttt{"1"}		&\texttt{PHP\_INI\_PERDIR}\\
\hline
magic\_quotes\_runtime	&\texttt{"0"}		&\texttt{PHP\_INI\_ALL}\\
\hline
zend.enable\_gc			&\texttt{"1"}		&\texttt{PHP\_INI\_ALL}\\
\hline
\end{longtable}


\subsection{assert.active}

assert.active(boolean)-激活assert()断言检查。

\subsection{assert.bail}

assert.bail(boolean)-失败的断言将中止脚本。

\subsection{assert.warning}

assert.warning(boolean)-每个失败的断言产生一条PHP警告信息。

\subsection{assert.callback}

assert.callback(string)-失败的断言将调用用户定义的回调函数。

\subsection{assert.quiet\_eval()}

assert.quiet\_eval(boolean)-在断言表达式执行时error\_reporting()使用当前的设置。

\begin{compactitem}
\item 如果启用了,在执行时错误将不会被显示(隐式的 error\_reporting(0))。
\item 如果禁用了,错误将根据 error\_reporting() 的设置来显示。
\end{compactitem}


\subsection{assert.exception}

assert.exception(boolean)-在断言(assert)失败时产生 AssertionError 异常。


\subsection{enable\_dl}

enable\_dl(boolean)-指令仅对 Apache 模块版本的 PHP 有效,可以针对每个虚拟机或每个目录开启或关闭 dl() 动态加载 PHP 模块。

\begin{compactitem}
\item 默认允许动态加载,除了使用 安全模式。
\item 在安全模式下,总是无法使用 dl()。
\end{compactitem}




关闭动态加载的主要原因是为了安全,否则通过动态加载,有可能忽略所有 open\_basedir 限制。

\subsection{max\_execution\_time}


max\_execution\_time(integer)-设置了脚本被解析器中止之前允许的最大执行时间(单位秒),有助于防止写得不好的脚本占尽服务器资源,默认设置为 30。

从命令行运行 PHP 时,默认设置为 0,最大执行时间不会影响系统调用和系统操作等。

在安全模式下不能通过 ini\_set() 来修改此设置,唯一的解决方法是关闭安全模式或者在 php.ini 中修改时间限制。

另外,Web服务器也可以有其他超时设置,也有可能中断 PHP 的执行。例如,Apache 有一个 Timeout 指令,IIS 有一个 CGI 超时功能,它们默认都是 300 秒。

\subsection{max\_input\_time}


max\_input\_time(integer)-脚本解析输入数据(类似 POST 和 GET)允许的最大时间(单位是秒),它是从接收所有数据到开始执行脚本进行测量的。

\subsection{max\_input\_nesting\_level}

max\_input\_nesting\_level(integer)-设置输入变量的嵌套深度 (例如\$\_GET,\$\_POST等)。

\subsection{max\_input\_vars}

max\_input\_vars(integer)-接受多少输入的变量(限制分别应用于\$\_GET、\$\_POST 和 \$\_COOKIE 超全局变量) 指令的使用减轻了以哈希碰撞来进行拒绝服务攻击的可能性。如果有超过指令指定数量的输入变量,将会导致 E\_WARNING 的产生, 更多的输入变量将会从请求中截断。

max\_input\_vars的限制可以通过流式输入来解除,例如:

\begin{lstlisting}[language=PHP]
<?php
$sRawInputData = fopen('php://input');
\end{lstlisting}

\subsection{magic\_quotes\_gpc}

magic\_quotes\_gpc(boolean,已废弃)-为 GPC (Get/Post/Cookie) 操作设置 magic\_quotes 状态。


\begin{compactitem}
\item 如果magic\_quotes 为 on,所有的 \texttt{'} (单引号)、\texttt{"}(双引号)、\textbackslash (反斜杠)和 \texttt{NUL's}被一个反斜杠自动转义。
\item 如果magic\_quotes\_sybase 也是 on,它会完全覆盖 magic\_quotes\_gpc。
\item 如果两个指令都启用意味着只有单引号被转义为 \texttt{''},双引号、反斜杠和\texttt{NUL's}不会被转义。
\end{compactitem}

\subsection{magic\_quotes\_runtime}

magic\_quotes\_runtime(boolean,已废弃)-如果启用了 magic\_quotes\_runtime,大多数返回任何形式外部数据的函数,包括数据库和文本段将会用反斜线转义引号。

如果启用了 magic\_quotes\_sybase,单引号会被单引号转义而不是反斜线。


受 magic\_quotes\_runtime 影响的函数(不包括 PECL 里的函数)包括:

\begin{compactitem}
\item get\_meta\_tags()
\item file\_get\_contents()
\item file()
\item fgets()
\item fwrite()
\item fread()
\item fputcsv()
\item stream\_socket\_recvfrom()
\item exec()
\item system()
\item passthru()
\item stream\_get\_contents()
\item bzread()
\item gzfile()
\item gzgets()
\item gzwrite()
\item gzread()
\item exif\_read\_data()
\item dba\_insert()
\item dba\_replace()
\item dba\_fetch()
\item ibase\_fetch\_row()
\item ibase\_fetch\_assoc()
\item ibase\_fetch\_object()
\item mssql\_fetch\_row()
\item mssql\_fetch\_object()
\item mssql\_fetch\_array()
\item mssql\_fetch\_assoc()
\item mysqli\_fetch\_row()
\item mysqli\_fetch\_array()
\item mysqli\_fetch\_assoc()
\item mysqli\_fetch\_object()
\item pg\_fetch\_row()
\item pg\_fetch\_assoc()
\item pg\_fetch\_array()
\item pg\_fetch\_object()
\item pg\_fetch\_all()
\item pg\_select()
\item sybase\_fetch\_object()
\item sybase\_fetch\_array()
\item sybase\_fetch\_assoc()
\item SplFileObject::fgets()
\item SplFileObject::fgetcsv()
\item SplFileObject::fwrite()
\end{compactitem}



\subsection{zend.enable\_gc}

zend.enable\_gc(boolean)-启用或禁用循环引用记数搜集器。


\section{Resource Type}

PHP的选项/信息函数没有定义资源类型。

\section{Predefined Constants}

下列常量作为 PHP 核心的一部分总是可用的。


\subsection{phpcredits()}


\begin{longtable}{|m{100pt}|m{30pt}|m{200pt}|}
%head
\multicolumn{3}{r}{}
\tabularnewline\hline
常量&值&描述
\endhead
%endhead

%firsthead
\caption{phpcredits()预定义常量}\\
\hline
常量&值&描述
\endfirsthead
%endfirsthead

%foot
\multicolumn{3}{r}{}
\endfoot
%endfoot

%lastfoot
\endlastfoot
%endlastfoot
\hline
CREDITS\_GROUP	&1&	核心开发者名单\\
\hline
CREDITS\_GENERAL&2&总的贡献:语言设计和理念,PHP 作者 和 SAPI 模块。\\
\hline
CREDITS\_SAPI	&4&	PHP 的服务器 API 模块列表,以及它们的作者。\\
\hline
CREDITS\_MODULES&8&PHP 扩展的列表,以及它们的作者。\\
\hline
CREDITS\_DOCS	&16&文档组的贡献。\\
\hline
CREDITS\_FULLPAGE&32&通常与其他标志组合使用。通过其他标志指示了完整独立的 HTML 页面,用于打印包含信息。\\
\hline
CREDITS\_QA	&64&质量保证团队的贡献。\\
\hline
CREDITS\_ALL	&-1&所有的贡献者,等于使用 CREDITS\_DOCS + CREDITS\_GENERAL + CREDITS\_GROUP + CREDITS\_MODULES + CREDITS\_QA CREDITS\_FULLPAGE。它以合适的标签产生了完整的独立 HTML 页面。这是默认的值。\\
\hline
\end{longtable}

\subsection{phpinfo()}


\begin{longtable}{|m{100pt}|m{30pt}|m{200pt}|}
%head
\multicolumn{3}{r}{}
\tabularnewline\hline
常量&值&描述
\endhead
%endhead

%firsthead
\caption{phpinfo()预定义常量}\\
\hline
常量&值&描述
\endfirsthead
%endfirsthead

%foot
\multicolumn{3}{r}{}
\endfoot
%endfoot

%lastfoot
\endlastfoot
%endlastfoot
\hline
INFO\_GENERAL&1&配置行,php.ini 的位置、构建日期,Web 服务器、操作系统及其他\\
\hline
INFO\_CREDITS&2&PHP 贡献者\\
\hline
INFO\_CONFIGURATION&4&当前 PHP 指令的本地(Local)和主(Master)值\\
\hline
INFO\_MODULES&8&已加载的模块和各自的设置\\
\hline
INFO\_ENVIRONMENT&16&环境变量信息在\$\_ENV 中亦有效\\
\hline
INFO\_VARIABLES&32&显示所有 EGPCS (环境变量、GET、POST、Cookie、Server)中的预定义变量\\
\hline
INFO\_LICENSE&64&PHP 版权信息\\
\hline
INFO\_ALL&-1&显示以上所有,这是默认值\\
\hline
\end{longtable}

\subsection{assert()}

断言常量用于设置 assert\_options() 中的断言标记 。

\begin{longtable}{|m{100pt}|m{30pt}|m{200pt}|}
%head
\multicolumn{3}{r}{}
\tabularnewline\hline
常量&INI设置&描述
\endhead
%endhead

%firsthead
\caption{assert()预定义常量}\\
\hline
常量&INI设置&描述
\endfirsthead
%endfirsthead

%foot
\multicolumn{3}{r}{}
\endfoot
%endfoot

%lastfoot
\endlastfoot
%endlastfoot
\hline
ASSERT\_ACTIVE		&assert.active&启用 assert()\\
\hline
ASSERT\_CALLBACK	&assert.callback&失败断言的回调函数\\
\hline
ASSERT\_BAIL			&assert.bail&断言失败时中止执行\\
\hline
ASSERT\_WARNING	&assert.warning&为每个失败的断言产生一条 PHP 警告\\
\hline
ASSERT\_QUIET\_EVAL	&assert.quiet\_eval&在执行断言表达式时禁用 error\_reporting\\
\hline
\end{longtable}

\subsection{Windows}

以下常量仅在主机操作系统是 Windows的情况下有效,能得到不同版本信息,能够检测利用一些功能。 


\begin{longtable}{|m{150pt}|m{20pt}|m{250pt}|}
%head
\multicolumn{3}{r}{}
\tabularnewline\hline
常量&值&描述
\endhead
%endhead

%firsthead
\caption{Windows特定常量}\\
\hline
常量&值&描述
\endfirsthead
%endfirsthead

%foot
\multicolumn{3}{r}{}
\endfoot
%endfoot

%lastfoot
\endlastfoot
%endlastfoot
\hline
PHP\_WINDOWS\_VERSION\_MAJOR&&windows 主版本,可以是 4 (NT4/Me/98/95)、 5 (XP/2003 R2/2003/2000) 或 6 (Vista/2008)\\
\hline
PHP\_WINDOWS\_VERSION\_MINOR&&Windows 副版本号,可以是 0 (Vista/2008/2000/NT4/95)、 1 (XP)、2 (2003 R2/2003/XP x64)、 10 (98) 或 90 (ME)\\
\hline
PHP\_WINDOWS\_VERSION\_BUILD&&Windows 内部版本号(例如 Windows Vista SP1 是 build 6001)\\
\hline
PHP\_WINDOWS\_VERSION\_PLATFORM&&PHP 当前运行的平台, Windows Vista/XP/2000/NT4、Server 2008/2003 的值是 2, Windows ME/98/95 下值是 1\\
\hline
PHP\_WINDOWS\_VERSION\_SP\_MAJOR&&安装的 service pack 主版本号,没有安装是 0。 例如, Windows XP service pack 3 上这个值是 3\\
\hline
PHP\_WINDOWS\_VERSION\_SP\_MINOR&&安装的 service pack 副版本号,如果没有安装则是 0 \\
\hline
PHP\_WINDOWS\_VERSION\_SUITEMASK&&The suitemask is a bitmask that can tell if various features of Windows is installed\\
\hline
PHP\_WINDOWS\_VERSION\_PRODUCTTYPE&&This contains the value used to determine the PHP\_WINDOWS\_NT\_* constants. This value may be one of the PHP\_WINDOWS\_NT\_* constants indicating the platform type\\
\hline
PHP\_WINDOWS\_NT\_DOMAIN\_CONTROLLER&&这是域控制器\\
\hline
PHP\_WINDOWS\_NT\_SERVER&&这是一个服务器系统 (eg. Server 2008/2003/2000),注意如果这是一个域控制器,通过 PHP\_WINDOWS\_NT\_DOMAIN\_CONTROLLER 报告\\
\hline
PHP\_WINDOWS\_NT\_WORKSTATION&&这是一个工作站系统 (例如 Vista/XP/2000/NT4)\\
\hline
\end{longtable}

此功能列表可以通过 PHP\_WINDOWS\_VERSION\_SUITEMASK 位掩码检测。



\begin{longtable}{|m{80pt}|m{250pt}|}
%head
\multicolumn{2}{r}{}
\tabularnewline\hline
Bits&描述
\endhead
%endhead

%firsthead
\caption{Windows suitemask 位字段}\\
\hline
Bits&描述
\endfirsthead
%endfirsthead

%foot
\multicolumn{2}{r}{}
\endfoot
%endfoot

%lastfoot
\endlastfoot
%endlastfoot
\hline
0x00000004	&安装的是 Microsoft BackOffice components\\
\hline
0x00000400	&安装的是 Windows Server 2003, Web Edition\\
\hline
0x00004000	&安装的是 Windows Server 2003, Compute Cluster Edition\\
\hline
0x00000080	&安装的是 Windows Server 2008 Datacenter, Windows Server 2003, Datacenter Edition or Windows 2000 Datacenter Server\\
\hline
0x00000002	&安装的是 Windows Server 2008 Enterprise, Windows Server 2003, Enterprise Edition, Windows 2000 Advanced Server 或 Windows NT Server 4.0 Enterprise Edition\\
\hline
0x00000040	&安装的是 Windows XP Embedded\\
\hline
0x00000200	&安装的是 Windows Vista Home Premium, Windows Vista Home Basic 或 Windows XP Home Edition\\
\hline
0x00000100	&Remote Desktop is supported, but only one interactive session is supported. This value is set unless the system is running in application server mode\\
\hline
0x00000001	&Microsoft Small Business Server was once installed on the system, but may have been upgraded to another version of Windows\\
\hline
0x00000020	&Microsoft Small Business Server is installed with the restrictive client license in force\\
\hline
0x00002000	&安装的是 Windows Storage Server 2003 R2 或 Windows Storage Server 2003\\
\hline
0x00000010	&中断服务安装了。这个值总是设置的。如果这个值设置了,但 0x00000100 没有设置,操作系统运行于 application server 模式\\
\hline
0x00008000	&安装的是 Windows Home Server\\
\hline
\end{longtable}

\subsection{INI}

\begin{longtable}{|m{100pt}|m{30pt}|m{200pt}|}
%head
\multicolumn{3}{r}{}
\tabularnewline\hline
常量&值&描述
\endhead
%endhead

%firsthead
\caption{INI预定义常量}\\
\hline
常量&值&描述
\endfirsthead
%endfirsthead

%foot
\multicolumn{3}{r}{}
\endfoot
%endfoot

%lastfoot
\endlastfoot
%endlastfoot
\hline
INI\_USER&1&Unused\\
\hline
INI\_PERDIR&2&Unused\\
\hline
INI\_SYSTEM&4&Unused\\
\hline
INI\_ALL&7&Unused\\
\hline
\end{longtable}



\chapter{Functions}


\section{assert\_options()}

设置/获取断言的各种标志
\section{assert()}

检查一个断言是否为 FALSE
\section{cli\_get\_process\_title()}

Returns the current process title
\section{cli\_set\_process\_title()}

Sets the process title
\section{dl()}

运行时载入一个 PHP 扩展
\section{extension\_loaded()}

检查一个扩展是否已经加载
\section{gc\_collect\_cycles()}

强制收集所有现存的垃圾循环周期
\section{gc\_disable()}

停用循环引用收集器
\section{gc\_enable()}

激活循环引用收集器
\section{gc\_enabled()}

返回循环引用计数器的状态
\section{gc\_mem\_caches()}

Reclaims memory used by the Zend Engine memory manager
\section{get\_cfg\_var()}

获取 PHP 配置选项的值
\section{get\_current\_user()}

获取当前 PHP 脚本所有者名称
\section{get\_defined\_constants()}

返回所有常量的关联数组,键是常量名,值是常量值
\section{get\_extension\_funcs()}

返回模块函数名称的数组
\section{get\_include\_path()}

获取当前的 include\_path 配置选项
\section{get\_included\_files()}

返回被 include 和 require 文件名的 array
\section{get\_loaded\_extensions()}

返回所有编译并加载模块名的 array
\section{get\_magic\_quotes\_gpc()}

获取当前 magic\_quotes\_gpc 的配置选项设置
\section{get\_magic\_quotes\_runtime()}

获取当前 magic\_quotes\_runtime 配置选项的激活状态
\section{get\_required\_files()}

别名 get\_included\_files
\section{get\_resources()}

Returns active resources
\section{getenv()}

获取一个环境变量的值
\section{getlastmod()}

获取页面最后修改的时间
\section{getmygid()}

获取当前 PHP 脚本拥有者的 GID
\section{getmyinode()}

获取当前脚本的索引节点(inode)
\section{getmypid()}

获取 PHP 进程的 ID
\section{getmyuid()}

获取 PHP 脚本所有者的 UID
\section{getopt()}

从命令行参数列表中获取选项
\section{getrusage()}

获取当前资源使用状况
\section{ini\_alter()}

别名 ini\_set
\section{ini\_get\_all()}

获取所有配置选项
\section{ini\_get()}

获取一个配置选项的值
\section{ini\_restore()}

恢复配置选项的值
\section{ini\_set()}

为一个配置选项设置值
\section{magic\_quotes\_runtime()}

别名 set\_magic\_quotes\_runtime
\section{main()}

虚拟的main
\section{memory\_get\_peak\_usage()}

返回分配给 PHP 内存的峰值
\section{memory\_get\_usage()}

返回分配给 PHP 的内存量
\section{php\_ini\_loaded\_file()}

取得已加载的 php.ini 文件的路径
\section{php\_ini\_scanned\_files()}

返回从额外 ini 目录里解析的 .ini 文件列表
\section{php\_logo\_guid()}

获取 logo 的 guid
\section{php\_sapi\_name()}

返回 web 服务器和 PHP 之间的接口类型
\section{php\_uname()}

返回运行 PHP 的系统的有关信息
\section{phpcredits()}

打印 PHP 贡献者名单
\section{phpinfo()}

输出关于 PHP 配置的信息
\section{phpversion()}

获取当前的PHP版本
\section{putenv()}

设置环境变量的值
\section{restore\_include\_path()}

还原 include\_path 配置选项的值
\section{set\_include\_path()}

设置 include\_path 配置选项
\section{set\_magic\_quotes\_runtime()}

设置当前 magic\_quotes\_runtime 配置选项的激活状态
\section{set\_time\_limit()}

设置脚本最大执行时间
\section{sys\_get\_temp\_dir()}

返回用于临时文件的目录
\section{version\_compare()}

对比两个「PHP 规范化」的版本数字字符串
\section{zend\_logo\_guid()}

获取 Zend guid
\section{zend\_thread\_id()}

返回当前线程的唯一识别符
\section{zend\_version()}

获取当前 Zend 引擎的版本
































