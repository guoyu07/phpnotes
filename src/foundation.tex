\part{Foundation}

{\CTEXnoindent\textbf{版权信息}}


Copyright © 1997 - 2013,PHP 文档组版权所有。发行本资料必须服从 Creative Commons Attribution 3.0 或更新版许可中阐明的条款及条件。\href{http://www.php.net/manual/zh/cc.license.php}{Creative Commons Attribution 3.0 license} 的副本已随本手册发行。其最新版本位于 »~\url{http://creativecommons.org/licenses/by/3.0/}。

如有兴趣再发行或再版此文档的全部或部分内容,不论修改过与否,或有任何问题,请联系版权所有者 » \href{doc-license@lists.php.net}{doc-license@lists.php.net}。注意,本地址映射到一个公开归档的邮件列表。

{\CTEXnoindent\textbf{作者与贡献者}}

在手册的首页上仅突出了目前最活跃的人员,但还有更多的贡献者正在帮助我们工作或在过去给项目提供过巨大的帮助。有许多不知名的人帮助在手册中写下用户评论,并不断地包含在参考中,也很感谢他们的努力。下面所提供的列表均以字母顺序排序。

{\CTEXnoindent\textbf{作者与编辑}}

下列人员曾经或者目前正在为本手册添砖加瓦: Bill Abt, Jouni Ahto, Alexander Aulbach, Daniel Beckham, Stig Bakken, Nilgün Belma Bugüner, Jesus M. Castagnetto, Ron Chmara, Sean Coates, John Coggeshall, Simone Cortesi, Peter Cowburn, Daniel Egeberg, Markus Fischer, Wez Furlong, Sara Golemon, Rui Hirokawa, Brad House, Pierre-Alain Joye, Etienne Kneuss, Moriyoshi Koizumi, Rasmus Lerdorf, Andrew Lindeman, Stanislav Malyshev, Justin Martin, Rafael Martinez, Rick McGuire, Moacir de Oliveira Miranda Júnior, Kalle Sommer Nielsen, Yasuo Ohgaki, Richard Quadling, Derick Rethans, Rob Richards, Sander Roobol, Egon Schmid, Thomas Schoefbeck, Sascha Schumann, Dan Scott, Masahiro Takagi, Yannick Torres, Michael Wallner, Lars Torben Wilson, Jim Winstead, Jeroen van Wolffelaar 和 Andrei Zmievski.

下列人员对本手册做了相当数量的编辑工作: Stig Bakken, Gabor Hojtsy, Hartmut Holzgraefe 和 Egon Schmid.


{\CTEXnoindent\textbf{用户评论维护者}}

目前最活跃的维护者是: Daniel Brown, Nuno Lopes, Felipe Pena, Thiago Pojda 和 Maciek Sokolewicz.

下列人员为维护用户评论作出了巨大的努力: Mehdi Achour, Daniel Beckham, Friedhelm Betz, Victor Boivie, Jesus M. Castagnetto, Nicolas Chaillan, Ron Chmara, Sean Coates, James Cox, Vincent Gevers, Sara Golemon, Zak Greant, Szabolcs Heilig, Oliver Hinckel, Hartmut Holzgraefe, Etienne Kneuss, Rasmus Lerdorf, Matthew Li, Andrew Lindeman, Aidan Lister, Hannes Magnusson, Maxim Maletsky, Bobby Matthis, James Moore, Philip Olson, Sebastian Picklum, Derick Rethans, Sander Roobol, Damien Seguy, Jason Sheets, Tom Sommer, Jani Taskinen, Yasuo Ohgaki, Jakub Vrana, Lars Torben Wilson, Jim Winstead, Jared Wyles 和 Jeroen van Wolffelaar.

{\CTEXnoindent\textbf{中文翻译外部支持团队}}

PHP手册中文翻译工作是一项长期而又艰巨的工作,为了让这个工作得以持久进行下去,我们组织了 » \href{http://code.google.com/p/phpdoc-zh/}{PHP 手册中文翻译补完计划}。下列人员目前正在参与该计划:codingall.com(赵磊)、cuimuxi(崔玉松)、cztviztor、gaojian1226、HaoHappy(陈浩)、HonestQiao(乔楚)、kendotom、lgg860911、loosen.copen、miusun01(李鼎峰)、ping3608、r.anerg(罗翀)、suppersoft(paris.wang)、fising(王祥中)、wind.golden(陈金)。

\chapter{PHP Syntax}

\vspace{-20pt}


使用了 PHP 的Web页面将被和通常的 HTML 页面一样处理,可以用通常建立 HTML 页面的方法来建立和编辑它们,但是用户无法在浏览器中通过查看源文档的方式来查看 PHP 的源代码 - 而是只能看到 PHP 文件的输出~——即纯粹的 HTML。这是因为在结果返回浏览器之前,脚本就已经在服务器执行了。




PHP 的脚本块\footnote{用 \texttt{<?php} 来表示 PHP 标识符的起始,然后放入 PHP 语句并通过加上一个终止标识符 \texttt{?>} 来退出 PHP 模式,可以根据需要在 HTML 文件中开启或关闭 PHP 模式。}以 \texttt{<?php} 开始,以 \texttt{?>} 结束,可以把 PHP 的脚本块放置在文档中的任何位置。当然,在支持简写的服务器上,可以使用 \texttt{<?} 和 \texttt{?>} 来开始和结束脚本块。不过,为了达到最好的兼容性,推荐使用标准形式 (\texttt{<?php}),而不是简写形式。

\begin{lstlisting}[language=PHP]
<?php
  ...
?>
\end{lstlisting}

PHP 文件通常会包含 HTML 标签,就像一个 HTML 文件,以及一些 PHP 脚本代码。

在Web服务器根目录(DOCUMENT\_ROOT)下建立一个文件名为 hello.php,然后完成如下内容,它可以向浏览器输出文本 "Hello World":

\begin{lstlisting}[language=HTML]
<!DOCTYPE html>
<html>
<head>
  <title>PHP Example</title>
</head>
<body>
<?php
  echo "Hello World";
?>
</body>
</html>
\end{lstlisting}

在浏览器的地址栏里输入 web 服务器的 URL 访问这个文件,在结尾加上“/hello.php”。如果本地开发,那么这个 URL 一般是 http://localhost/hello.php 或者 http://127.0.0.1/hello.php,当然这取决于 web 服务器的设置。如果所有的设置都正确,那么这个文件将被 PHP 解析,浏览器中将会输出如下结果:

\begin{lstlisting}[language=HTML]
<!DOCTYPE html>
<html>
 <head>
  <title>PHP Example</title>
 </head>
 <body>
 <p>Hello World</p>
 </body>
</html>
\end{lstlisting}

该程序非常的简单,它仅仅只是利用了 PHP 的 echo 语句显示了 Hello World。注意,这个范例和其它用 C 或 Perl 语言写的脚本之间的区别~——与用大量的命令来编写程序以输出 HTML 不同的是,PHP 页面就是 HTML,只不过在其中嵌入了一些代码来做一些事情,PHP文件无需被执行或以任何方式指定。服务器会找到该文件并提供给 PHP 进行解释,因为使用了“.php”的扩展名,服务器已被配置成自动传递有着“.php”扩展名的文件给 PHP。

{\CTEXnoindent\textbf{PHP结束符}}

PHP 中的每个代码行都必须以分号\footnote{分号是一种分隔符,用于把指令集区分开来。}结束。



有两种通过 PHP 来输出文本的基础指令:echo 和 print。在上面的例子中就是使用echo 语句来输出文本 "Hello World"。

下面建立一个最著名的 PHP 脚本,调用函数 \texttt{phpinfo()},将会看到很多有关自己系统的有用信息,例如预定义变量、已经加载的 PHP 模块和配置信息。

\begin{lstlisting}[language=HTML]
<!DOCTYPE html>
<html>
<head>
  <title>PHP Example</title>
</head>
<body>
<?php
  phpinfo();
?>
</body>
</html>
\end{lstlisting}

尽管换行在 HTML 中的实际意义不是很大,但适当地使用换行可以使 HTML 代码易读且美观。PHP 会在输出时自动删除其结束符 \texttt{?>} 后的一个换行。该功能主要是针对在一个页面中嵌入多段 PHP 代码或者包含了无实质性输出的 PHP 文件而设计,与此同时也造成了一些疑惑。如果需要在 PHP 结束符 \texttt{?>}之后输出换行的话,可以在其后加一个空格,或者在最后的一个 echo/print 语句中加入一个换行。



{\CTEXnoindent\textbf{PHP注释}}

与C/C++/C\#/Java 等语言一样,PHP使用 \texttt{//} 来编写单行注释,或者使用 \texttt{/*} 和 \texttt{*/} 来编写大的注释块。


\begin{lstlisting}[language=HTML]
<!DOCTYPE html>
<html>
<head>
<title></title>
</head>
<body>

<?php

//This is a comment

/*
This is
a comment
block
*/

?>

</body>
</html>
\end{lstlisting}

和客户端的 JavaScript 不同的是,PHP 代码是运行在服务端的。如果在服务器上建立了如上例类似的代码,则在运行该脚本后,客户端就能接收到其结果,但他们无法得知其背后的代码是如何运作的。甚至可以将 web 服务器设置成让 PHP 来处理所有的 HTML 文件,这么一来,用户就无法得知服务端到底做了什么。

如果希望用文本编辑工具\footnote{如果使用 Windows 记事本来编写 PHP 脚本,需要注意在保存文件时,文件的后缀名应该为 .php(记事本将自动在文件名后面加上 .txt 后缀,除非采取以下措施之一来避免这种情况)。当保存文件时,系统会提示指定文件的文件名,这时需要将文件名加上引号(例如 ``hello.php")。或者,也可以点击“另存为”对话框中的“保存类型”下拉菜单,并将设置改为“所有文件”。这样在输入文件名的时候就不用加引号了。}来处理PHP脚本,必须保证将结果存成了纯文本格式,否则 PHP 将无法读取并运行这些脚本。



\chapter{PHP Variable}






PHP 中的所有变量都是以 \texttt{\$} 符号开始的,变量用于存储值,比如数字、字符串或函数的结果,这样我们就可以在脚本中多次使用它们了。与C++等不同的是,不需要在变量使用前确定该变量的类型,而是由所赋的值所决定。


在决定了一个变量的类型后,不要轻易改动,以免发生错误,下面是在 PHP 中设置变量的语法:

\begin{lstlisting}[language=PHP]
$var_name = value;
\end{lstlisting}



如果忘记在变量的前面的 \texttt{\$} 符号的话,变量将是无效的。下面将创建一个存有字符串的变量,和一个存有数值的变量:


\begin{lstlisting}[language=PHP]
<?php
$txt = "Hello World!";
$number = 16;
?>
\end{lstlisting}


PHP 是一门松散类型的语言(Loosely Typed Language),不需要在设置变量之前声明该变量,也不必向 PHP 声明该变量的数据类型,根据变量被设置的方式,PHP 会自动地把变量转换为正确的数据类型。

在强类型的编程语言中,必须在使用前声明变量的类型和名称,而在 PHP 中,变量会在使用时被自动声明。

PHP中变量的命名规则如下:

\begin{compactitem}
\item 变量名必须以字母或下划线 "\_" 开头。
\item 变量名只能包含字母数字字符以及下划线。
\item 变量名不能包含空格。如果变量名由多个单词组成,那么应该使用下划线进行分隔(比如 \texttt{\$my\_string}),或者以大写字母开头(比如 \texttt{\$myString})。
\end{compactitem}


\chapter{PHP String}


PHP中的字符串变量用于包含字符串的值,存储并处理文本片段。同时,PHP提供了很多的字符串函数供用户对字符串进行操作,从而更加灵活的处理字符串,而且PHP字符串函数是 PHP 核心的组成部分,无需安装即可使用这些函数。


在创建字符串之后就可以对它进行操作了,可以直接在函数中使用字符串,或者把它存储在变量中,比如在下面的例子中,PHP 脚本把字符串 "Hello World" 赋值给名为 \texttt{\$txt} 的字符串变量:


\begin{lstlisting}[language=PHP]
<?php
$txt="Hello World";
echo $txt;
?>
\end{lstlisting}


以上代码的输出:\verb|Hello World|


\section{Concatenation Operator}



在 PHP 中,只有一个字符串运算符,称为并置运算符 (\texttt{.}),用于把两个字符串值连接起来。

要把两个变量连接在一起,可以使用这个点运算符 (.) :

\begin{lstlisting}[language=PHP]
<?php
  $txt1="Hello World";
  $txt2="1234";
  echo $txt1 . " " . $txt2;
?>
\end{lstlisting}


以上代码的输出:\verb|Hello World 1234|


在上面的例子中使用了两次并置运算符,这是由于我们需要插入第三个字符串。为了分隔这两个变量,我们在 \$txt1 与 \$txt2 之间插入了一个空格。


\section{strlen()}


strlen() 函数用于计算字符串的长度,下面的示例中使用strlen()来计算出字符串 "Hello world!" 的长度:

\begin{lstlisting}[language=PHP]
<?php
  echo strlen("Hello world!");
?>
\end{lstlisting}


以上代码的输出:\verb|12|

字符串的长度信息常常用在循环或其他函数中,因为那时确定字符串何时结束是很重要的(例如,在循环中,我们需要在字符串中的最后一个字符之后结束循环)。

\section{strpos()}

strpos() 函数用于在字符串内检索一段字符串或一个字符。如果在字符串中找到匹配,该函数会返回第一个匹配的位置。如果未找到匹配,则返回 FALSE。

下面的示例演示如何在字符串中找到子字符串 "world":



\begin{lstlisting}[language=PHP]
<?php
  echo strpos("Hello world!", "world");
?>
\end{lstlisting}

以上代码的输出是:\verb|6|

在字符串"Hello world!"中,字符串 "world" 的位置是 6,至于返回 6 而不是 7,是由于字符串中的首个位置是 0,而不是 1。


\section{PHP String Functions}



\begin{longtable}{|m{120pt}|m{250pt}|m{20pt}|}
%head
\multicolumn{3}{r}{}
\tabularnewline\hline
函数	&描述	&PHP
\endhead
%endhead

%firsthead
\caption{PHP String 函数}\\
\hline
函数	&描述	&PHP
\endfirsthead
%endfirsthead

%foot
\multicolumn{3}{r}{}
\endfoot
%endfoot

%lastfoot
\endlastfoot
%endlastfoot

\hline
addcslashes()				&在指定的字符前添加反斜杠。	&4\\
\hline
addslashes()				&在指定的预定义字符前添加反斜杠。&	3\\
\hline
bin2hex()					&把 ASCII 字符的字符串转换为十六进制值。&	3\\
\hline
chop()						&rtrim() 的别名。	&3\\
\hline
chr()						&从指定的 ASCII 值返回字符。&	3\\
\hline
chunk\_split()				&把字符串分割为一连串更小的部分。&	3\\
\hline
convert\_cyr\_string()		&把字符由一种 Cyrillic 字符转换成另一种。&	3\\
\hline
convert\_uudecode()			&对 uuencode 编码的字符串进行解码。	&5\\
\hline
convert\_uuencode()			&使用 uuencode 算法对字符串进行编码。&	5\\
\hline
count\_chars()				&返回字符串所用字符的信息。	&4\\
\hline
crc32()						&计算一个字符串的 32-bit CRC。&	4\\
\hline
crypt()						&单向的字符串加密法 (hashing)。&	3\\
\hline
echo()						&输出字符串。	&3\\
\hline
explode()					&把字符串打散为数组。&	3\\
\hline
fprintf()						&把格式化的字符串写到指定的输出流。&	5\\
\hline
get\_html\_translation\_table()&返回翻译表。	&4\\
\hline
hebrev()					&把希伯来文本从右至左的流转换为左至右的流。	&3\\
\hline
hebrevc()					&同上,同时把({\textbackslash}n) 转为	<br />。	&3\\
\hline
html\_entity\_decode()		&把 HTML 实体转换为字符。	&4\\
\hline
htmlentities()				&把字符转换为 HTML 实体。&	3\\
\hline
htmlspecialchars\_decode()	&把一些预定义的 HTML 实体转换为字符。&	5\\
\hline
htmlspecialchars()			&把一些预定义的字符转换为 HTML 实体。&	3\\
\hline
implode()					&把数组元素组合为一个字符串。	&3\\
\hline
join()						&implode() 的别名。	&3\\
\hline
levenshtein()				&返回两个字符串之间的 Levenshtein 距离。&	3\\
\hline
localeconv()					&返回包含本地数字及货币信息格式的数组。&	4\\
\hline
ltrim()						&从字符串左侧删除空格或其他预定义字符。&	3\\
\hline
md5()						&计算字符串的 MD5 散列。	&3\\
\hline
md5\_file()					&计算文件的 MD5 散列。	&4\\
\hline
metaphone()				&计算字符串的 metaphone 键。&	4\\
\hline
money\_format()				&把字符串格式化为货币字符串。&	4\\
\hline
nl\_langinfo()				&返回指定的本地信息。	&4\\
\hline
nl2br()						&在字符串中的每个新行之前插入 HTML 换行符。	&3\\
\hline
number\_format()			&通过千位分组来格式化数字。	&3\\
\hline
ord()						&返回字符串第一个字符的 ASCII 值。&	3\\
\hline
parse\_str()					&把查询字符串解析到变量中。	&3\\
\hline
print()						&输出一个或多个字符串。	&3\\
\hline
printf()						&输出格式化的字符串。	&3\\
\hline
quoted\_printable\_decode()	&解码 quoted-printable 字符串。&	3\\
\hline
quotemeta()				&在字符串中某些预定义的字符前添加反斜杠。	&3\\
\hline
rtrim()						&从字符串的末端开始删除空白字符或其他预定义字符。	&3\\
\hline
setlocale()					&设置地区信息(地域信息)。	&3\\
\hline
sha1()						&计算字符串的 SHA-1 散列。	&4\\
\hline
sha1\_file()					&计算文件的 SHA-1 散列。	&4\\
\hline
similar\_text()				&计算两个字符串的匹配字符的数目。	&3\\
\hline
soundex()					&计算字符串的 soundex 键。	&3\\
\hline
sprintf()					&把格式化的字符串写写入一个变量中。	&3\\
\hline
sscanf()						&根据指定的格式解析来自一个字符串的输入。	&4\\
\hline
str\_ireplace()				&替换字符串中的一些字符。\newline(对大小写不敏感)	&5\\
\hline
str\_pad()					&把字符串填充为新的长度。	&4\\
\hline
str\_repeat()				&把字符串重复指定的次数。	&4\\
\hline
str\_replace()				&替换字符串中的一些字符。\newline(对大小写敏感)	&3\\
\hline
str\_rot13()					&对字符串执行 ROT13 编码。	&4\\
\hline
str\_shuffle()				&随机地打乱字符串中的所有字符。&	4\\
\hline
str\_split()					&把字符串分割到数组中。	&5\\
\hline
str\_word\_count()			&计算字符串中的单词数。&	4\\
\hline
strcasecmp()				&比较两个字符串。\newline(对大小写不敏感)&	3\\
\hline
strchr()						&搜索字符串在另一字符串中的第一次出现。\newline strstr() 的别名	&3\\
\hline
strcmp()					&比较两个字符串。\newline(对大小写敏感)	&3\\
\hline
strcoll()						&比较两个字符串(根据本地设置)。	&4\\
\hline
strcspn()					&返回在找到任何指定的字符之前,在字符串查找的字符数。&	3\\
\hline
strip\_tags()					&剥去 HTML、XML 以及 PHP 的标签。	&3\\
\hline
stripcslashes()				&删除由 addcslashes() 函数添加的反斜杠。&	4\\
\hline
stripslashes()				&删除由 addslashes() 函数添加的反斜杠。&	3\\
\hline
stripos()					&返回字符串在另一字符串中第一次出现的位置。\newline (大小写不敏感)	&5\\
\hline
stristr()						&查找字符串在另一字符串中第一次出现的位置。\newline (大小写不敏感)	&3\\
\hline
strlen()						&返回字符串的长度。	&3\\
\hline
strnatcasecmp()				&使用一种“自然”算法来比较两个字符串。\newline(对大小写不敏感)&	4\\
\hline
strnatcmp()					&使用一种“自然”算法来比较两个字符串。\newline(对大小写敏感)	&4\\
\hline
strncasecmp()				&前 n 个字符的字符串比较。\newline(对大小写不敏感)。	&4\\
\hline
strncmp()					&前 n 个字符的字符串比较。\newline(对大小写敏感)。&	4\\
\hline
strpbrk()					&在字符串中搜索指定字符中的任意一个。	&5\\
\hline
strpos()					&返回字符串在另一字符串中首次出现的位置。\newline(对大小写敏感)&	3\\
\hline
strrchr()					&查找字符串在另一个字符串中最后一次出现的位置。	&3\\
\hline
strrev()						&反转字符串。	&3\\
\hline
strripos()					&查找字符串在另一字符串中最后出现的位置。\newline (对大小写不敏感)	&5\\
\hline
strrpos()					&查找字符串在另一字符串中最后出现的位置。\newline (对大小写敏感)	&3\\
\hline
strspn()						&返回在字符串中包含的特定字符的数目。	&3\\
\hline
strstr()						&搜索字符串在另一字符串中的首次出现。\newline(对大小写敏感)	&3\\
\hline
strtok()						&把字符串分割为更小的字符串。	&3\\
\hline
strtolower()				&把字符串转换为小写。	&3\\
\hline
strtoupper()				&把字符串转换为大写。&	3\\
\hline
strtr()						&转换字符串中特定的字符。&	3\\
\hline
substr()						&返回字符串的一部分。	&3\\
\hline
substr\_compare()			&从指定的开始长度比较两个字符串。	&5\\
\hline
substr\_count()				&计算子串在字符串中出现的次数。	&4\\
\hline
substr\_replace()			&把字符串的一部分替换为另一个字符串。&	4\\
\hline
trim()						&从字符串的两端删除空白字符和其他预定义字符。&	3\\
\hline
ucfirst()						&把字符串中的首字符转换为大写。	&3\\
\hline
ucwords()					&把字符串中每个单词的首字符转换为大写。&	3\\
\hline
vfprintf()					&把格式化的字符串写到指定的输出流。	&5\\
\hline
vprintf()					&输出格式化的字符串。	&4\\
\hline
vsprintf()					&把格式化字符串写入变量中。&	4\\
\hline
wordwrap()					&按照指定长度对字符串进行折行处理。&	4\\
\hline
\end{longtable}


\section{PHP String Constants}





\begin{longtable}{|m{120pt}|m{250pt}|m{20pt}|}
%head
\multicolumn{3}{r}{}
\tabularnewline\hline
常量	&描述	&PHP
\endhead
%endhead

%firsthead
\caption{PHP String 常量}\\
\hline
常量	&描述	&PHP
\endfirsthead
%endfirsthead

%foot
\multicolumn{3}{r}{}
\endfoot
%endfoot

%lastfoot
\endlastfoot
%endlastfoot

\hline
CRYPT\_SALT\_LENGTH	&包含系统默认加密方法的长度。\newline 对于标准 DES 加密,长度是 2。	 &\\
\hline
CRYPT\_STD\_DES		&如果支持 2 字符 salt 的 DES 加密,则设置为 1,否则为 0。	 &\\
\hline
CRYPT\_EXT\_DES		&如果支持 9 字符 salt 的 DES 加密,则设置为 1,否则为 0。	 &\\
\hline
CRYPT\_MD5			&如果支持以$1$开始的 12 字符 salt 的MD5加密,则设置为1,否则为0。	 &\\
\hline
CRYPT\_BLOWFISH		&如果支持以 $2$ 或 $2a$ 开始的 16 字符 salt 的 Blowfish 加密,则设置为 1,否则为 0。	 &\\
\hline
HTML\_SPECIALCHARS	& 	 &\\
\hline
HTML\_ENTITIES	 	 	&&\\
\hline
ENT\_COMPAT	 	 	&&\\
\hline
ENT\_QUOTES	 	 	&&\\
\hline
ENT\_NOQUOTES	 	& &\\
\hline
CHAR\_MAX	 	 		&&\\
\hline
LC\_CTYPE	 	 		&&\\
\hline
LC\_NUMERIC	 	 	&&\\
\hline
LC\_TIME	 	 		&&\\
\hline
LC\_COLLATE	 	 	&&\\
\hline
LC\_MONETARY	 	 	&&\\
\hline
LC\_ALL	 	 			&&\\
\hline
LC\_MESSAGES	 	 	&&\\
\hline
STR\_PAD\_LEFT	 	 	&&\\
\hline
STR\_PAD\_RIGHT	 	& &\\
\hline
STR\_PAD\_BOTH	 	&&\\
\hline
\end{longtable}



\chapter{PHP Operators}

PHP的运算符包括\verb|+ - * / > < >= <=|等,与C++十分类似。


\section{Arithmetic Operators}

\begin{longtable}{|m{35pt}|m{180pt}|m{80pt}|m{30pt}|}
%head
\multicolumn{4}{r}{}
\tabularnewline\hline
运算符	&说明	&示例	&结果
\endhead
%endhead

%firsthead
\caption{PHP 算术运算符}\\
\hline
运算符	&说明	&示例	&结果
\endfirsthead
%endfirsthead

%foot
\multicolumn{4}{r}{}
\endfoot
%endfoot

%lastfoot
\endlastfoot
%endlastfoot
\hline
+	&Addition		&x=2 \newline x+2		&4\\
\hline
-	&Subtraction	&x=2 \newline 5-x		&3\\
\hline
*	&Multiplication	&x=4 \newline x*5		&20\\
\hline
/	&Division		&15/5 \newline 5/2		&3 \newline 2.5\\
\hline
\%	&Modulus (division remainder)	&5\%2 \newline 10\%8 \newline 10\%2	&1 \newline 2 \newline 0\\
\hline
++	&Increment		& x=5 \newline x++	 &x=6\\
\hline
-\/-	&Decrement	&x=5 \newline x-\/-	&x=4\\
\hline

\end{longtable}



\section{Assignment Operators}

\begin{longtable}{|m{35pt}|m{180pt}|m{80pt}|m{30pt}|}
%head
\multicolumn{4}{r}{}
\tabularnewline\hline
运算符	&说明	&示例	&结果
\endhead
%endhead

%firsthead
\caption{PHP 赋值运算符}\\
\hline
运算符	&说明	&示例	&结果
\endfirsthead
%endfirsthead

%foot
\multicolumn{4}{r}{}
\endfoot
%endfoot

%lastfoot
\endlastfoot
%endlastfoot
\hline
=		&x=y		&x=y&\\
\hline
+\/=	&x+\/=y	&x=x+y&\\
\hline
-\/=		&x-\/=y		&x=x-y&\\
\hline
*\/=	&x*\/=y	&x=x*y&\\
\hline
/\/=		&x/\/=y		&x=x/y&\\
\hline
.\/=		&x.\/=y		&x=x.y&\\
\hline
\%\/=	&x\%\/=y	&x=x\%y&\\
\hline
\end{longtable}



\section{Comparison Operators}


\begin{longtable}{|m{35pt}|m{180pt}|m{80pt}|m{30pt}|}
%head
\multicolumn{4}{r}{}
\tabularnewline\hline
运算符	&说明	&示例	&结果
\endhead
%endhead

%firsthead
\caption{PHP 比较运算符}\\
\hline
运算符	&说明	&示例	&结果
\endfirsthead
%endfirsthead

%foot
\multicolumn{4}{r}{}
\endfoot
%endfoot

%lastfoot
\endlastfoot
%endlastfoot
\hline
=\/=	&is equal to					&5==8 returns false&false\\
\hline
!\/=	&is not equal					&5!=8 returns true&true\\
\hline
>	&is greater than					&5>8 returns false&false\\
\hline
<	&is less than					&5<8 returns true&true\\
\hline
>\/=	&is greater than or equal to &5>=8 returns false&false\\
\hline
<\/=	&is less than or equal to	&5<=8 returns true&true\\
\hline

\end{longtable}


\section{Logical Operators}



\begin{longtable}{|m{35pt}|m{60pt}|m{200pt}|m{30pt}|}
%head
\multicolumn{4}{r}{}
\tabularnewline\hline
运算符	&说明	&示例	&结果
\endhead
%endhead

%firsthead
\caption{PHP 逻辑运算符}\\
\hline
运算符	&说明	&示例	&结果
\endfirsthead
%endfirsthead

%foot
\multicolumn{4}{r}{}
\endfoot
%endfoot

%lastfoot
\endlastfoot
%endlastfoot
\hline
\&\&	&and	 				&x=6 \newline y=3 \newline (x < 10 \&\& y > 1) returns true&true\\
\hline
||		&or	 					& x=6 \newline y=3 \newline (x==5 || y==5) returns false&false\\
\hline
!		&not	 				& x=6 \newline y=3 \newline !(x==y) returns true		&true\\
\hline
\end{longtable}





\chapter{PHP Statements}

if、elseif 以及 else 语句用于执行基于不同条件的不同动作。



\section{Conditional statements}

编写代码时,常常需要为不同的判断执行不同的动作,这时可以在代码中使用条件语句来完成此任务。

\begin{compactitem}
\item if...else

在条件成立时执行一块代码,条件不成立时执行另一块代码

\item elseif

与 if...else 配合使用,在若干条件之一成立时执行一个代码块
\end{compactitem}



\subsection{if...else statements}

如果希望在某个条件成立时执行一些代码,在条件不成立时执行另一些代码,使用 if....else 语句。

\begin{lstlisting}[language=PHP]
if (condition)
  code to be executed if condition is true;
else
  code to be executed if condition is false; 
\end{lstlisting}


如果当前日期是周五,下面的代码将输出 "Have a nice weekend!",否则会输出 "Have a nice day!":


\begin{lstlisting}[language=PHP]
<!DOCTYPE html>
<html>
<head>
<title></title>
</head>
<body>

<?php
$d=date("D");
if ($d=="Fri")
  echo "Have a nice weekend!"; 
else
  echo "Have a nice day!"; 
?>

</body>
</html>
\end{lstlisting}

如果需要在条件成立或不成立时执行多行代码,应该把这些代码行包括在花括号中:

\begin{lstlisting}[language=PHP]
<!DOCTYPE html>
<html>
<head>
<title></title>
</head>
<body>

<?php
$d=date("D");
if ($d=="Fri")
  {
  echo "Hello!<br />"; 
  echo "Have a nice weekend!";
  echo "See you on Monday!";
  }
?>

</body>
</html>
\end{lstlisting}


\subsection{elseif statements}


如果希望在多个条件之一成立时执行代码,使用 elseif 语句:

\begin{lstlisting}[language=PHP]
if (condition)
  code to be executed if condition is true;
elseif (condition)
  code to be executed if condition is true;
else
  code to be executed if condition is false; 
\end{lstlisting}

如果当前日期是周五,下面的例子会输出 "Have a nice weekend!",如果是周日,则输出 "Have a nice Sunday!",否则输出 "Have a nice day!":

\begin{lstlisting}[language=PHP]
<!DOCTYPE html>
<html>
<head>
<title></title>
</head>
<body>

<?php
$d=date("D");
if ($d=="Fri")
  echo "Have a nice weekend!"; 
elseif ($d=="Sun")
  echo "Have a nice Sunday!"; 
else
  echo "Have a nice day!"; 
?>

</body>
</html>
\end{lstlisting}



\section{Select statements}

PHP 中的switch 语句用于执行基于多个不同条件的不同动作,通过switch语句可以可以避免冗长的 if..elseif..else 代码块,从而有选择地执行若干代码块之一。


\subsection{switch...case...default statements}



\begin{lstlisting}[language=PHP]
switch (expression)
{
case label1:
  code to be executed if expression = label1;
  break;  
case label2:
  code to be executed if expression = label2;
  break;
default:
  code to be executed
  if expression is different 
  from both label1 and label2;
}
\end{lstlisting}

switch语句的工作原理如下:

\begin{compactenum}
\item 对表达式(通常是变量)进行一次计算
\item 把表达式的值与结构中 case 的值进行比较
\item 如果存在匹配,则执行与 case 关联的代码
\item 代码执行后,break 语句阻止代码跳入下一个 case 中继续执行
\item 如果没有 case 为真,则使用 default 语句
\end{compactenum}


\begin{lstlisting}[language=PHP]
<!DOCTYPE html>
<html>
<head>
<title></title>
</head>
<body>
<?php
switch ($x)
{
case 1:
  echo "Number 1";
  break;
case 2:
  echo "Number 2";
  break;
case 3:
  echo "Number 3";
  break;
default:
  echo "No number between 1 and 3";
}
?>

</body>
</html>
\end{lstlisting}

\section{Loop statements}



在编写代码时,经常需要让相同的代码块运行很多次,可以在代码中使用循环语句来完成这个任务。


PHP 中的循环语句用于执行相同的代码块指定的次数,循环语句的种类包括:


\begin{compactitem}
\item while - 只要指定的条件成立,则循环执行代码块
\item do...while - 首先执行一次代码块,然后在指定的条件成立时重复这个循环
\item for - 循环执行代码块指定的次数
\item foreach - 根据数组中每个元素来循环代码块
\end{compactitem}



\subsection{while statements}

只要指定的条件成立,while 语句将重复执行代码块。


\begin{lstlisting}[language=PHP]
while (condition)
code to be executed;
\end{lstlisting}

下面的例子示范了一个循环,只要变量 i 小于或等于 5,代码就会一直循环执行下去。循环每循环一次,变量就会递增 1:

\begin{lstlisting}[language=PHP]
<!DOCTYPE html>
<html>
<head>
<title></title>
</head>
<body>

<?php 
$i=1;
while($i<=5)
  {
  echo "The number is " . $i . "<br />";
  $i++;
  }
?>

</body>
</html>
\end{lstlisting}


\subsection{do...while statements}

do...while 语句会至少执行一次代码 - 然后,只要条件成立,就会重复进行循环。

\begin{lstlisting}[language=PHP]
do
{
  code to be executed;
}
while (condition); 
\end{lstlisting}

下面的例子将对 i 的值进行一次累加,然后,只要 i 小于 5 的条件成立,就会继续累加下去:

\begin{lstlisting}[language=PHP]
<!DOCTYPE html>
<html>
<head>
<title></title>
</head>
<body>

<?php 
$i=0;
do {
  $i++;
  echo "The number is " . $i . "<br />";
}
while ($i<5);
?>

</body>
</html>
\end{lstlisting}

\subsection{for statements}

如果已经确定了代码块的重复执行次数,则可以使用 for 语句。

\begin{lstlisting}[language=PHP]
for (initialization; condition; increment)
{
  code to be executed;
}
\end{lstlisting}

for 语句有三个参数。第一个参数初始化变量,第二个参数保存条件,第三个参数包含执行循环所需的增量。如果 initialization 或 increment 参数中包括了多个变量,需要用逗号进行分隔。而条件必须计算为 true 或者 false。

\begin{lstlisting}[language=PHP]
for (initialization; condition; increment)
{
  code to be executed;
}
\end{lstlisting}

下面的例子会把文本 "Hello World!" 显示 5 次:

\begin{lstlisting}[language=PHP]
<!DOCTYPE html>
<html>
<head>
<title></title>
</head>
<body>

<?php
for ($i=1; $i<=5; $i++)
{
  echo "Hello World!<br />";
}
?>

</body>
</html>
\end{lstlisting}

\subsection{foreach statements}

foreach 语句用于循环遍历数组。每进行一次循环,当前数组元素的值就会被赋值给 value 变量(数组指针会逐一地移动) - 以此类推。


\begin{lstlisting}[language=PHP]
foreach (array as value)
{
    code to be executed;
}
\end{lstlisting}

下面的例子示范了一个循环,这个循环可以输出给定数组的值:

\begin{lstlisting}[language=PHP]
<!DOCTYPE html>
<html>
<head>
<title></title>
</head>
<body>

<?php
$arr=array("one", "two", "three");

foreach ($arr as $value)
{
  echo "Value: " . $value . "<br />";
}
?>

</body>
</html>
\end{lstlisting}





\chapter{PHP Array}



在使用 PHP 进行开发的过程中,或早或晚,都会需要创建许多相似的变量,通过PHP数组就能够在单独的变量名中存储一个或多个值。

在PHP中,定义数组会用到array关键字,同时数组是可以定义索引的,方便快捷查询。

数组中的元素都有自己的 ID,因此可以方便地访问它们,PHP有三种数组类型:

\begin{compactitem}
\item 数值数组 - 带有数字 ID 键的数组

\item 关联数组 - 数组中的每个 ID 键关联一个值

\item 多维数组 - 包含一个或多个数组的数组
\end{compactitem}






\section{Numeric array}

数值数组存储的每个元素都带有一个数字 ID 键,可以使用不同的方法来创建数值数组:

\begin{compactenum}
\item[I] 自动分配 ID 键

\begin{lstlisting}[language=PHP]
$names = array("Peter","Quagmire","Joe");
\end{lstlisting}

\item[II] 人工分配ID 键

\begin{lstlisting}[language=PHP]
$names[0] = "Peter";
$names[1] = "Quagmire";
$names[2] = "Joe";
\end{lstlisting}

可以在脚本中使用这些 ID 键:


\begin{lstlisting}[language=PHP]
<?php

$names[0] = "Peter";
$names[1] = "Quagmire";
$names[2] = "Joe";

echo $names[1] . " and " . $names[2] . " are ". $names[0] . "'s neighbors";
?>
\end{lstlisting}


\end{compactenum}





\section{Associative array}


关联数组,它的每个 ID 键都关联一个值。在存储有关具体命名的值的数据时,使用数值数组不是最好的做法。

通过关联数组,我们可以把值作为键,并向它们赋值。

在下面的示例中,使用一个数组把年龄分配给不同的人:


\begin{lstlisting}[language=PHP]
$ages = array("Peter"=>32, "Quagmire"=>30, "Joe"=>34);
\end{lstlisting}

本例与上面相同,不过展示了另一种创建数组的方法:

\begin{lstlisting}[language=PHP]
$ages['Peter'] = "32";
$ages['Quagmire'] = "30";
$ages['Joe'] = "34";
\end{lstlisting}



可以在脚本中使用 ID 键:

\begin{lstlisting}[language=PHP]
<?php

$ages['Peter'] = "32";
$ages['Quagmire'] = "30";
$ages['Joe'] = "34";

echo "Peter is " . $ages['Peter'] . " years old.";
?>
\end{lstlisting}

\section{Multidimensional array}



在多维数组中,主数组中的每个元素也是一个数组。在子数组中的每个元素也可以是数组,以此类推。


下面的示例中创建了一个带有自动分配的 ID 键的多维数组:


\begin{lstlisting}[language=PHP]
$families = array
(
  "Griffin"=>array
  (
  "Peter",
  "Lois",
  "Megan"
  ),
  "Quagmire"=>array
  (
  "Glenn"
  ),
  "Brown"=>array
  (
  "Cleveland",
  "Loretta",
  "Junior"
  )
);
\end{lstlisting}

如果输出这个数组的话,应该类似这样:


\begin{lstlisting}[language=PHP]
Array
(
[Griffin] => Array
  (
  [0] => Peter
  [1] => Lois
  [2] => Megan
  )
[Quagmire] => Array
  (
  [0] => Glenn
  )
[Brown] => Array
  (
  [0] => Cleveland
  [1] => Loretta
  [2] => Junior
  )
)
\end{lstlisting}

如果要显示上面的数组中的一个单一的值:


\begin{lstlisting}[language=PHP]
echo "Is " . $families['Griffin'][2] . " a part of the Griffin family?"; 
\end{lstlisting}





\section{PHP Array Functions}


PHP array 函数允许用户对数组进行操作,而且PHP 支持单维和多维的数组,同时提供了用数据库查询结果来构造数组的函数。

PHP array 函数是 PHP 核心的组成部分,无需安装即可使用这些函数。



\begin{longtable}{|m{120pt}|m{250pt}|m{20pt}|}
%head
\multicolumn{3}{r}{}
\tabularnewline\hline
函数	&描述	&PHP
\endhead
%endhead

%firsthead
\caption{PHP Array 函数}\\
\hline
函数	&描述	&PHP
\endfirsthead
%endfirsthead

%foot
\multicolumn{3}{r}{}
\endfoot
%endfoot

%lastfoot
\endlastfoot
%endlastfoot

\hline
array()							&创建数组。	&3\\
\hline
array\_change\_key\_case()		&返回其键均为大写或小写的数组。	&4\\
\hline
array\_chunk()					&把一个数组分割为新的数组块。	&4\\
\hline
array\_combine()				&通过合并两个数组来创建一个新数组。	&5\\
\hline
array\_count\_values()			&用于统计数组中所有值出现的次数。	&4\\
\hline
array\_diff()						&返回两个数组的差集数组。	&4\\
\hline
array\_diff\_assoc()				&比较键名和键值,并返回两个数组的差集数组。	&4\\
\hline
array\_diff\_key()				&比较键名,并返回两个数组的差集数组。	&5\\
\hline
array\_diff\_uassoc()			&通过用户提供的回调函数做索引检查来计算数组的差集。	&5\\
\hline
array\_diff\_ukey()				&用回调函数对键名比较计算数组的差集。	&5\\
\hline
array\_fill()						&用给定的值填充数组。	&4\\
\hline
array\_filter()					&用回调函数过滤数组中的元素。	&4\\
\hline
array\_flip()						&交换数组中的键和值。	&4\\
\hline
array\_intersect()				&计算数组的交集。	&4\\
\hline
array\_intersect\_assoc()		&比较键名和键值,并返回两个数组的交集数组。	&4\\
\hline
array\_intersect\_key()			&使用键名比较计算数组的交集。	&5\\
\hline
array\_intersect\_uassoc()		&带索引检查计算数组的交集,用回调函数比较索引。	&5\\
\hline
array\_intersect\_ukey()			&用回调函数比较键名来计算数组的交集。	&5\\
\hline
array\_key\_exists()				&检查给定的键名或索引是否存在于数组中。&	4\\
\hline
array\_keys()					&返回数组中所有的键名。	&4\\
\hline
array\_map()					&将回调函数作用到给定数组的单元上。	&4	\\
\hline
array\_merge()					&把一个或多个数组合并为一个数组。	&4\\
\hline
array\_merge\_recursive()		&递归地合并一个或多个数组。	&4\\
\hline
array\_multisort()				&对多个数组或多维数组进行排序。&	4\\
\hline
array\_pad()					&用值将数组填补到指定长度。	&4\\
\hline
array\_pop()					&将数组最后一个单元弹出(出栈)。&	4\\
\hline
array\_product()				&计算数组中所有值的乘积。	&5\\
\hline
array\_push()					&将一个或多个单元(元素)压入数组的末尾(入栈)。	&4\\
\hline
array\_rand()					&从数组中随机选出一个或多个元素,并返回。	&4\\
\hline
array\_reduce()					&用回调函数迭代地将数组简化为单一的值。	&4\\
\hline
array\_reverse()				&将原数组中的元素顺序翻转,创建新的数组并返回。	&4\\
\hline
array\_search()					&在数组中搜索给定的值,如果成功则返回相应的键名。	&4\\
\hline
array\_shift()					&删除数组中的第一个元素,并返回被删除元素的值。	&4\\
\hline
array\_slice()					&在数组中根据条件取出一段值,并返回。	&4\\
\hline
array\_splice()					&把数组中的一部分去掉并用其它值取代。	&4\\
\hline
array\_sum()					&计算数组中所有值的和。	&4\\
\hline
array\_udiff()					&用回调函数比较数据来计算数组的差集。	&5\\
\hline
array\_udiff\_assoc()			&带索引检查计算数组的差集,用回调函数比较数据。	&5\\
\hline
array\_udiff\_uassoc()			&带索引检查计算数组的差集,用回调函数比较数据和索引。	&5\\
\hline
array\_uintersect()				&计算数组的交集,用回调函数比较数据。	&5\\
\hline
array\_uintersect\_assoc()		&带索引检查计算数组的交集,用回调函数比较数据。	&5\\
\hline
array\_uintersect\_uassoc()		&带索引检查计算数组的交集,用回调函数比较数据和索引。	&5\\
\hline
array\_unique()					&删除数组中重复的值。	&4\\
\hline
array\_unshift()					&在数组开头插入一个或多个元素。	&4\\
\hline
array\_values()					&返回数组中所有的值。	&4\\
\hline
array\_walk()					&对数组中的每个成员应用用户函数。	&3\\
\hline
array\_walk\_recursive()		&对数组中的每个成员递归地应用用户函数。	&5\\
\hline
arsort()							&对数组进行逆向排序并保持索引关系。	&3\\
\hline
asort()						&对数组进行排序并保持索引关系。	&3\\
\hline
compact()					&建立一个数组,包括变量名和它们的值。&	4\\
\hline
count()						&计算数组中的元素数目或对象中的属性个数。	&3\\
\hline
current()					&返回数组中的当前元素。	&3\\
\hline
each()						&返回数组中当前的键/值对并将数组指针向前移动一步。	&3\\
\hline
end()						&将数组的内部指针指向最后一个元素。	&3\\
\hline
extract()					&从数组中将变量导入到当前的符号表。	&3\\
\hline
in\_array()					&检查数组中是否存在指定的值。	&4\\
\hline
key()						&从关联数组中取得键名。	&3\\
\hline
krsort()						&对数组按照键名逆向排序。	&3\\
\hline
ksort()						&对数组按照键名排序。	&3\\
\hline
list()						&把数组中的值赋给一些变量。&	3\\
\hline
natcasesort()				&用“自然排序”算法对数组进行不区分大小写字母的排序。	&4\\
\hline
natsort()					&用“自然排序”算法对数组排序。	&4\\
\hline
next()						&将数组中的内部指针向前移动一位。&	3\\
\hline
pos()						&current() 的别名。	&3\\
\hline
prev()						&将数组的内部指针倒回一位。&	3\\
\hline
range()						&建立一个包含指定范围的元素的数组。	&3\\
\hline
reset()						&将数组的内部指针指向第一个元素。	&3\\
\hline
rsort()						&对数组逆向排序。	&3\\
\hline
shuffle()					&把数组中的元素按随机顺序重新排列。	&3\\
\hline
sizeof()						&count() 的别名。	&3\\
\hline
sort()						&对数组排序。	&3\\
\hline
uasort()						&使用用户自定义的比较函数对数组中的值进行排序并保持索引关联。	&3\\
\hline
uksort()						&使用用户自定义的比较函数对数组中的键名进行排序。	&3\\
\hline
usort()						&使用用户自定义的比较函数对数组中的值进行排序。	&3\\
\hline
\end{longtable}



\section{PHP Array Constants}




\begin{longtable}{|m{120pt}|m{250pt}|m{20pt}|}
%head
\multicolumn{3}{r}{}
\tabularnewline\hline
常量	&描述	&PHP
\endhead
%endhead

%firsthead
\caption{PHP Array 常量}\\
\hline
常量	&描述	&PHP
\endfirsthead
%endfirsthead

%foot
\multicolumn{3}{r}{}
\endfoot
%endfoot

%lastfoot
\endlastfoot
%endlastfoot

\hline

CASE\_LOWER	&用在 array\_change\_key\_case() 中将数组键名转换成小写字母。&	 \\
\hline
CASE\_UPPER	&用在 array\_change\_key\_case() 中将数组键名转换成大写字母。&	 \\
\hline
SORT\_ASC		&用在 array\_multisort() 函数中,使其升序排列。	 &\\
\hline
SORT\_DESC		&用在 array\_multisort() 函数中,使其降序排列。	 &\\
\hline
SORT\_REGULAR	&用于对对象进行通常比较。	 &\\
\hline
SORT\_NUMERIC	&用于对对象进行数值比较。	 &\\
\hline
SORT\_STRING	&用于对对象进行字符串比较。	 &\\
\hline
SORT\_LOCALE\_STRING	&基于当前区域来对对象进行字符串比较。	&4\\
\hline
COUNT\_NORMAL	 	& &\\
\hline
COUNT\_RECURSIVE	 	& &\\
\hline
EXTR\_OVERWRITE	 	& &\\
\hline
EXTR\_SKIP	 	 &&\\
\hline
EXTR\_PREFIX\_SAME	 	& &\\
\hline
EXTR\_PREFIX\_ALL	 	& &\\
\hline
EXTR\_PREFIX\_INVALID	& 	 &\\
\hline
EXTR\_PREFIX\_IF\_EXISTS	& 	 &\\
\hline
EXTR\_IF\_EXISTS	 	 &&\\
\hline
EXTR\_REFS	 	 &&\\
\hline
\end{longtable}










\bibliographystyle{plainnat}
\bibliography{phpnotes}






































