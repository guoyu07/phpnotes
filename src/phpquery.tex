\part{phpQuery}

\chapter{Overview}


phpQuery is a server-side, chainable, CSS3 selector driven Document Object Model (DOM) API based on jQuery JavaScript Library.

phpQuery(\url{http://github.com/TobiaszCudnik/phpquery})是一个服务器端运行的链式的、CSS3选择器驱动的DOM解析工具。

phpQuery参考了jQuery的语法,基于PHP5开发,并且提供了额外的命令行工具。





\begin{lstlisting}[language=PHP]
$ pear channel-discover phpquery-pear.appspot.com 
$ pear install phpquery/phpQuery
\end{lstlisting}


使用phpQuery采集网页的示例如下:


\begin{lstlisting}[language=bash]
$ phpquery 'http://code.google.com/p/phpquery/downloads/list?can=1'  --find '.vt.col_4 a' --contents --getString null array_sum
\end{lstlisting}



phpQuery依赖PHP的DOM扩展


\begin{lstlisting}[language=PHP]
<?php
require('phpQuery/phpQuery.php');
// INITIALIZE IT
// phpQuery::newDocumentHTML($markup);
// phpQuery::newDocumentXML();
// phpQuery::newDocumentFileXHTML('test.html');
// phpQuery::newDocumentFilePHP('test.php');
// phpQuery::newDocument('test.xml', 'application/rss+xml');
// this one defaults to text/html in utf8
$doc = phpQuery::newDocument('<div/>');
// FILL IT
// array syntax works like ->find() here
$doc['div']->append('<ul></ul>');
// array set changes inner html
$doc['div ul'] = '<li>1</li> <li>2</li> <li>3</li>';
// MANIPULATE IT
$li = null;
// almost everything can be a chain
$doc['ul > li']
	->addClass('my-new-class')
	->filter(':last')
		->addClass('last-li')
// save it anywhere in the chain
		->toReference($li);
// SELECT DOCUMENT
// pq(); is using selected document as default
phpQuery::selectDocument($doc);
// documents are selected when created or by above method
// query all unordered lists in last selected document
$ul = pq('ul')->insertAfter('div');
// ITERATE IT
// all direct LIs from $ul
foreach($ul['> li'] as $li) {
	// iteration returns PLAIN dom nodes, NOT phpQuery objects
	$tagName = $li->tagName;
	$childNodes = $li->childNodes;
	// so you NEED to wrap it within phpQuery, using pq();
	pq($li)->addClass('my-second-new-class');
}
// PRINT OUTPUT
// 1st way
print phpQuery::getDocument($doc->getDocumentID());
// 2nd way
print phpQuery::getDocument(pq('div')->getDocumentID());
// 3rd way
print pq('div')->getDocument();
// 4th way
print $doc->htmlOuter();
// 5th way
print $doc;
// another...
print $doc['ul'];
\end{lstlisting}

\section{Basics}





\begin{lstlisting}[language=PHP]
phpQuery::newDocumentFileXHTML('my-xhtml.html')->find('p'); $ul = pq('ul');
\end{lstlisting}


\subsection{Loading Document}


\begin{compactitem}
\item \texttt{phpQuery::newDocument(\$html, \$contentType = null)}

Creates new document from markup. If no \$contentType, autodetection is made (based on markup). If it fails, text/html in utf-8 is used.
\item \texttt{phpQuery::newDocumentFile(\$file, \$contentType = null)} 

Creates new document from file. Works like newDocument()
\item \texttt{phpQuery::newDocumentHTML(\$html, \$charset = 'utf-8')}

\item \texttt{phpQuery::newDocumentXHTML(\$html, \$charset = 'utf-8')}

\item \texttt{phpQuery::newDocumentXML(\$html, \$charset = 'utf-8')}
\item \texttt{phpQuery::newDocumentPHP(\$html, \$contentType = null)}

Read more about it on PHPSupport page
\item \texttt{phpQuery::newDocumentFileHTML(\$file, \$charset = 'utf-8')}

\item \texttt{phpQuery::newDocumentFileXHTML(\$file, \$charset = 'utf-8')}
\item \texttt{phpQuery::newDocumentFileXML(\$file, \$charset = 'utf-8')}

\item \texttt{phpQuery::newDocumentFilePHP(\$file, \$contentType)}
\end{compactitem}


\subsection{pq() function}





\begin{lstlisting}[language=PHP]
pq($param, $context = null);
\end{lstlisting}

pq()类似于jQuery的\$(),pq()支持三种类型的操作。

\begin{compactenum}
\item Importing markup

\begin{lstlisting}[language=PHP]
// Import into selected document: 
// doesn't accept text nodes at beginning of input string 
pq('<div/>') 
// Import into document with ID from $pq->getDocumentID(): 
pq('<div/>', $pq->getDocumentID()) 
// Import into same document as DOMNode belongs to: 
pq('<div/>', DOMNode) 
// Import into document from phpQuery object: 
pq('<div/>', $pq)
\end{lstlisting}

\item Running queries

\begin{lstlisting}[language=PHP]
// Run query on last selected document: 
pq('div.myClass') 
// Run query on document with ID from $pq->getDocumentID(): 
pq('div.myClass', $pq->getDocumentID()) 
// Run query on same document as DOMNode belongs to and use node(s)as root for query: 
pq('div.myClass', DOMNode) 
// Run query on document from phpQuery object 
// and use object's stack as root node(s) for query: 
pq('div.myClass', $pq)
\end{lstlisting}

\item Wrapping DOMNodes with phpQuery objects

\begin{lstlisting}[language=PHP]
foreach(pq('li') as $li) // $li is pure DOMNode, change it to phpQuery object pq($li);
\end{lstlisting}

\end{compactenum}




\section{jQuery}


phpQuery提供了很多和jQuery类似的特性,例如:

\begin{compactitem}
\item htmlOuter() 
\item xml() 
\item xmlOuter() 
\item markup() 
\item markupOuter() 
\item getString() 
\item reverse() 
\item contentsUnwrap() 
\item switchWith() 
\item all from PHPSupport 
\item all from Basic 
\item all from MultiDocumentSupport
\end{compactitem}




\subsection{Selectors}




\begin{lstlisting}[language=bash]

\end{lstlisting}


\subsection{Attributes}


\begin{lstlisting}[language=bash]

\end{lstlisting}


\subsection{Traversing}



\begin{lstlisting}[language=bash]

\end{lstlisting}


\subsection{Manipulation}



\begin{lstlisting}[language=bash]

\end{lstlisting}


\subsection{Ajax}



\begin{lstlisting}[language=bash]

\end{lstlisting}


\subsection{Events}



\begin{lstlisting}[language=bash]

\end{lstlisting}

\subsection{Utilities}


\begin{lstlisting}[language=bash]

\end{lstlisting}


\subsection{Plugin}


\begin{lstlisting}[language=bash]

\end{lstlisting}




\begin{lstlisting}[language=bash]

\end{lstlisting}




\begin{lstlisting}[language=bash]

\end{lstlisting}




\begin{lstlisting}[language=bash]

\end{lstlisting}




\begin{lstlisting}[language=bash]

\end{lstlisting}





\begin{lstlisting}[language=bash]

\end{lstlisting}





\begin{lstlisting}[language=bash]

\end{lstlisting}





\begin{lstlisting}[language=bash]

\end{lstlisting}





\begin{lstlisting}[language=bash]

\end{lstlisting}



\begin{lstlisting}[language=bash]

\end{lstlisting}


\chapter{QueryList}


\section{Overview}


如果需要抓取一个网页的内容,但是只需要特定部分的信息,通常会用正则来解决。

虽然正则是一个通用解决方案,不过特定情况下,往往有更简单快 捷的方法。例如,为了查询一个编程方面的问题,除了可以使用Google,stackoverflow 作为一个专业的编程问答社区,会提供给你更多,更靠谱的答案。

不推荐在HTML页面使用正则,如果HTML页面过大,正则的效率无法保证,这种情况下可以考虑使用phpQuery来获取某个特定元素。

phpQuery是一个用php实现的类似jQuery的开源项目,可以在服务器端以jQuery的语法形式解析网页元素。

phpQuery支持使用类似jQuery的选择器,而且phpQuery可以对DOM进行任何复杂的操作。

QueryList相当于phpQuery的子集,可以发挥phpQuery采集方面的强大功能。

作为一个基于phpQuery的PHP通用列表采集类,QueryList让使用PHP采集网页信息时和使用jQuery选择元素一样简单。

\begin{compactitem}
\item QueryList只有一个核心的API——静态方法Query
\item QueryList支持使用jQuery选择器来选择页面元素
\item QueryList内置的过滤功能可以清洗无用的内容数据
\item QueryList支持扩展来实现多线程批量采集,模拟登陆采集等功能
\item QueryList采集结果直接以采集规则并按照列表的形式有序的返回
\item QueryList支持无限层级嵌套采集
\end{compactitem}

QueryList3使用PSR-4和Composer,不兼容以前的版本。

\begin{lstlisting}[language=bash]
$ composer require jaeger/querylist
\end{lstlisting}

在使用Composer安装了QueryList之后,只需要引入vendor/autoload.php文件就可以使用QueryList及其所有插件(如果安装了插件的话)。




\begin{lstlisting}[language=PHP]
<?php
require 'vendor/autoload.php';

use QL/QueryList;

$hj = QueryList::Query('http://mobile.csdn.net/', array("url"=>array('.unit h1 a', 'href')));

$data = $hj->getData(function($x){
   return $x['url'];
});
print_r($data);
\end{lstlisting}

如果手动下载phpQuery.php和QueryList.php,可以在代码中使用require引入。


\begin{lstlisting}[language=PHP]
<?php
require 'phpQuery.php';
require 'QueryList.php';

use QL\QueryList;

$hj = QueryList::Query('http://mobile.csdn.net/',array("url"=>array('.unit h1 a','href')));
$data = $hj->getData(function($x){
    return $x['url'];
});
print_r($data);
\end{lstlisting}


同理,QueryList扩展也可以手动安装。例如,假设QueryList所在目录为/path/to/QueryList/,那么扩展目录就是/path/to/QueryList/Ext/,然后手动引入需要用到的插件文件就可以运行插件。

所有的QueryList扩展都依赖一个基类AQuery,这个文件也存放在扩展目录(即/path/to/QueryList/Ext/)下。

QueryList扩展所依赖的类库存放目录为/path/to/QueryList/Ext/Lib/。

所有的QueryList扩展通过QueryList::run()方法运行。

QueryList扩展可以根据需要选择性地安装:


\begin{compactitem}
\item Request网络操作扩展

\begin{lstlisting}[language=bash]
$ composer require jaeger/querylist-ext-request
\end{lstlisting}

\item Multi多线程扩展

\begin{lstlisting}[language=bash]
$ composer require jaeger/querylist-ext-multi
\end{lstlisting}

\item Login模拟登录扩展

\begin{lstlisting}[language=bash]
$ composer require jaeger/querylist-ext-login
\end{lstlisting}
\end{compactitem}

包含QueyListy以及扩展的完整配置如下:

\begin{lstlisting}[language=JavaScript]
{
    "require": {
        "jaeger/querylist": "^3.1",
        "jaeger/querylist-ext-request":"^1.0",
        "jaeger/querylist-ext-multi":"^1.0",
        "jaeger/querylist-ext-login":"^1.0"
    }
}
\end{lstlisting}

以下类库是上面扩展的依赖,安装扩展的时候会自动安装,也可以选择单独引入使用:

\begin{compactitem}
\item Http类

\begin{lstlisting}[language=bash]
$ composer require jaeger/http
\end{lstlisting}

\item CurlMulti多线程类

\begin{lstlisting}[language=bash]
$ composer require jaeger/curlmulti
\end{lstlisting}

\end{compactitem}

QueryList支持遵循当前使用的PHP框架的引入类库的规则来加入到不同的框架中。

\subsection{Login}

QueryList的Login扩展可以实现模拟登陆然后采集。

\subsection{Multi}

QueryList的Multi扩展支持多线程(多进程)采集。


\subsection{Request}

QueryList的Request扩展可以实现携带cookie、伪造来路等任意复杂的网络请求。






\begin{lstlisting}[language=bash]

\end{lstlisting}




\begin{lstlisting}[language=bash]

\end{lstlisting}




\begin{lstlisting}[language=bash]

\end{lstlisting}




\begin{lstlisting}[language=bash]

\end{lstlisting}




\begin{lstlisting}[language=bash]

\end{lstlisting}





\begin{lstlisting}[language=bash]

\end{lstlisting}





\begin{lstlisting}[language=bash]

\end{lstlisting}





\begin{lstlisting}[language=bash]

\end{lstlisting}
