\part{Laravel}



\chapter{Overview}


\section{Framework}


框架可以避免重新造轮子来构建 Web 应用,而且框架通常都抽象了许多底层常用的逻辑,并提供了简便的方法来完成常见的任务。

虽然并不一定要在每个项目中都使用框架,因为有时候原生的 PHP 才是正确的选择,但是如果需要一个框架,那么有如下三种主要类型:

\begin{compactitem}
\item 微型框架
\item 全栈框架
\item 组件框架
\end{compactitem}

微型框架基本上是一个封装的路由,用来转发 HTTP 请求至一个闭包、控制器或方法等,尽可能地加快开发的速度,有时还会使用一些类库来帮助开发(例如基本的数据库封装等),因此微型框架适合用来构建HTTP的服务。

在微型框架的功能齐备后往往演化为全栈框架,这些全栈框架通常会提供 ORMs和身份认证扩展包等。

组件框架是多个独立的类库所结合起来的,而且不同的组件框架可以一起使用在微型或是全栈框架上。

实际上,「组件」是另一种建立、发布及推动开源的方式,现在存在的组件库主要包括Packagist和PEAR,而且它们都有用来安装和升级的命令行工具。

此外,还有基于组件的构成的框架的提供商提供不包含框架的组件,这些项目通常和其他的组件或者特定的框架没有依赖关系。例如,可以使用 FuelPHP 验证类库而不使用 FuelPHP 整个框架,或者使用Illuminate组件来实现Laravel框架解耦。












