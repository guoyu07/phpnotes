\part{FastCGI}

\chapter{Overview}


如果Web服务器不支持以服务器模块(例如Apache)的方式运行PHP等脚本语言,就必须考虑使用CGI或FastCGI的方式。

\section{CGI}


CGI(Common Gateway Interface,通用网关接口)是一种描述了服务器和请求处理程序之间传输数据的标准,可以让一个客户端从网页浏览器向执行在网络服务器上的程序请求数据。


CGI最初由NCSA(美国国家超级电脑应用中心)在1993年为NCSA HTTPd Web服务器开发,在NCSA HTTPd Web服务器中使用了UNIX shell 环境变量来保存从Web服务器传递出去的参数,然后生成一个运行CGI的独立的进程。


CGI的目标之一是要独立于任何语言,并且要求Web服务器无须在这个问题上对语言有任何了解,因此CGI程序可以用任何脚本语言或者是完全独立的编程语言实现,只要这个语言可以在这个系统上运行。例如,Unix shell script, Python, Ruby, PHP, Tcl, C/C++和Visual Basic等都可以用来开发CGI程序。

下面以实现维基百科编辑的CGI程序为例进行说明,首先用户代理程序(例如浏览器)向这个CGI程序请求某个名称的条目,如果该条目页面存在,CGI程序就会去获取那个条目页面的原始数据,然后把它转换成HTML并把结果输出给浏览器;如果该条目页面不存在,CGI程序则会提示用户新建一个页面。所有维基操作都是通过这个CGI程序来处理的。


从Web服务器的角度看,CGI的工作方式就是在特定的位置(比如\url{http://www.example.com/wiki.cgi})定义了可以运行CGI程序,这样当收到一个匹配URL的请求时,相应的程序就会被调用,并将客户端发送的数据作为输入。程序的输出会由Web服务器收集,并加上合适的HTTP头,再发送回客户端。

一般情况下,每次的CGI请求都需要新生成一个程序的副本来运行,这样大的工作量会很快将服务器压垮,因此通常是使用更有效的技术(例如mod\_perl)来让脚本解释器直接作为模块集成在Web服务器(例如Apache)中,从而避免重复载入和初始化解释器。

不过,这只是针对那些需要解释器的高级语言(即解释语言)而言的,使用C语言等编译型语言则可以避免这种额外负荷,而且C语言程序运行速度更快、对操作系统的负荷更小,因此使用编译型语言程序可以达到更高的执行效率。

如果代码的变动很少,那么可以在服务器产生一个新的进程在编译代码之前进行代码转译处理。例如,FastCGI会在第一次调用脚本时,在系统的某个地方保存脚本编译过的版本,这样对这个文件以后的请求就会自动转向这个编译过的代码,而不用每次调用脚本解释器来解释脚本。当更改了代码时,需要清空临时缓存,就可以保证调用的是最新的版本的脚本代码。

除了存储中间转译结果之外,还可以直接把解释器集成在Web服务器中,这样就无须新建一个进程来执行脚本。例如,Apache就提供了很多可集成的解释器模块。

\begin{compactitem}
\item mod\_cplusplus
\item mod\_perl
\item mod\_php
\item mod\_python
\item mod\_ruby
\item mod\_mono
\end{compactitem}


CGI使外部程序与Web服务器之间交互成为可能,不过CGI程序运行在独立的进程中,并对每个Web请求都建立一个进程。

虽然CGI非常容易实现,但是其效率很差且难以扩展,在处理大量请求时,进程的大量建立和消亡使操作系统性能急剧下降,而且地址空间无法共享,也限制了资源重用。


\section{FastCGI}





FastCGI(Fast Common Gateway Interface)是早期通用网关接口(CGI)的增强版本,可以让交互程序与 Web 服务器通信。

FastCGI 致力于减少网页服务器与 CGI 程序之间交互的开销,从而使服务器可以同时处理更多的网页请求。


与 CGI 为每个请求创建一个新的进程不同,FastCGI 使用持久运行的进程来处理一连串的请求,而且这些进程由 FastCGI 服务器管理(而不是 Web 服务器)。例如,Apache HTTP Server 通过mod\_fastcgi或mod\_fcgid 模块实现了 FastCGI。

\begin{compactitem}
\item 如果FastCGI进程与 Web 服务器都位于本地,当接收到一个请求时,Web 服务器把环境变量和页面请求通过一个 socket传递给 FastCGI 进程。
\item  如果FastCGI 进程在远端的应用服务器集群中,当接收到一个请求时,Web 服务器把环境变量和页面请求通过一个TCP connection传递给 FastCGI 进程。
\end{compactitem}


