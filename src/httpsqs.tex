\part{HTTPSQS}

\chapter{Overview}


\begin{lstlisting}[language=bash]
$ mkdir php_httpsqs_client 
$ cd php_httpsqs_client 
$ unzip php_httpsqs_0.1.zip 
$ /usr/local/php/bin/phpize 
$ ./configure --enable-httpsqs --with-php-config=/usr/local/php/bin/php-config 
# make && make install
\end{lstlisting}

在php.ini中添加如下的配置项来启用HTTPSQS扩展:

\begin{lstlisting}[language=bash]
extension=httpsqs.so;
\end{lstlisting}




\begin{lstlisting}[language=PHP]
/** 
* 创建httpsqs连接 
* @param string $host 服务器地址,可以为空,默认为127.0.0.1 
* @param int $port 服务器端口,可以为空,默认为1218 
* @return resource 
*/ 
$hr = httpsqs_connect("127.0.0.1", 1218);

/** 
* 写入队列数据 
* @param resource $hr 服务器连接句柄 
* @param string $queue 队列名称 
* @param string $data 写入数据 
* @param string $charset 字符集,可以为空,默认为utf-8 
* @return boolean 
*/ 
$putRes = httpsqs_put($hr, "testQueue", "This is a test Data", "UTF-8");

/** 
* 获取队列最后一条数据 
* @param resource $hr 
* @param string $queue 
* @param boolean $return_array 是否返回数组,可以为空,默认为false 返回数组格式:array('pos'=>'队列插入点', 'data'=>'数据值') 
* @param string $charset 可以为空 
* @return mixed 
*/ 
$content = httpsqs_get($hr, "testQueue", true, "UTF-8");

/** 
* 获取队列状态 
* @param resource $hr 
* @param string $queue 
* @param boolean $return_json 是否返回状态的json格式,可以为空,默认为false 
* @return string 
*/ 
$status = httpsqs_status($hr, "testQueue", true);

/** 
* 获取队列某个点数据 
* @param resource $hr 
* @param string $queue 
* @param int $pos 要获取的某条数据的位置 
* @param string $charset 可以为空 
* @return string 
*/ 
$posData = httpsqs_view($hr, "testQueue", 10, "UTF-8");

/** 
* 队列重置 
* @param resource $hr 
* @param string $queue 
* @return boolean 
*/ 
$resetRes = httpsqs_reset($hr, "testQueue");

/** 
* 设置队列最大数据条数 
* @param resource $hr 
* @param string $queue 
* @param int $maxqueue 队列最大数据条数 
* @return boolean 
*/ 
$maxqueueRes = httpsqs_maxqueue($hr, "testQueue", 10000);

/** 
* 修改定时刷新内存缓冲区内容到磁盘的间隔时间 
* @param resource $hr 
* @param string $queue 
* @param int $synctime 间隔时间 
* @return boolean 
*/ 
$synctimeRes = httpsqs_synctime($hr, "testQueue", 10);
\end{lstlisting}

对象调用


\begin{lstlisting}[language=PHP]
// 参数与httpsqs_connect对应 
$hr = new HttpSQS($host, $port);

// 参数与httpsqs_get对应 
$hr->get($queuename, $return_array, $charset);

// 参数与httpsqs_put对应 
$hr->put($queuename, $data, $charset);

// 参数与httpsqs_status对应 
$hr->status($queuename, $return_json);

// 参数与httpsqs_view对应 
$hr->view($queuename, $pos);

// 参数与httpsqs_reset对应 
$hr->reset($queuename);

// 参数与httpsqs_maxqueue对应 
$hr->maxqueue($queuename);

// 参数与httpsqs_synctime对应 
$hr->synctime($queuename);
\end{lstlisting}

HTTPSQS示例如下:


\begin{lstlisting}[language=PHP]
// 取数据Daemon 
$hr = httpsqs_connect($host, $port); 
while (1) { 
   $data = httpsqs_get($hr, $queuename, $charset); 
   if ($data === false) { 
      sleep(1); 
   } else { 
      // do something... 
   } 
}

// 或者 
$hr = new HttpSQS($host, $port); 
while (1) { 
    $data = $hr->get($queuename, $charset); 
    if ($data === false) { 
        sleep(1); 
    } else { 
        // do something... 
    } 
}

// 写数据 
$hr = httpsqs_connect($hort, $port); 
httpsqs_put($hr, $queuename, $data, $charset);

// 或者 $hr = new HttpSQS($hort, $port); 
$hr->put($queuename, $data, $charset); 
\end{lstlisting}




\begin{lstlisting}[language=bash]

\end{lstlisting}



\begin{lstlisting}[language=bash]

\end{lstlisting}




\begin{lstlisting}[language=bash]

\end{lstlisting}




\begin{lstlisting}[language=bash]

\end{lstlisting}



\begin{lstlisting}[language=bash]

\end{lstlisting}




\begin{lstlisting}[language=bash]

\end{lstlisting}




\begin{lstlisting}[language=bash]

\end{lstlisting}




\begin{lstlisting}[language=bash]

\end{lstlisting}




\begin{lstlisting}[language=bash]

\end{lstlisting}



\begin{lstlisting}[language=bash]

\end{lstlisting}




\begin{lstlisting}[language=bash]

\end{lstlisting}