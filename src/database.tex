\part{Database}



\chapter{Introduction}








MySQL可以进行缩减来支持嵌入的数据库应用程序,而且对于那些支持巨大数据和访问量的网站,MySQL 是事实上的标准数据库(比如 Friendster, Yahoo, Google)。

从PHP5开始,MySQLi模块提供了更加有效的方法和实用工具来处理数据库操作,这些方法大都是以面向对象的方式实现。

\section{mysql\_connect()}


在能够访问并处理数据库中的数据之前,必须创建到达数据库的连接。在 PHP 中,这个任务通过 mysql\_connect() 函数完成。


\begin{lstlisting}[language=PHP]
mysql_connect(servername,username,password);
\end{lstlisting}

\begin{compactitem}
\item servername

可选,规定要连接的服务器,默认是``localhost:3306"。
\item username

可选,规定登录所使用的用户名,默认值是拥有服务器进程的用户的名称。
\item password

可选,规定登录所用的密码,默认是``"。
\end{compactitem}

下面的脚本中使用变量\$con存放数据库连接。如果连接失败,将执行``die" 部分:

\begin{example}
mysql\_connect()
\begin{lstlisting}[language=PHP]
<?php
  $con = mysql_connect("localhost","admin","testtest");
  if(!$con){
    die('Could not connect: ' . mysql_error() );
  }
  //some code
?>
\end{lstlisting}
\end{example}

\section{mysql\_close()}

脚本执行结束,就会关闭数据库连接。若要提前关闭连接,使用 mysql\_close() 函数。


\begin{example}
mysql\_close()
\begin{lstlisting}[language=PHP]
<?php
  $con = mysql_connect("localhost","admin","testtest");
  if(!$con){
    die('Could not connect: ' . mysql_error() );
  }
  //some code
  
  mysql_close($con);
?>
\end{lstlisting}
\end{example}


\section{mysql\_query()}

为了让 PHP 执行MySQL数据库操作语句,必须使用 mysql\_query() 函数,此函数用于向 MySQL 连接发送查询或命令。

CREATE DATABASE 语句用于在 MySQL 中创建数据库。

\begin{lstlisting}[language=SQL]
CREATE DATABASE database_name
\end{lstlisting}



SQL 语句对大小写不敏感,创建MySQL数据库时必须使用mysql\_query()函数。在下面的示例创建了一个名为``my\_db" 的数据库:

\begin{example}
创建数据库
\begin{lstlisting}[language=PHP]
<?php
  $con = mysql_connect("localhost","admin","testtest");
  if(!$con){
    die('Could not connect ' . mysql_error() );
  }
  //Create database
  if(mysql_query("CREATE DATABASE my_db",$con) ){
    echo "Database created.";
  }else{
    echo "Error creating database: " . mysql_error();
  }
  
  mysql_close($con);
?>
\end{lstlisting}
\end{example}

CREATE TABLE 用于在 MySQL 中创建数据库表。

\begin{lstlisting}[language=SQL]
CREATE TABLE table_name
(
  column_name1 data_type,
  column_name2 data_type,
  column_name3 data_type,
  ... ...
)
\end{lstlisting}

为了创建表,必须向 mysql\_query() 函数添加 CREATE TABLE 语句。下面的示例创建了一个名为``Persons" 的表,而且该表中有三列,列名分别是``FirstName", ``LastName" 以及``Age":




\begin{example}
创建表
\begin{lstlisting}[language=PHP]
<?php
  $con = mysql_connect("localhost","admin","testtest");
  if(!$con){
    die('Could not connect: ' . mysql_error() );
  }
  //Create database
  if(mysql_query("CREATE DATABASE my_db"),$con){
    echo "Database created.";
  }else{
    echo "Error creating database: " . mysql_error();
  }
  //Create table in my_db database
  mysql_select_db("my_db",$con);
  $sql = "CREATE TABLE Persons
  (
    FirstName varchar(15),
    LastName varchar(15),
    Age int
  )";
  mysql_query($sql,$con);
  
  mysql_close($con);
?>
\end{lstlisting}
\end{example}


在创建表之前,必须首先选择MySQL数据库,通过 mysql\_select\_db() 函数完成选取数据库操作。当创建 varchar 类型的数据库字段时,必须规定该字段的最大长度,例如varchar(15)。

\zihao{6}

\begin{longtable}{|m{110pt}|m{280pt}|}
%%
\multicolumn{2}{r}{}
\tabularnewline\hline
数据类型 &描述
\endhead
%%

%%
\caption{MySQL 数据类型}\\
\hline
数据类型 &描述
\endfirsthead
%%

%%
\multicolumn{2}{r}{}
\endfoot
%%

%%
\endlastfoot
%%
\hline
int(size) \newline 
smallint(size) \newline 
tinyint(size) \newline 
mediumint(size) \newline 
bigint(size) 					&仅支持整数。在 size 参数中规定数字的最大值。\\
\hline
decimal(size,d) \newline 
double(size,d) \newline 
float(size,d) 				& 支持带有小数的数字。
							\newline 在 size 参数中规定数字的最大值。在 d 参数中规定小数点右侧的数字的最大值。\\
\hline
char(size)					&支持固定长度的字符串。(可包含字母、数字以及特殊符号)。
							\newline 在 size 参数中规定固定长度。\\
\hline
varchar(size)				& 支持可变长度的字符串。(可包含字母、数字以及特殊符号)。
							\newline 在 size 参数中规定最大长度。\\
\hline
tinytext						&支持可变长度的字符串,最大长度是 255 个字符。\\
\hline
text \newline 
blob						&支持可变长度的字符串,最大长度是 65535 个字符。\\
\hline
mediumtext \newline 
mediumblob					&支持可变长度的字符串,最大长度是 16777215 个字符。\\
\hline
longtext \newline 
longblob					&支持可变长度的字符串,最大长度是 4294967295 个字符。\\
\hline
date(yyyy-mm-dd) \newline
datetime(yyyy-mm-dd hh:mm:ss) \newline 
timestamp(yyyymmddhhmmss) \newline 
time(hh:mm:ss) 				&支持日期或时间\\
\hline
enum(value1,value2,ect)		&ENUM 是 ENUMERATED列表的缩写。可以在括号中存放最多 65535 个值。\\
\hline
set							&SET与ENUM相似。但是SET可拥有最多64个列表项目,并可存放不止一个choice\\
\hline
\end{longtable}

\zihao{5}

主键用于对表中的行进行唯一标识,每个表都应有一个主键字段。


每个主键值在表中必须是唯一的,而且主键字段不能为空,这是由于数据库引擎需要一个值来对记录进行定位。

下面的例子把 personID 字段设置为主键字段。主键字段通常是 ID 号,且通常使用 AUTO\_INCREMENT规定。AUTO\_INCREMENT 会在新记录被添加时逐一增加该字段的值。

要确保主键字段不为空,就必须向该字段添加 NOT NULL 设置。

\begin{example}
主键
\begin{lstlisting}[language=PHP]
$sql = "CREATE TABLE Persons
(
  personID int NOT NULL AUTO_INCREMENT,
  PRIMARY KEY(personID),
  FirstName varchar(15),
  LastName varchar(15),
  Age int
)";

mysql_query($sql,$con);
\end{lstlisting}
\end{example}


主键字段永远要被编入索引,这条规则没有例外。

对主键字段进行索引才能使数据库引擎快速定位给予该键值的行。

INSERT INTO 语句用于向数据库表添加新记录。


\begin{lstlisting}[language=PHP]
INSERT INTO table_name
VALUES (value1, value2,....)
\end{lstlisting}

还可以规定希望在其中插入数据的列:

\begin{lstlisting}[language=PHP]
INSERT INTO table_name (column1, column2,...)
VALUES (value1, value2,....)
\end{lstlisting}

为了向MySQL数据库表插入数据,必须使用 mysql\_query() 函数。下面的例子向``Persons" 表添加了两个新记录:

\begin{example}
插入数据
\begin{lstlisting}[language=PHP]
<?php
  $con = mysql_connect("localhost","admin","testtest");
  if(!$con){
    die('Could not connect: ' . mysql_error() );
  }
  mysql_select("my_db",$con);
  mysql_query("INSERT INTO Persons(FirstName,LastName,Age)
  VALUES('Peter','Griffin','34')
  ");
  mysql_query("INSERT INTO Persons(FirstName,LastName,Age)
  VALUES('Green','Jim','30')
  ");
  
  mysql_close($con);
?>
\end{lstlisting}
\end{example}


把来自表单的数据插入数据库时,首先通过 \$\_POST 变量从表单取回值,然后mysql\_query() 函数执行 INSERT INTO 语句,这样一条新的记录才会添加到数据库表中。



\begin{example}
HTML表单
\begin{lstlisting}[language=HTML]
<!DOCTYPE html>
<html>
<title>HTML Form Example</title>
<body>

<form action="insert.php" method="post">
Firstname: <input type="text" name="firstname" />
Lastname: <input type="text" name="lastname" />
Age: <input type="text" name="age" />
<input type="submit" />
</form>

</body>
</html>
\end{lstlisting}
\end{example}

当用户点击示例中 HTML 表单中的提交按钮时,表单数据被发送到``insert.php",insert.php中的脚本完成新数据插入操作。


\begin{example}
将HTML表单插入数据库
\begin{lstlisting}[language=PHP]
<?php
  $con = mysql_connect("localhost","admin","testtest");
  if(!$con){
    die('Could not connect: ' . mysql_error() );
  }
  mysql_select_db("my_db",$con);
  $sql = "INSERT INTO Persons(FirstName,LastName,Age)
  VALUES
  ('$_POST['firstname']','$_POST['lastname]','$_POST['age']')";
  if(!mysql_query($sql,$con)){
    die('Error: ' . mysql_error() );
  }
  echo "1 record added.";
  
  mysql_close($con);
?>
\end{lstlisting}
\end{example}








\begin{example}

\begin{lstlisting}[language=PHP]

\end{lstlisting}
\end{example}











\begin{example}

\begin{lstlisting}[language=PHP]

\end{lstlisting}
\end{example}










\begin{example}

\begin{lstlisting}[language=PHP]

\end{lstlisting}
\end{example}











\begin{example}

\begin{lstlisting}[language=PHP]

\end{lstlisting}
\end{example}