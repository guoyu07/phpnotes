\part{FPM}


\chapter{Overview}


\section{PHP-FPM}


FPM (FastCGI 进程管理器) 是一个可选的 PHP FastCGI 实现,并且提供了一些针对高负载网站的特性,可以用来替换 PHP FastCGI 的大部分功能,提高网站负载。

\begin{compactitem}
\item 支持平滑停止/启动的高级进程管理功能;

\item 可以工作于不同的 uid/gid/chroot 环境下,并监听不同的端口和使用不同的 php.ini 配置文件(可取代 safe\_mode 的设置);

\item stdout 和 stderr 日志记录;

\item 在发生意外情况的时候能够重新启动并缓存被破坏的 opcode;

\item 文件上传优化支持;

\item ``慢日志" - 记录脚本(不仅记录文件名,还记录 PHP backtrace 信息,可以使用 ptrace或者类似工具读取和分析远程进程的运行数据)运行所导致的异常缓慢;

\item \texttt{fastcgi\_finish\_request()} - 在请求完成和刷新数据后继续在后台执行耗时的工作(录入视频转换、统计处理等);

\item 动态/静态子进程产生;

\item 基本 SAPI 运行状态信息(类似Apache的 mod\_status);

\item 基于 php.ini 的配置文件。

\end{compactitem}

编译 PHP 时需要 \texttt{-\/-enable-fpm} 配置选项来激活 FPM 支持,以下为 FPM 编译的具体配置参数(全部为可选参数):

\begin{compactitem}
\item \texttt{-\/-with-fpm-user} 设置 FPM 运行的用户身份(默认 - nobody)

\item \texttt{-\/-with-fpm-group} 设置 FPM 运行时的用户组(默认 - nobody)

\item \texttt{-\/-with-fpm-systemd} 启用 systemd 集成 (默认 - no)

\item \texttt{-\/-with-fpm-acl} 使用POSIX 访问控制列表 (默认 - no)
\end{compactitem}

FPM 配置文件为 php-fpm.conf,其语法类似 php.ini 。


\begin{compactenum}
\item php-fpm.conf 全局配置段

\begin{compactitem}
\item \texttt{pid string}

PID文件的位置。默认为空。

\item \texttt{error\_log string}

错误日志的位置。默认安装路径 \texttt{\#INSTALL\_PREFIX\#/log/php-fpm.log}。

\item \texttt{log\_level string}

错误级别。可用级别为:\texttt{alert}(必须立即处理),\texttt{error}(错误情况),\texttt{warning}(警告情况),\texttt{notice}(一般重要信息),\texttt{debug}(调试信息)。默认:\texttt{notice}。

\item \texttt{emergency\_restart\_threshold int}

如果子进程在 \texttt{emergency\_restart\_interval} 设定的时间内收到该参数设定次数的 \texttt{SIGSEGV} 或者 \texttt{SIGBUS}退出信息号,则FPM会重新启动。0 表示“关闭该功能”。默认值:0(关闭)。

\item \texttt{emergency\_restart\_interval mixed}

\texttt{emergency\_restart\_interval}用于设定平滑重启的间隔时间。这么做有助于解决加速器中共享内存的使用问题。可用单位:s(秒),m(分),h(小时)或者 d(天)。默认单位:s(秒)。默认值:0(关闭)。

\item \texttt{process\_control\_timeout mixed}

设置子进程接受主进程复用信号的超时时间。可用单位:s(秒),m(分),h(小时)或者 d(天)。默认单位:s(秒)。默认值:0(关闭)。

\item \texttt{daemonize boolean}

设置 FPM 在后台运行。设置“no”将 FPM 保持在前台运行用于调试。默认值:yes。

\end{compactitem}

\item 运行配置区段

在FPM中,可以使用不同的设置来运行多个进程池。 这些设置可以针对每个进程池单独设置。

\begin{compactitem}
\item \texttt{listen string}

设置接受 FastCGI 请求的地址。可用格式为:\texttt{'ip:port'},\texttt{'port'},\texttt{'/path/to/unix/socket'}。每个进程池都需要设置。

\item \texttt{listen.backlog int}
设置 \texttt{listen(2)}的半连接队列长度。“-1”表示无限制。默认值:-1。

\item \texttt{listen.allowed\_clients string}

设置允许连接到 FastCGI 的服务器 IPV4 地址。等同于 PHP FastCGI (5.2.2+) 中的 \texttt{FCGI\_WEB\_SERVER\_ADDRS}环境变量。仅对 TCP 监听起作用。每个地址是用逗号分隔,如果没有设置或者为空,则允许任何服务器请求连接。默认值:\texttt{any}。

\item \texttt{listen.owner string}

如果使用,表示设置 Unix 套接字的权限。在Linux中,读写权限必须设置,以便用于 WEB 服务器连接。在很多 BSD 派生的系统中可以忽略权限允许自由连接。默认值:运行所使用的用户和组,权限为 0666。

\item \texttt{listen.group string}

参见 \texttt{listen.owner}。

\item \texttt{listen.mode string}

参见 \texttt{listen.owner}。

\item \texttt{user string}

FPM 进程运行的Unix用户。必须设置。

\item \texttt{group string}

FPM 进程运行的 Unix 用户组。如果没有设置,则默认用户的组被使用。

\item \texttt{pm string}

设置进程管理器如何管理子进程。可用值:\texttt{static},\texttt{ondemand},\texttt{dynamic}。必须设置。

\begin{compactitem}
\item \texttt{static} - 子进程的数量是固定的(\texttt{pm.max\_children})。
\item \texttt{ondemand} - 进程在有需求时才产生(当请求时,与 \texttt{dynamic}相反,\texttt{pm.start\_servers}在服务启动时即启动。
\item \texttt{dynamic} - 子进程的数量在下面配置的基础上动态设置:\texttt{pm.max\_children},\texttt{pm.start\_servers},\texttt{pm.min\_spare\_servers},\texttt{pm.max\_spare\_servers}。
\end{compactitem}


\item \texttt{pm.max\_children int}

\texttt{pm}设置为\texttt{static}时表示创建的子进程的数量,\texttt{pm}设置为\texttt{dynamic}时表示最大可创建的子进程的数量。必须设置。

该选项设置可以同时提供服务的请求数限制。类似 Apache 的 \texttt{mpm\_prefork}中 \texttt{MaxClients}的设置和普通PHP FastCGI中的 \texttt{PHP\_FCGI\_CHILDREN}环境变量。

\item \texttt{pm.start\_servers in}

设置启动时创建的子进程数目。仅在\texttt{pm}设置为\texttt{dynamic}时使用。默认值:\texttt{min\_spare\_servers + (max\_spare\_servers - min\_spare\_servers) / 2}。

\item \texttt{pm.min\_spare\_servers int}

设置空闲服务进程的最低数目。仅在\texttt{pm}设置为\texttt{dynamic}时使用。必须设置。

\item \texttt{pm.max\_spare\_servers int}

设置空闲服务进程的最大数目。仅在\texttt{pm}设置为\texttt{dynamic}时使用。必须设置。

\item \texttt{pm.max\_requests int}

设置每个子进程重生之前服务的请求数。对于可能存在内存泄漏的第三方模块来说是非常有用的。如果设置为 '0' 则一直接受请求,等同于 \texttt{PHP\_FCGI\_MAX\_REQUESTS}环境变量。默认值:0。

\item \texttt{pm.status\_path string}

FPM 状态页面的网址。如果没有设置,则无法访问状态页面,默认值:无。

\item \texttt{ping.path string}

FPM 监控页面的 ping 网址。如果没有设置,则无法访问 ping 页面。该页面用于外部检测 FPM 是否存活并且可以响应请求。请注意必须以斜线开头(/)。

\item \texttt{ping.response string}

用于定义 ping 请求的返回响应。返回为 HTTP 200 的 text/plain 格式文本。默认值:\texttt{pong}。

\item \texttt{request\_terminate\_timeout mixed}

设置单个请求的超时中止时间。该选项可能会对 php.ini 设置中的 \texttt{'max\_execution\_time'}因为某些特殊原因没有中止运行的脚本有用。设置为 \texttt{'0'} 表示 \texttt{'Off'}。可用单位:s(秒),m(分),h(小时)或者 d(天)。默认单位:s(秒)。默认值:0(关闭)。

\item \texttt{request\_slowlog\_timeout mixed}

当一个请求该设置的超时时间后,就会将对应的 PHP 调用堆栈信息完整写入到慢日志中。设置为 \texttt{'0'}表示 \texttt{'Off'}。可用单位:s(秒),m(分),h(小时)或者 d(天)。默认单位:s(秒)。默认值:0(关闭)。

\item \texttt{slowlog string}

慢请求的记录日志。默认值:\texttt{\#INSTALL\_PREFIX\#/log/php-fpm.log.slow}。

\item \texttt{rlimit\_files int}

设置文件打开描述符的 rlimit 限制。默认值:系统定义值。

\item \texttt{rlimit\_core int}

设置核心 \texttt{rlimit}最大限制值。可用值:\texttt{'unlimited'},0 或者正整数。默认值:系统定义值。

\item \texttt{chroot string}

启动时的 Chroot 目录。所定义的目录需要是绝对路径。如果没有设置,则 chroot 不被使用。

\item \texttt{chdir string}

设置启动目录,启动时会自动 Chdir 到该目录。所定义的目录需要是绝对路径。默认值:当前目录,或者根目录(chroot时)。

\item \texttt{catch\_workers\_output boolean}

重定向运行过程中的 \texttt{stdout}和\texttt{stderr}到主要的错误日志文件中。如果没有设置,\texttt{stdout}和\texttt{stderr}将会根据 FastCGI 的规则被重定向到\texttt{/dev/null}。默认值:无。

\end{compactitem}

\end{compactenum}

还可以在为一个运行池传递附加的环境变量,或者更新 PHP 的配置值。可以在 php-fpm.conf 中如下面的配置参数来做到:

\begin{lstlisting}[language=bash]
env[HOSTNAME] = $HOSTNAME
       env[PATH] = /usr/local/bin:/usr/bin:/bin
       env[TMP] = /tmp
       env[TMPDIR] = /tmp
       env[TEMP] = /tmp

       php_admin_value[sendmail_path] = /usr/sbin/sendmail -t -i -f www@my.domain.com
       php_flag[display_errors] = off
       php_admin_value[error_log] = /var/log/fpm-php.www.log
       php_admin_flag[log_errors] = on
       php_admin_value[memory_limit] = 32M
\end{lstlisting}

PHP配置值通过 \texttt{php\_value}或者\texttt{php\_flag}设置,并且会覆盖以前的值。注意\texttt{disable\_functions}或者\texttt{disable\_classes}在 php.ini 之中定义的值不会被覆盖掉,但是会将新的设置附加在原有值的后面。

使用 \texttt{php\_admin\_value}或者\texttt{php\_admin\_flag}定义的值,不能被 PHP 代码中的 \texttt{ini\_set()}覆盖。

自PHP 5.3.3 起,也可以通过Web服务器设置 PHP 的设定。例如,在 nginx.conf 中设定 PHP的示例如下:

\begin{lstlisting}[language=bash]
set $php_value "pcre.backtrack_limit=424242";
set $php_value "$php_value \n pcre.recursion_limit=99999";
fastcgi_param  PHP_VALUE $php_value;

fastcgi_param  PHP_ADMIN_VALUE "open_basedir=/var/www/htdocs";
\end{lstlisting}

上述这些设定是以 FastCGI 标头传递给 php-fpm,因此php-fpm不应被绑定到外部网可以访问的地址上,否则任何人就能可以修改 PHP 的配置选项。
