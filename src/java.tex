\part{Java}


\chapter{Overview}

除了可以使用Zend Server提供的Java Bridge功能之外,还可以使用开源的PHP Java Bridge

安装基础安装包(例如php5-dev,sun java和automake等)

\begin{lstlisting}[language=bash]
# apt-get install build-essential php5-dev sun-java6-jre \
    sun-java6-jdk sun-java6-source automake
\end{lstlisting}


下载PHP/Java Bridge源代码

\begin{lstlisting}[language=PHP]
# cd /usr/src
# wget http://internap.dl.sourceforge.net/sourceforge/php-java-bridge/php-java-bridge_5.2.0.tar.gz
# tar xzfv php-java-bridge_5.2.0.tar.gz
\end{lstlisting}

安装PHP/Java Bridge


\begin{lstlisting}[language=PHP]
# cd /usr/src/php-java-bridge-5.2.0
# phpize
# ./configure --with-java=/usr/lib/jvm/java-6-sun-1.6.0.03,/usr/lib/jvm/java-6-sun-1.6.0.03
# make
# chmod +x install.sh
# ./install.sh
\end{lstlisting}

拷贝相关*.inc到对应的目录下

\begin{lstlisting}[language=PHP]
# mkdir /usr/share/php/java
# cp *.inc /usr/share/php/java
\end{lstlisting}



禁用java.hosts和java.servlet并执行测试:




\begin{lstlisting}[language=PHP]
# php test.php
\end{lstlisting}

默认情况下,Java仅仅在PHP的命令行模式下可用,如果需要在全局使用Java,那么可以在php.ini修改如下配置:

\begin{lstlisting}[language=PHP]
;; java.ini: Activate the PHP/Java extension

;; zend_extension = "/path/to/java.so"
extension = java.so

;; If you have installed the java-servlet.ini leave this file alone,
;; edit the servlet or standalone ini file.  Otherwise uncomment the
;; following java section and one of the following options:
[java]
\end{lstlisting}

Zend的Java Bridge已经有一个名为java\_require的函数,开源的PHP/Java Bridge需要包括这个文件,而且PHP/Java Bridge需要java\_cast()函数,Zend的Java Bridge不需要这个文件和函数。

根据上述两个差异,开发者需要在bootstrap文件(或对应代码)覆盖它们,这样应用程序就可以与这两个Java/Bridge一起工作。

\begin{lstlisting}[language=PHP]
/* no java_require() include the java.inc for PHP/Java Bridge */
if ( ! function_exists('java_require') ) {
    include "java/Java.inc";
}

/* declare this, it doesn't exist with Zend Java, but is needed for PHP/Java Bridge */
if ( ! function_exists('java_cast')) {
    function java_cast($whatever) {
        return $whatever;
    }
}
\end{lstlisting}

上述解决并非完美,在实际情况中需要根据需求进行相应的配置。

\begin{lstlisting}[language=Java]
// Get a reference to a container-managed resource from WebLogic Server
function getWebLogicResource($jndiName) {
  // Set WebLogic-specific parameters for creation of jndi naming context:
  // (weblogic.jar must be on the PHP-Java Bridge classpath)
  $envt = array(
     "java.naming.factory.initial" => "weblogic.jndi.WLInitialContextFactory",
     "java.naming.provider.url" => "t3://localhost:7001"   // replace with your target server URL
  );  

  // The bridge creates the naming context object in the local JVM.
  $ctx = new Java('javax.naming.InitialContext', $envt);

  // Now JNDI crosses the network, to obtain the resource.
  return $ctx->lookup($jndiName);
}
\end{lstlisting}





\begin{lstlisting}[language=PHP]

\end{lstlisting}




\begin{lstlisting}[language=PHP]

\end{lstlisting}



\begin{lstlisting}[language=PHP]

\end{lstlisting}



\begin{lstlisting}[language=PHP]

\end{lstlisting}



\begin{lstlisting}[language=PHP]

\end{lstlisting}



\begin{lstlisting}[language=PHP]

\end{lstlisting}



\begin{lstlisting}[language=PHP]

\end{lstlisting}



\begin{lstlisting}[language=PHP]

\end{lstlisting}






\begin{lstlisting}[language=PHP]

\end{lstlisting}




\begin{lstlisting}[language=PHP]

\end{lstlisting}



\begin{lstlisting}[language=PHP]

\end{lstlisting}



\begin{lstlisting}[language=PHP]

\end{lstlisting}



\begin{lstlisting}[language=PHP]

\end{lstlisting}



\begin{lstlisting}[language=PHP]

\end{lstlisting}



\begin{lstlisting}[language=PHP]

\end{lstlisting}



\begin{lstlisting}[language=PHP]

\end{lstlisting}






\begin{lstlisting}[language=PHP]

\end{lstlisting}




\begin{lstlisting}[language=PHP]

\end{lstlisting}



\begin{lstlisting}[language=PHP]

\end{lstlisting}



\begin{lstlisting}[language=PHP]

\end{lstlisting}



\begin{lstlisting}[language=PHP]

\end{lstlisting}



\begin{lstlisting}[language=PHP]

\end{lstlisting}



\begin{lstlisting}[language=PHP]

\end{lstlisting}



\begin{lstlisting}[language=PHP]

\end{lstlisting}