\part{PHP Form}







\chapter{PHP Forms Processing}


PHP 一个很有用的特点体现在它处理 PHP 表单的方式,表单的任何元素都在 PHP 脚本中自动生效。


\begin{compactitem}
\item GET\footnote{对于GET方式,服务器端用Request.QueryString获取变量的值。}

通过HTTP GET机制,把参数数据队列加到提交表单的action属性所指的URL中,值和表单内各个字段一一对应,在URL中可以看到,因此GET安全性非常低,建议用POST方式传输机密信息\footnote{具体来说,在做数据查询时,建议用GET方式,而在做数据添加、修改或删除时,建议用POST方式。}。

通过HTTP GET机制传送的数据量较小,不能大于2KB,因此GET执行效率比POST方法好。

\item POST\footnote{对于POST方式,服务器端用Request.Form获取提交的数据。}

通过HTTP POST机制,将表单内各个字段与其内容放置在HTML header内一起传送到action属性所指的URL地址,用户是看不到这个过程的,因此POST安全性较高。

通过HTTP POST机制传送的数据量较大,一般被默认为不受限制。但理论上,IIS4中最大量为80KB,IIS5中为100KB。
\end{compactitem}




表单是Web应用程序和用户交互时重要的手段,PHP 的 \textcolor{Blue}{\texttt{\$\_GET}} 和 \textcolor{Blue}{\texttt{\$\_POST}} 用于检索表单中的值,比如用户输入等。



\begin{compactitem}
\item \textcolor{Blue}{\texttt{\$\_GET}} 变量用于收集来自 method=``get" 的表单中的值。
\item \textcolor{Blue}{\texttt{\$\_POST}}变量用于收集来自 method=``post" 的表单中的值。
\end{compactitem}

Web设计中常见的表单示例如下:

\begin{lstlisting}[language=HTML]
<!DOCTYPE html>
<html>
<head>
<title></title>
</head>
<body>

<form action="welcome.php" method="post">
Name: <input type="text" name="name" />
Age: <input type="text" name="age" />
<input type="submit" />
</form>

</body>
</html>
\end{lstlisting}

上面的 HTML 页面实例包含了两个输入框和一个提交按钮。当用户填写了该表单并单击了提交按钮后,页面 welcome.php 将被调用。表单的数据会被送往 "welcome.php" 这个文件,而``welcome.php" 文件类似这样:

\begin{lstlisting}[language=HTML]
<!DOCTYPE html>
<html>
<head>
<title></title>
</head>
<body>

Welcome <?php echo $_POST["name"]; ?>.<br />
You are <?php echo $_POST["age"]; ?> years old.

</body>
</html>
\end{lstlisting}

更严格的PHP脚本如下:

\begin{lstlisting}[language=PHP]
Welcome <?php echo htmlspecialchars($_POST['name']); ?>.<br />
You are <?php echo (int)$_POST["age"]; ?> years old.
\end{lstlisting}

除了 \texttt{htmlspecialchars()} 和 \texttt{(int)} 部分,这段程序做什么用显而易见。

\texttt{htmlspecialchars()} 使得 HTML 之中的特殊字符被正确的编码,从而不会被使用者在页面注入 HTML 标签或者 Javascript 代码。

我们明确知道age 字段是一个数值,因此将它转换为一个整形值\texttt{(integer)}来自动的消除任何不必要的字符。也可以使用 PHP 的 \texttt{filter} 扩展来自动完成该工作。

PHP 将自动设置 \texttt{\$\_POST['name']} 和 \texttt{\$\_POST['age']} 变量。在这之前我们使用了超全局变量 \texttt{\$\_SERVER},现在引入超全局变量是 \texttt{\$\_POST},它包含了所有的 POST 数据。

如果表单提交数据的方法(method)使用的是 \texttt{GET}方法,那么表单中的信息将被储存到超全局变量 \texttt{\$\_GET} 中。

如果并不关心请求数据的来源,也可以用超全局变量 \texttt{\$\_REQUEST},它包含了所有 \texttt{GET}、\texttt{POST}、\texttt{COOKIE}和 \texttt{FILE}的数据。

也可以在 PHP 中处理 XForms 的输入,尽管用户可能更喜欢使用长久以来支持良好的 HTML 表单。


\chapter{PHP Forms Validation}

应该在任何可能的时候对用户输入进行验证,客户端验证的优势在于速度更快,从而可以减轻服务器的负载。


不过,任何流量很高以至于不得不担心服务器资源的站点,也有必要担心站点的安全性。如果表单访问的是数据库,就非常有必要采用服务器端的验证。

在服务器验证表单的一种好的方式是,把表单传给它自己,而不是跳转到不同的页面。这样用户就可以在同一张表单页面得到错误信息,也就更容易发现错误了。



\chapter{PHP \$\_GET}

\textcolor{Blue}{\texttt{\$\_GET}} 变量是一个数组,内容是由 HTTP GET 方法\footnote{HTTP GET 方法不适合大型的变量值,不能超过 100 个字符。}发送的变量名称和值。


\textcolor{Blue}{\texttt{\$\_GET}} 变量用于收集来自 method="get" 的表单中的值。从带有 GET 方法的表单发送的信息,对任何人都是可见的(会显示在浏览器的地址栏),并且对发送的信息量也有限制(最多 100 个字符)。

\begin{lstlisting}[language=HTML]
<form action="welcome.php" method="get">
Name: <input type="text" name="name" />
Age: <input type="text" name="age" />
<input type="submit" />
</form>
\end{lstlisting}

当用户点击提交按钮时,发送的 URL 会类似这样:

\begin{lstlisting}[language=bash]
http://www.domain.com.cn/welcome.php?name=Jim&age=30
\end{lstlisting}

``welcome.php" 文件通过\textcolor{Blue}{\texttt{\$\_GET}} 变量来获取表单数据时,表单域的名称会自动成为\textcolor{Blue}{\texttt{\$\_GET}}数组中的 ID 键:

\begin{lstlisting}[language=HTML]
Welcome <?php echo $_GET["name"]; ?>.<br />
You are <?php echo $_GET["age"]; ?> years old!
\end{lstlisting}

在使用\textcolor{Blue}{\texttt{\$\_GET}}变量时,所有的变量名和值都会显示在 URL 中。所以在发送密码或其他敏感信息时,不应该使用这个方法。不过,正因为变量显示在 URL 中,因此可以在收藏夹中收藏该页面。在某些情况下,这是很有用的。


\chapter{PHP \$\_POST}

\textcolor{Blue}{\texttt{\$\_POST}}变量是一个数组,内容是由 HTTP POST 方法发送的变量名称和值。



\textcolor{Blue}{\texttt{\$\_POST}}变量用于收集来自 method="post" 的表单中的值。从带有 POST 方法的表单发送的信息,对任何人都是不可见的(不会显示在浏览器的地址栏),并且对发送信息的量也没有限制。

\begin{lstlisting}[language=HTML]
<form action="welcome.php" method="post">
Enter your name: <input type="text" name="name" />
Enter your age: <input type="text" name="age" />
<input type="submit" />
</form>
\end{lstlisting}

当用户点击提交按钮,URL 不会含有任何表单数据,看上去类似这样:


\begin{lstlisting}[language=bash]
http://www.domain.com.cn/welcome.php
\end{lstlisting}

``welcome.php" 文件现在可以通过\textcolor{Blue}{\texttt{\$\_POST}}变量来获取表单数据时,表单域的名称会自动成为\textcolor{Blue}{\texttt{\$\_POST}}数组中的 ID 键。

\begin{lstlisting}[language=HTML]
Welcome <?php echo $_POST["name"]; ?>.<br />
You are <?php echo $_POST["age"]; ?> years old!
\end{lstlisting}


使用\textcolor{Blue}{\texttt{\$\_POST}},通过 HTTP POST 发送的变量不会显示在 URL 中,而且变量没有长度限制。不过,由于变量不显示在 URL 中,所以无法把页面加入书签。







\chapter{PHP \$\_REQUEST}


PHP 的\textcolor{Blue}{\texttt{\$\_REQUEST}}变量包含了\textcolor{Blue}{\texttt{\$\_GET}}, \textcolor{Blue}{\texttt{\$\_POST}}以及\textcolor{Blue}{\texttt{\$\_COOKIE}}的内容。

PHP 的\textcolor{Blue}{\texttt{\$\_REQUEST}}变量可用来取得通过\textcolor{Blue}{\texttt{GET}}和\textcolor{Blue}{\texttt{POST}}等方法发送的表单数据的结果。



\begin{lstlisting}[language=HTML]
Welcome <?php echo $_REQUEST["name"]; ?>.<br />
You are <?php echo $_REQUEST["age"]; ?> years old!
\end{lstlisting}







\bibliographystyle{plainnat}
\bibliography{phpnotes}
















































