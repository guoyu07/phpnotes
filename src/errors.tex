\part{Errors}


\chapter{Overview}

PHP使用内部错误号来报告错误,而且每个PHP错误都带有一个类型,PHP可以显示或记录错误。

如果没有设置错误处理器(error handler),PHP会根据配置文件来处理错误。例如,php.ini中的errror\_reporting配置项说明了哪些错误需要报告,哪些错误需要忽略。

用户可以使用error\_reporting()在运行时设置错误报告级别和处理,推荐在php.ini中配置错误报告选项,这样有些错误可以脚本执行之前报告。

\begin{compactitem}
\item 开发阶段建议配置:\texttt{E\_ALL}
\item 生产环境建议配置:\texttt{E\_ALL \& \~{}E\_NOTICE \& \~{}E\_STRICT \& \~{}E\_DEPRECATED}
\end{compactitem}

php.ini中的两个配置选项说明如何进一步处理错误:

\begin{compactitem}
\item display\_errors控制是否把错误显示为PHP脚本的输出(在生产环境中应该始终关闭)

\item log\_errors控制是否记录错误到日志文件或syslog(通过error\_log指令)
\end{compactitem}

如果PHP默认的错误处理器不满足需求,用户还可以定制自己的错误处理器来针对不同类型的错误进行处理。 

在生产环境中可以记录错误并生成错误报告来定位错误,不过PHP 7已经改变了大多数错误的报告方式,不同于传统的错误报告机制,现在大多数错误被作为 Error 异常抛出。

现在Error 异常可以像 Exception 异常一样被第一个匹配的 try / catch 块所捕获。

\begin{compactitem}
\item 如果没有匹配的 catch 块,则调用异常处理函数(事先通过 set\_exception\_handler() 注册)进行处理。 
\item 如果尚未注册异常处理函数,则按照传统方式处理(被报告为一个致命错误)。
\end{compactitem}


Error 类并非继承自 Exception 类,所以不能用 \texttt{catch (Exception \$e) \{ ... \}} 来捕获 Error,可以用\texttt{catch (Error \$e) \{ ... \}},或者通过注册异常处理函数( set\_exception\_handler())来捕获 Error。

\section{Error Handling}

\subsection{error\_reporting()}


\subsection{set\_error\_handler()}



\section{Hierarchical structure}


\subsection{Throwable}



\subsection{ArithmeticError}



\subsection{DivisionByZeroError}


\subsection{AssertionError}


\subsection{ParseError}


\subsection{TypeError}






