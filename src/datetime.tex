\part{Date/Time}


\chapter{Overview}


DateTime 类的作用是在读、写、比较或者计算日期和时间时提供帮助。除了 DateTime 类之外,PHP 还有很多与日期和时间相关的函数,不过DateTime类为大多数常规使用提供了优秀的面向对象接口,而且DateTime类还可以处理时区。

在使用 DateTime 之前,通过 createFromFormat() 工厂方法将原始的日期与时间字符串转换为对象或使用 new DateTime 来取得当前的日期和时间。使用 format() 将 DateTime 转换回字符串用于输出。

\section{createFromFormat()}



\section{format()}


\begin{lstlisting}[language=PHP]
<?php
$raw = '22. 11. 1968';
$start = DateTime::createFromFormat('d. m. Y', $raw);

echo 'Start date: ' . $start->format('Y-m-d') ."\n";
\end{lstlisting}


对 DateTime 进行计算时可以使用 DateInterval 类。DateTime 类具有例如 add() 和 sub() 等将 DateInterval 当作参数的方法。

编写代码时注意不要认为每一天都是由相同的秒数构成的,不论是夏令时(DST)还是时区转换,使用时间戳计算都会遇到问题,应当选择日期间隔。

使用 diff() 方法来计算日期之间的间隔,它会返回新的 DateInterval,而且非常容易进行展示。

\section{add()}



\section{sub()}


\section{diff()}


\begin{lstlisting}[language=PHP]
<?php
// create a copy of $start and add one month and 6 days
$end = clone $start;
$end->add(new DateInterval('P1M6D'));

$diff = $end->diff($start);
echo 'Difference: ' . $diff->format('%m month, %d days (total: %a days)') . "\n";
// Difference: 1 month, 6 days (total: 37 days)
\end{lstlisting}

DateTime 对象之间可以直接进行比较:



\begin{lstlisting}[language=PHP]
<?php
if ($start < $end) {
    echo "Start is before end!\n";
}
\end{lstlisting}

最后一个例子来演示 DatePeriod 类。它用来对循环的事件进行迭代。向它传入开始时间、结束时间和间隔区间,会得到这其中所有的时间。



\begin{lstlisting}[language=PHP]
<?php
// output all thursdays between $start and $end
$periodInterval = DateInterval::createFromDateString('first thursday');
$periodIterator = new DatePeriod($start, $periodInterval, $end, DatePeriod::EXCLUDE_START_DATE);
foreach ($periodIterator as $date) {
    // output each date in the period
    echo $date->format('Y-m-d') . ' ';
}
\end{lstlisting}



\begin{lstlisting}[language=PHP]

\end{lstlisting}



\begin{lstlisting}[language=PHP]

\end{lstlisting}




\begin{lstlisting}[language=PHP]

\end{lstlisting}




\begin{lstlisting}[language=PHP]

\end{lstlisting}


\begin{lstlisting}[language=PHP]

\end{lstlisting}


\begin{lstlisting}[language=PHP]

\end{lstlisting}



\begin{lstlisting}[language=PHP]

\end{lstlisting}



\begin{lstlisting}[language=PHP]

\end{lstlisting}



\begin{lstlisting}[language=PHP]

\end{lstlisting}




\begin{lstlisting}[language=PHP]

\end{lstlisting}