\part{Stomp}


\chapter{Overview}

Stomp(Simple(or Streaming) Text Oriented Message Protocol)前身为TTMP,是一个与消息中间件(Message-Oriented Middleware)交互的基于文本的通信协议。

Stomp提供了一个互操作性的wire format来允许Stomp客户端和任何支持Stomp协议的消息中间件(Message Broker)通信,因此Stomp是语言无关的,使用不同语言开发的Stomp客户端和消息中间件仍然可以进行通信。

Stomp协议和HTTP协议很相似,二者都是在运行在TCP协议之上的。

Stomp协议在TCP协议上运行时支持以下命令:

\begin{compactitem}
\item CONNECT
\item SEND
\item SUBSCRIBE
\item UNSUBSCRIBE
\item BEGIN
\item COMMIT
\item ABORT
\item ACK
\item NACK
\item DISCONNECT
\end{compactitem}

Stomp客户端和Stomp服务器端进行通信时使用的包含多个行的帧(Frame)进行的,其中第一行包含命令,后面的行中包含的是key:value形式的header,接下来是一个空行,然后是body内容,最后以一个null字符结束。

Stomp客户端和Stomp服务器端的通信过程中的帧主要是MESSAGE(消息帧)、RECEIPT(收据帧)和ERROR(出错帧)等,而且不同类型的帧的header和body格式是相似的。

支持Stomp的MOM包括ActiveMQ、HornetQ、OpenMQ、RabbitMQ和Spring等。


