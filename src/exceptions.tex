\part{Exceptions}




\chapter{Introduction}

异常(Exception)用于在指定的错误发生时改变脚本的正常流程,这种情况称为异常。

PHP 5 提供了类似于其它面向对象的异常处理模块\footnote{PHP 内部函数主要使用错误报告, 只有现代面向对象的扩展才使用异常,但错误可以很容易的通过ErrorException转换为异常。PHP标准库 (SPL) 提供了许多内建的异常类。}。当异常被触发时,通常会发生的情况如下:

\begin{compactitem}
\item 当前代码状态被保存
\item 代码执行被切换到预定义的异常处理器函数
\item 根据情况,处理器也许会从保存的代码状态重新开始执行代码,终止脚本执行,或从代码中另外的位置继续执行脚本
\end{compactitem}

在 PHP 代码中所产生的异常可被 throw 语句抛出并被 catch 语句捕获。需要进行异常处理的代码都必须放入 try 代码块内,以便捕获可能存在的异常。每一个 try 至少要有一个与之对应的 catch。使用多个 catch 可以捕获不同的类所产生的异常。当 try 代码块不再抛出异常或者找不到 catch 能匹配所抛出的异常时,PHP 代码就会在跳转到最后一个 catch 的后面继续执行。当然,PHP 允许在 catch 代码块内再次抛出(throw)异常。


\section{Basic Exception}


当一个异常被抛出时,其后的代码(指抛出异常时所在的代码块)不会继续执行,而PHP 会尝试查找第一个能与之匹配的``catch" 代码块。

如果异常没有被捕获,而且又没用使用 set\_exception\_handler() 作相应的处理的话,那么 PHP 将会产生一个严重的错误(致命错误),并且输出 ``Uncaught Exception ...''(未捕获异常)的错误提示信息。



在下面的示例中,尝试抛出一个异常,同时不去捕获它:


\begin{lstlisting}[language=PHP]
<?php
//create function with an exception
function checkNum($number)
 {
 if($number>1)
  {
  throw new Exception("Value must be 1 or below");
  }
 return true;
 }

//trigger exception
checkNum(2);
?>
\end{lstlisting}

上面的代码会获得类似这样的一个错误:


\begin{lstlisting}[language=bash]
Fatal error: Uncaught exception 'Exception' 
with message 'Value must be 1 or below' in test.php:6 
Stack trace: #0 test.php(12): 
checkNum(28) #1 {main} thrown in test.php on line 6
\end{lstlisting}






\begin{example}
抛出一个异常
\begin{lstlisting}[language=PHP]
<![CDATA[
<?php
function inverse($x) {
    if (!$x) {
        throw new Exception('Division by zero.');
    }
    else return 1/$x;
}

try {
    echo inverse(5) . "\n";
    echo inverse(0) . "\n";
} catch (Exception $e) {
    echo 'Caught exception: ',  $e->getMessage(), "\n";
}

// Continue execution
echo 'Hello World';
?>
\end{lstlisting}
\end{example}

以上例程会输出:

\begin{verbatim}
0.2
Caught exception: Division by zero.
Hello World
\end{verbatim}

\begin{example}
嵌套的异常
\begin{lstlisting}[language=PHP]
<?php

class MyException extends Exception { }

class Test {
    public function testing() {
        try {
            try {
                throw new MyException('foo!');
            } catch (MyException $e) {
                /* rethrow it */
                throw $e;
            }
        } catch (Exception $e) {
            var_dump($e->getMessage());
        }
    }
}

$foo = new Test;
$foo->testing();

?>
\end{lstlisting}
\end{example}

以上例程会输出:

\begin{verbatim}
string(4) "foo!"
\end{verbatim}

\section{try,throw and catch}

要避免上面例子出现的错误,我们需要创建适当的代码来处理异常。




异常处理程序应当包括:



\begin{compactenum}
\item try - 使用异常的函数应该位于 "try" 代码块内。如果没有触发异常,则代码将照常继续执行。但是如果异常被触发,会抛出一个异常。
\item throw - 这里规定如何触发异常。每一个 "throw" 必须对应至少一个 "catch"
\item catch - "catch" 代码块会捕获异常,并创建一个包含异常信息的对象
\end{compactenum}

下面的示例将触发一个异常:

\begin{lstlisting}[language=PHP]
<?php
//创建可抛出一个异常的函数
function checkNum($number)
 {
 if($number>1)
  {
  throw new Exception("Value must be 1 or below");
  }
 return true;
 }

//在 "try" 代码块中触发异常
try
 {
 checkNum(2);
 //If the exception is thrown, this text will not be shown
 echo 'If you see this, the number is 1 or below';
 }

//捕获异常
catch(Exception $e)
 {
 echo 'Message: ' .$e->getMessage();
 }
?>
\end{lstlisting}


上面代码将获得类似这样一个错误:\verb|Message: Value must be 1 or below|

上面的代码抛出了一个异常,并捕获了它:

\begin{compactenum}
\item 创建 checkNum() 函数。它检测数字是否大于 1。如果是,则抛出一个异常。
\item 在``try" 代码块中调用 checkNum() 函数。
\item checkNum() 函数中的异常被抛出
\item ``catch" 代码块接收到该异常,并创建一个包含异常信息的对象 (\texttt{\$e})。
\item 通过从这个 exception 对象调用 \texttt{\$e->getMessage()},输出来自该异常的错误消息
\end{compactenum}

不过,为了遵循“每个 throw 必须对应一个 catch”的原则,可以设置一个顶层的异常处理器来处理漏掉的错误。



\section{Custom Exception Class}

以下这段代码只为说明内置异常处理类的结构,它并不是一段有实际意义的可用代码。

\begin{example}
内置的异常处理类
\begin{lstlisting}[language=PHP]
<?php
class Exception
{
    protected $message = 'Unknown exception';   // 异常信息
    protected $code = 0;                        // 用户自定义异常代码
    protected $file;                            // 发生异常的文件名
    protected $line;                            // 发生异常的代码行号

    function __construct($message = null, $code = 0);

    final function getMessage();                // 返回异常信息
    final function getCode();                   // 返回异常代码
    final function getFile();                   // 返回发生异常的文件名
    final function getLine();                   // 返回发生异常的代码行号
    final function getTrace();                  // backtrace() 数组
    final function getTraceAsString();          // 已格成化成字符串的 getTrace() 信息

    /* 可重载的方法 */
    function __toString();                       // 可输出的字符串
}
?>
\end{lstlisting}
\end{example}


用户可以用自定义的异常处理类来扩展 PHP 内置的异常处理类,而创建自定义的异常处理程序非常简单。

通过自定义exception类,当 PHP 中发生异常时,可调用其函数,注意该类必须是 Exception 类的一个扩展。

以下的代码说明了在内置的异常处理类中,哪些属性和方法在子类中是可访问和可继承的。


这个自定义的 exception 类继承了 PHP 的 Exception 类的所有属性,可向其添加自定义的函数。


\begin{example}
创建 exception 类
\begin{lstlisting}[language=PHP]
<?php
class customException extends Exception
 {
 public function errorMessage()
  {
  //error message
  $errorMsg = 'Error on line '.$this->getLine().' in '.$this->getFile()
  .': <b>'.$this->getMessage().'</b> is not a valid E-Mail address';
  return $errorMsg;
  }
 }

$email = "someone@example...com";

try
 {
 //check if 
 if(filter_var($email, FILTER_VALIDATE_EMAIL) === FALSE)
  {
  //throw exception if email is not valid
  throw new customException($email);
  }
 }

catch (customException $e)
 {
 //display custom message
 echo $e->errorMessage();
 }
?>
\end{lstlisting}
\end{example}

这个新的类是旧的 Exception 类的副本,外加 errorMessage() 函数。正因为它是旧类的副本,因此它从旧类继承了属性和方法,我们可以使用 Exception 类的方法,比如 getLine() 、 getFile() 以及 getMessage()。

上面的代码抛出了一个异常,并通过一个自定义的 exception 类来捕获它:

\begin{compactenum}
\item customException() 类是作为旧的 exception 类的一个扩展来创建的。这样它就继承了旧类的所有属性和方法。
\item 创建 errorMessage() 函数。如果 e-mail 地址不合法,则该函数返回一条错误消息
\item 把 \$email 变量设置为不合法的 e-mail 地址字符串
\item 执行 "try" 代码块,由于 e-mail 地址不合法,因此抛出一个异常
\item "catch" 代码块捕获异常,并显示错误消息
\end{compactenum}

如果使用自定义的类来扩展内置异常处理类,并且要重新定义构造函数的话,建议同时调用 parent::\_\_construct() 来检查所有的变量是否已被赋值。当对象要输出字符串的时候,可以重载 \_\_toString() 并自定义输出的样式。

\begin{example}
扩展 PHP 内置的异常处理类
\begin{lstlisting}[language=PHP]
<?php
/**
 * 自定义一个异常处理类
 */
class MyException extends Exception
{
    // 重定义构造器使 message 变为必须被指定的属性
    public function __construct($message, $code = 0) {
        // 自定义的代码

        // 确保所有变量都被正确赋值
        parent::__construct($message, $code);
    }

    // 自定义字符串输出的样式
    public function __toString() {
        return __CLASS__ . ": [{$this->code}]: {$this->message}\n";
    }

    public function customFunction() {
        echo "A Custom function for this type of exception\n";
    }
}


/**
 * 创建一个用于测试异常处理机制的类
 */
class TestException
{
    public $var;

    const THROW_NONE    = 0;
    const THROW_CUSTOM  = 1;
    const THROW_DEFAULT = 2;

    function __construct($avalue = self::THROW_NONE) {

        switch ($avalue) {
            case self::THROW_CUSTOM:
                // 抛出自定义异常
                throw new MyException('1 is an invalid parameter', 5);
                break;

            case self::THROW_DEFAULT:
                // 抛出默认的异常
                throw new Exception('2 isnt allowed as a parameter', 6);
                break;

            default:
                // 没有异常的情况下,创建一个对象
                $this->var = $avalue;
                break;
        }
    }
}


// 例子 1
try {
    $o = new TestException(TestException::THROW_CUSTOM);
} catch (MyException $e) {      // 捕获异常
    echo "Caught my exception\n", $e;
    $e->customFunction();
} catch (Exception $e) {        // 被忽略
    echo "Caught Default Exception\n", $e;
}

// 执行后续代码
var_dump($o);
echo "\n\n";


// 例子 2
try {
    $o = new TestException(TestException::THROW_DEFAULT);
} catch (MyException $e) {      // 不能匹配异常的种类,被忽略
    echo "Caught my exception\n", $e;
    $e->customFunction();
} catch (Exception $e) {        // 捕获异常
    echo "Caught Default Exception\n", $e;
}

// 执行后续代码
var_dump($o);
echo "\n\n";


// 例子 3
try {
    $o = new TestException(TestException::THROW_CUSTOM);
} catch (Exception $e) {        // 捕获异常
    echo "Default Exception caught\n", $e;
}

// 执行后续代码
var_dump($o);
echo "\n\n";


// 例子 4
try {
    $o = new TestException();
} catch (Exception $e) {        // 没有异常,被忽略
    echo "Default Exception caught\n", $e;
}

// 执行后续代码
var_dump($o);
echo "\n\n";
?>
\end{lstlisting}
\end{example}



\section{Multiple Exceptions}

可以为一段脚本使用多个异常,来检测多种情况。


可以使用多个 if..else 代码块,或一个 switch 代码块,或者嵌套多个异常。这些异常能够使用不同的 exception 类,并返回不同的错误消息:

\begin{lstlisting}[language=PHP]
<?php
class customException extends Exception
{
public function errorMessage()
{
//error message
$errorMsg = 'Error on line '.$this->getLine().' in '.$this->getFile()
.': <b>'.$this->getMessage().'</b> is not a valid E-Mail address';
return $errorMsg;
}
}

$email = "someone@example.com";

try
 {
 //check if 
 if(filter_var($email, FILTER_VALIDATE_EMAIL) === FALSE)
  {
  //throw exception if email is not valid
  throw new customException($email);
  }
 //check for "example" in mail address
 if(strpos($email, "example") !== FALSE)
  {
  throw new Exception("$email is an example e-mail");
  }
 }

catch (customException $e)
 {
 echo $e->errorMessage();
 }

catch(Exception $e)
 {
 echo $e->getMessage();
 }
?>
\end{lstlisting}

上面的代码测试了两种条件,如何任何条件不成立,则抛出一个异常:

\begin{compactenum}
\item customException() 类是作为旧的 exception 类的一个扩展来创建的。这样它就继承了旧类的所有属性和方法。
\item 创建 errorMessage() 函数。如果 e-mail 地址不合法,则该函数返回一个错误消息。
\item 执行 "try" 代码块,在第一个条件下,不会抛出异常。
\item 由于 e-mail 含有字符串 "example",第二个条件会触发异常。
\item "catch" 代码块会捕获异常,并显示恰当的错误消息
\end{compactenum}


如果没有捕获 customException,紧紧捕获了 base exception,则在那里处理异常。




\section{Re-throwing Exceptions}

有时,当异常被抛出时,开发者也许希望以不同于标准的方式对它进行处理,因此可以在一个 "catch" 代码块中再次抛出异常。


对程序员来说,系统错误也许很重要,但是用户对它们并不感兴趣,因此脚本应该对用户隐藏系统错误。

为了让用户更容易使用,开发者可以再次抛出带有对用户比较友好的消息的异常:

\begin{lstlisting}[language=PHP]
<?php
class customException extends Exception
 {
 public function errorMessage()
  {
  //error message
  $errorMsg = $this->getMessage().' is not a valid E-Mail address.';
  return $errorMsg;
  }
 }

$email = "someone@example.com";

try
 {
 try
  {
  //check for "example" in mail address
  if(strpos($email, "example") !== FALSE)
   {
   //throw exception if email is not valid
   throw new Exception($email);
   }
  }
 catch(Exception $e)
  {
  //re-throw exception
  throw new customException($email);
  }
 }

catch (customException $e)
 {
 //display custom message
 echo $e->errorMessage();
 }
?>
\end{lstlisting}

上面的代码检测在邮件地址中是否含有字符串 "example"。如果有,则再次抛出异常:

\begin{compactenum}
\item customException() 类是作为旧的 exception 类的一个扩展来创建的。这样它就继承了旧类的所有属性和方法。
\item 创建 errorMessage() 函数。如果 e-mail 地址不合法,则该函数返回一个错误消息。
\item 把 \$email 变量设置为一个有效的邮件地址,但含有字符串``example"。
\item ``try" 代码块包含另一个``try" 代码块,这样就可以再次抛出异常。
\item 由于 e-mail 包含字符串``example",因此触发异常。
\item ``catch" 捕获到该异常,并重新抛出``customException"。
\item 捕获到``customException",并显示一条错误消息。
\end{compactenum}

如果在其目前的``try" 代码块中异常没有被捕获,则它将在更高层级上查找 catch 代码块。


\section{Top Level Exception Handler}

set\_exception\_handler() 函数可设置处理所有未捕获异常的用户定义函数。

\begin{lstlisting}[language=PHP]
<?php
function myException($exception)
{
echo "<b>Exception:</b> " , $exception->getMessage();
}

set_exception_handler('myException');

throw new Exception('Uncaught Exception occurred');
?>
\end{lstlisting}

以上代码的输出应该类似这样:\verb|Exception:Uncaught Exception ocuured|

在上面的代码中,不存在 "catch" 代码块,而是触发顶层的异常处理程序。应该使用此函数来捕获所有未被捕获的异常。





\section{Rules for Exception}

异常处理的规则包括:

\begin{compactitem}
\item 需要进行异常处理的代码应该放入 try 代码块内,以便捕获潜在的异常。
\item 每个 try 或 throw 代码块必须至少拥有一个对应的 catch 代码块。
\item 使用多个 catch 代码块可以捕获不同种类的异常。
\item 可以在 try 代码块内的 catch 代码块中再次抛出(re-thrown)异常。
\end{compactitem}

简而言之就是,如果抛出了异常,就必须捕获它。


\chapter{Exception}


Exception是所有异常的基类\footnote{自PHP 5.1.0起,引入Exception类。}。

\textbf{Exception类摘要}

\begin{lstlisting}[language=PHP]
Exception {
  /* 属性 */
  protected string $message ;
  protected int $code ;
  protected string $file ;
  protected int $line ;
  /* 方法 */
  public __construct ([ string $message = "" [, int $code = 0 [, Exception $previous = NULL ]]] )
  final public string getMessage ( void )
  final public Exception getPrevious ( void )
  final public int getCode ( void )
  final public string getFile ( void )
  final public int getLine ( void )
  final public array getTrace ( void )
  final public string getTraceAsString ( void )
  public string __toString ( void )
  final private void __clone ( void )
}
\end{lstlisting}


\textbf{Exception类属性}

\begin{compactitem}
\item message

异常消息内容
\item code

异常代码
\item file

抛出异常的文件名
\item line

抛出异常在该文件中的行号

\end{compactitem}



\textbf{Exception类方法}

\begin{compactitem}
\item Exception::\_\_construct — 异常构造函数
\item Exception::getMessage — 获取异常消息内容
\item Exception::getPrevious — 返回异常链中的前一个异常
\item Exception::getCode — 获取异常代码
\item Exception::getFile — 获取发生异常的程序文件名称
\item Exception::getLine — 获取发生异常的代码在文件中的行号
\item Exception::getTrace — 获取异常追踪信息
\item Exception::getTraceAsString — 获取字符串类型的异常追踪信息
\item Exception::\_\_toString — 将异常对象转换为字符串
\item Exception::\_\_clone — 异常克隆
\end{compactitem}

\section{Exception::\_\_construct}


Exception::\_\_construct —— 异常构造函数\footnote{自PHP 5.3.0起,增加previous参数。}。


\begin{lstlisting}[language=PHP]
public Exception::__construct() ([ string $message = "" [, int $code = 0 [, Exception $previous = NULL ]]] )
\end{lstlisting}




\textbf{参数}

\begin{compactitem}
\item message
抛出的异常消息内容,The message is NOT binary safe.
\item code
异常代码。
\item previous
异常链中的前一个异常。
\end{compactitem}


\section{Exception::getMessage}

Exception::getMessage — 获取异常消息内容,返回字符串类型的异常消息内容。此函数没有参数。

\begin{lstlisting}[language=PHP]
final public string Exception::getMessage ( void )
\end{lstlisting}




\begin{example}
Exception::getMessage()示例
\begin{lstlisting}[language=PHP]
<?php
try {
    throw new Exception("Some error message");
} catch(Exception $e) {
    echo $e->getMessage();
}
?>
\end{lstlisting}
\end{example}

以上例程的输出类似于:

\begin{verbatim}
Some error message
\end{verbatim}

\section{Exception::getPrevious}

Exception::getPrevious — 返回异常链中的前一个异常( Exception::\_\_construct()方法的第三个参数),否则返回NULL。此函数没有参数。

\begin{lstlisting}[language=PHP]
final public Exception Exception::getPrevious ( void )
\end{lstlisting}



\begin{example}
Exception::getPrevious()示例 - 追踪异常,并循环打印。
\begin{lstlisting}[language=PHP]
<?php
class MyCustomException extends Exception {}

function doStuff() {
    try {
        throw new InvalidArgumentException("You are doing it wrong!", 112);
    } catch(Exception $e) {
        throw new MyCustomException("Something happend", 911, $e);
    }
}


try {
    doStuff();
} catch(Exception $e) {
    do {
        printf("%s:%d %s (%d) [%s]\n", $e->getFile(), $e->getLine(), $e->getMessage(), $e->getCode(), get_class($e));
    } while($e = $e->getPrevious());
}
?>
\end{lstlisting}
\end{example}


以上例程的输出类似于:

\begin{verbatim}
/home/bjori/ex.php:8 Something happend (911) [MyCustomException]
/home/bjori/ex.php:6 You are doing it wrong! (112) [InvalidArgumentException]
\end{verbatim}




\section{Exception::getCode}


Exception::getCode — 获取并返回整型(integer)类型的异常代码,此函数没有参数。


\begin{lstlisting}[language=PHP]
final public int Exception::getCode ( void )
\end{lstlisting}



\begin{example}
Exception::getCode()示例
\begin{lstlisting}[language=PHP]
<?php
try {
    throw new Exception("Some error message", 30);
} catch(Exception $e) {
    echo "The exception code is: " . $e->getCode();
}
?>
\end{lstlisting}
\end{example}

以上例程的输出类似于:


\begin{verbatim}
The exception code is: 30
\end{verbatim}

\section{Exception::getFile}


Exception::getFile — 获取并返回发生异常的程序文件名称,此函数没有参数。

\begin{lstlisting}[language=PHP]
final public string Exception::getFile ( void )
\end{lstlisting}



\begin{example}
Exception::getFile()示例
\begin{lstlisting}[language=PHP]
<?php
try {
    throw new Exception;
} catch(Exception $e) {
    echo $e->getFile();
}
?>
\end{lstlisting}
\end{example}

以上例程的输出类似于:

\begin{verbatim}
/home/bjori/tmp/ex.php
\end{verbatim}


\section{Exception::getLine}

Exception::getLine — 获取并返回发生异常的代码在文件中的行号,此函数没有参数。


\begin{lstlisting}[language=PHP]
final public int Exception::getLine ( void )
\end{lstlisting}




\begin{example}
Exception::getLine()示例
\begin{lstlisting}[language=PHP]
<?php
try {
    throw new Exception("Some error message");
} catch(Exception $e) {
    echo "The exception was thrown on line: " . $e->getLine();
}
?>
\end{lstlisting}
\end{example}

以上例程的输出类似于:

\begin{verbatim}
The exception was thrown on line: 3
\end{verbatim}


\section{Exception::getTrace}

Exception::getTrace — 获取并返回异常追踪信息的数组(array),此函数没有参数。

\begin{lstlisting}[language=PHP]
final public array Exception::getTrace ( void )
\end{lstlisting}




\begin{example}
Exception::getTrace()示例
\begin{lstlisting}[language=PHP]
<?php
function test() {
 throw new Exception;
}

try {
 test();
} catch(Exception $e) {
 var_dump($e->getTrace());
}
?>
\end{lstlisting}
\end{example}

以上例程的输出类似于:

\begin{verbatim}
array(1) {
  [0]=>
  array(4) {
    ["file"]=>
    string(22) "/home/bjori/tmp/ex.php"
    ["line"]=>
    int(7)
    ["function"]=>
    string(4) "test"
    ["args"]=>
    array(0) {
    }
  }
}
\end{verbatim}


\section{Exception::getTraceAsString}


Exception::getTraceAsString — 获取并返回字符串类型的异常追踪信息。此函数没有参数。


\begin{lstlisting}[language=PHP]
final public string Exception::getTraceAsString ( void )
\end{lstlisting}



\begin{example}
Exception::getTraceAsString()示例
\begin{lstlisting}[language=PHP]
<?php
function test() {
    throw new Exception;
}

try {
    test();
} catch(Exception $e) {
    echo $e->getTraceAsString();
}
?>
\end{lstlisting}
\end{example}

以上例程的输出类似于:

\begin{verbatim}
#0 /home/bjori/tmp/ex.php(7): test()
#1 {main}
\end{verbatim}

\section{Exception::\_\_toString}

Exception::\_\_toString — 将异常对象转换为字符串并返回。此函数没有参数。

\begin{lstlisting}[language=PHP]
public string Exception::__toString ( void )
\end{lstlisting}



\begin{example}
Exception::\_\_toString()示例
\begin{lstlisting}[language=PHP]
<?php
try {
    throw new Exception("Some error message");
} catch(Exception $e) {
    echo $e;
}
?>
\end{lstlisting}
\end{example}

以上例程的输出类似于:

\begin{verbatim}
exception 'Exception' with message 'Some error message' in /home/bjori/tmp/ex.php:3
Stack trace:
#0 {main}
\end{verbatim}


\section{Exception::\_\_clone}

Exception::\_\_clone — 尝试异常克隆。因为异常被不允许克隆,因而这将导致一个致命错误。此函数没有参数,没有返回值。

\begin{lstlisting}[language=PHP]
final private void Exception::__clone ( void )
\end{lstlisting}



\chapter{ErrorException}

错误异常\footnote{自PHP 5.1.0起,引入ErrorException类。}。

\textbf{ErrorException类摘要}



\begin{lstlisting}[language=PHP]
ErrorException extends Exception {
  /* 属性 */
   protected int $severity ;
  /* 方法 */
  public __construct ([ string $message = "" [, int $code = 0 [, int $severity = 1 [, string $filename = __FILE__ [, int $lineno = __LINE__ [, Exception $previous = NULL ]]]]]] )
  final public int getSeverity ( void )
  /* 继承的方法 */
  final public string Exception::getMessage ( void )
  final public Exception Exception::getPrevious ( void )
  final public int Exception::getCode ( void )
  final public string Exception::getFile ( void )
  final public int Exception::getLine ( void )
  final public array Exception::getTrace ( void )
  final public string Exception::getTraceAsString ( void )
  public string Exception::__toString ( void )
  final private void Exception::__clone ( void )
}
\end{lstlisting}

\textbf{ErrorException类属性}

\begin{compactitem}
\item severity
异常级别
\end{compactitem}


\textbf{ErrorException类方法}

\begin{compactitem}
\item ErrorException::\_\_construct — 异常构造函数
\item ErrorException::getSeverity — 获取异常的严重程度
\end{compactitem}



\begin{example}
使用 set\_error\_handler()函数将错误信息托管至ErrorException
\begin{lstlisting}[language=PHP]
<?php
function exception_error_handler($errno, $errstr, $errfile, $errline ) {
    throw new ErrorException($errstr, 0, $errno, $errfile, $errline);
}
set_error_handler("exception_error_handler");

/* Trigger exception */
strpos();
?>
\end{lstlisting}
\end{example}

以上例程的输出类似于:

\begin{verbatim}
Fatal error: Uncaught exception 'ErrorException' with message 'Wrong parameter count for strpos()' in /home/bjori/tmp/ex.php:8
Stack trace:
#0 [internal function]: exception_error_handler(2, 'Wrong parameter...', '/home/bjori/php...', 8, Array)
#1 /home/bjori/php/cleandocs/test.php(8): strpos()
#2 {main}
  thrown in /home/bjori/tmp/ex.php on line 8
\end{verbatim}



\section{ErrorException::\_\_construct}


ErrorException::\_\_construct — 异常构造函数\footnote{自PHP 5.3.0起,增加previous参数。}。

\begin{lstlisting}[language=PHP]
public ErrorException::__construct() ([ string $message = "" [, int $code = 0 [, int $severity = 1 [, string $filename = __FILE__ [, int $lineno = __LINE__ [, Exception $previous = NULL ]]]]]] )
\end{lstlisting}

\begin{compactitem}
\item message
抛出的异常消息内容。
\item code
异常代码。
\item severity
异常的严重级别。
\item filename
抛出异常所在的文件名。
\item lineno
抛出异常所在的行号。
\item previous
异常链中的前一个异常。
\end{compactitem}


\section{ErrorException::getSeverity}

ErrorException::getSeverity — 获取并返回异常的严重程度,此函数没有参数。


\begin{lstlisting}[language=PHP]
final public int ErrorException::getSeverity ( void )
\end{lstlisting}


\begin{example}
ErrorException::getSeverity() 例子
\begin{lstlisting}[language=PHP]
<?php
try {
    throw new ErrorException("Exception message", 0, 75);
} catch(ErrorException $e) {
    echo "This exception severity is: " . $e->getSeverity();
}
?>
\end{lstlisting}
\end{example}

以上例程的输出类似于:

\begin{verbatim}
This exception severity is: 75
\end{verbatim}









\bibliographystyle{plainnat}
\bibliography{phpnotes}
\clearpage
