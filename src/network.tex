\part{Network}



\chapter{Overview}



PHP核心提供的网络函数族包含一系列联网函数。

\section{Requirement}

checkdnsrr()、getmxrr()和dns\_get\_record()函数需要Linux系统中安装有Bind来提供支持。

\section{Resource Type}

PHP为网络函数族中的fsockopen()和pfsockopen()定义了一个文件指针资源类型。


\section{Build-in Module}

联网函数由PHP核心直接支持。

\section{Runtime Configure}

联网函数的行为受到php.ini中的define\_syslog\_variables配置项的影响。



\begin{longtable}{|m{100pt}|m{30pt}|m{50pt}|m{100pt}|}
%head
\multicolumn{4}{r}{}
\tabularnewline\hline
名字&默认&可修改范围&说明
\endhead
%endhead

%firsthead
\caption{Network Configuration选项}\\
\hline
名字&默认&可修改范围&说明
\endfirsthead
%endfirsthead

%foot
\multicolumn{4}{r}{}
\endfoot
%endfoot

%lastfoot
\endlastfoot
%endlastfoot
\hline
define\_syslog\_variables&\texttt{"0"}&PHP\_INI\_ALL&已移除\\
\hline
\end{longtable}

define\_syslog\_variables配置项说明是否定义各种syslog变量(例如\$LOG\_PID和\$LOG\_CRON等),关闭或移除define\_syslog\_variables配置项可以提供性能,推荐用户在运行时通过调用define\_syslog\_variables()定义这些变量。



\section{Predefined Constants}

PHP为网络函数族预先定义了一些常量,并且这些常量作为PHP核心的一部分总是可用的。


\begin{longtable}{|m{100pt}|m{300pt}|}
%head
\multicolumn{2}{r}{}
\tabularnewline\hline
常量名&说明
\endhead
%endhead

%firsthead
\caption{用于openlog()选项的常量}\\
\hline
常量名&说明
\endfirsthead
%endfirsthead

%foot
\multicolumn{2}{r}{}
\endfoot
%endfoot

%lastfoot
\endlastfoot
%endlastfoot
\hline
LOG\_CONS	&if there is an error while sending data to the system logger, write directly to the system console\\
\hline
LOG\_NDELAY	&open the connection to the logger immediately\\
\hline
LOG\_ODELAY&(default) delay opening the connection until the first message is logged\\
\hline
LOG\_NOWAIT&\\
\hline
LOG\_PERROR&print log message also to standard error\\
\hline
LOG\_PID	&include PID with each message\\
\hline
\end{longtable}



\begin{longtable}{|m{100pt}|m{300pt}|}
%head
\multicolumn{2}{r}{}
\tabularnewline\hline
常量名&说明
\endhead
%endhead

%firsthead
\caption{用于openlog()设备的常量}\\
\hline
常量名&说明
\endfirsthead
%endfirsthead

%foot
\multicolumn{2}{r}{}
\endfoot
%endfoot

%lastfoot
\endlastfoot
%endlastfoot
\hline
LOG\_AUTH	&security/authorization messages (use LOG\_AUTHPRIV instead in systems where that constant is defined)\\
\hline
LOG\_AUTHPRIV&	security/authorization messages (private)\\
\hline
LOG\_CRON	&clock daemon (cron and at)\\
\hline
LOG\_DAEMON&other system daemons\\
\hline
LOG\_KERN	&kernel messages\\
\hline
LOG\_LOCAL0 ... LOG\_LOCAL7&reserved for local use, these are not available in Windows\\
\hline
LOG\_LPR	&line printer subsystem\\
\hline
LOG\_MAIL	&mail subsystem\\
\hline
LOG\_NEWS	&USENET news subsystem\\
\hline
LOG\_SYSLOG	&messages generated internally by syslogd\\
\hline
LOG\_USER	&generic user-level messages\\
\hline
LOG\_UUCP	&UUCP subsystem\\
\hline
\end{longtable}




\begin{longtable}{|m{100pt}|m{300pt}|}
%head
\multicolumn{2}{r}{}
\tabularnewline\hline
常量名&说明
\endhead
%endhead

%firsthead
\caption{用于syslog()的优先级(按降序排列)的常量}\\
\hline
常量名&说明
\endfirsthead
%endfirsthead

%foot
\multicolumn{2}{r}{}
\endfoot
%endfoot

%lastfoot
\endlastfoot
%endlastfoot
\hline
LOG\_EMERG	&system is unusable\\
\hline
LOG\_ALERT	&action must be taken immediately\\
\hline
LOG\_CRIT	&critical conditions\\
\hline
LOG\_ERR	&error conditions\\
\hline
LOG\_WARNING&warning conditions\\
\hline
LOG\_NOTICE	&normal, but significant, condition\\
\hline
LOG\_INFO	&informational message\\
\hline
LOG\_DEBUG	&debug-level message\\
\hline
\end{longtable}





\begin{longtable}{|m{100pt}|m{300pt}|}
%head
\multicolumn{2}{r}{}
\tabularnewline\hline
常量名&说明
\endhead
%endhead

%firsthead
\caption{用于dns\_get\_record()选项的常量}\\
\hline
常量名&说明
\endfirsthead
%endfirsthead

%foot
\multicolumn{2}{r}{}
\endfoot
%endfoot

%lastfoot
\endlastfoot
%endlastfoot
\hline
DNS\_A	&IPv4 Address Resource\\
\hline
DNS\_MX&Mail Exchanger Resource\\
\hline
DNS\_CNAME&Alias (Canonical Name) Resource\\
\hline
DNS\_NS	&Authoritative Name Server Resource\\
\hline
DNS\_PTR&Pointer Resource\\
\hline
DNS\_HINFO&Host Info Resource \\
\hline
DNS\_SOA&Start of Authority Resource\\
\hline
DNS\_TXT&Text Resource\\
\hline
DNS\_ANY&Any Resource Record. On most systems this returns all resource records, however it should not be counted upon for critical uses. Try DNS\_ALL instead.\\
\hline
DNS\_AAAA&IPv6 Address Resource\\
\hline
DNS\_ALL&Iteratively query the name server for each available record type.\\
\hline
\end{longtable}


\chapter{Functions}


\section{checkdnsrr()}

给指定的主机(域名)或者IP地址做DNS通信检查。

\section{closelog()}

关闭系统日志链接。

\section{define\_syslog\_variables()}

初始化所有syslog相关的变量。

\section{dns\_check\_record()}

别名checkdnsrr()。

\section{dns\_get\_mx()}

别名getmxrr()。

\section{dns\_get\_record()}

获取指定主机的DNS记录。

\section{fsockopen()}

打开一个网络连接或者一个UNIX套接字连接。

\section{gethostbyaddr()}

获取指定的IP地址对应的主机名。

\section{gethostbyname()}

获取与给定Internet主机名对应的IPv4地址。

\section{gethostbynamel()}

获取与给定Internet主机名对应的IPv4地址的列表。

\section{gethostname()}

获取主机名。

\section{getmxrr()}

获取与给定Internet主机名对应的MX记录。

\section{getprotobyname()}

获取与协议名称相关的协议号。

\section{getprotobynumber()}

获取和协议号相关的协议名。

\section{getservbyname()}

获取与Internet服务和协议相关联的端口号

\section{getservbyport()}

获取对应端口和协议的Internet服务。

\section{header\_register\_callback()}

注册将在PHP开始发送输出时调用的回调函数。

\section{header\_remove()}

删除以前使用header()设置的旧的HTTP标头。


\section{header()}

发送原生HTTP头。

\section{headers\_list()}

返回发送(或准备发送)的响应头的列表。

\section{headers\_sent()}

检查是否已经发送或在何处发送了标头。

\section{http\_response\_code()}

获取或设置HTTP响应代码

\section{inet\_ntop()}

将打包的互联网地址转换为人类可读的表示形式。

\section{inet\_pton()}

将人类可读的IP地址转换为其打包的in\_addr表示形式


\section{ip2long()}

将一个IPV4的字符串互联网协议转换成数字格式

\section{long2ip()}

将长整型地址转换为IPv4格式的Internet标准点分隔格式的字符串

\section{openlog()}

打开与系统日志记录器的连接

\section{pfsockopen()}

打开一个持久的网络连接或者Unix套接字连接。

\section{setcookie()}

发送cookie。

\section{setrawcookie()}

发送没有对cookie值进行编码的cookie


\section{socket\_get\_status()}

别名stream\_get\_meta\_data()

\section{socket\_set\_blocking()}

别名stream\_set\_blocking()

\section{socket\_set\_timeout()}

别名stream\_set\_timeout()

\section{syslog()}

生成一条系统日志消息。


















