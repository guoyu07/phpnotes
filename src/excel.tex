\part{Excel}


\chapter{Overview}






\chapter{PHPExcel}



\section{PHPOffice/PHPExcel}


PHPExcel是一个读写不同格式的电子表格文件的PHP库,用户可以使用PHPExcel处理Excel (BIFF) .xls, Excel 2007 (OfficeOpenXML) .xlsx, CSV, Libre/OpenOffice Calc .ods, Gnumeric, PDF和HTML等类型的文件。

PHPExcel遵循Microsoft OpenXML标准,并且可以设置电子表格文件的元数据(作者、标题和描述等)、多工作表、多字体和字体样式、单元格、填充、渐变、图片、公式和文件类型转换等操作。



\begin{compactenum}
\item Reading

\begin{compactitem}
\item BIFF 5-8 (.xls) Excel 95和更高
\item Office Open XML (.xlsx) Excel 2007和更高
\item SpreadsheetML (.xml) Excel 2003
\item Open Document Format/OASIS (.ods)
\item Gnumeric
\item HTML
\item SYLK
\item CSV
\end{compactitem}

\item Writing

\begin{compactitem}
\item BIFF 8 (.xls) Excel 95 and above
\item Office Open XML (.xlsx) Excel 2007 and above
\item HTML
\item CSV
\item PDF (需要独立安装tcPDF, DomPDF或mPDF库)
\end{compactitem}

\end{compactenum}



如果需要在项目中使用PHPExcel,可以和PHPExcel的源文件放入项目的指定目录下,这样就可以在项目中集成对电子表格文件的读写功能,例如:

\begin{lstlisting}
/var/www/Classes/PHPExcel.php
/var/www/Classes/PHPExcel/Calculation.php
/var/www/Classes/PHPExcel/Cell.php
...
\end{lstlisting}

为了了解PHPExcel提供的具体功能,可以把Examples目录放入项目的指定目录下,并且在浏览器中访问指定的文件。

\begin{lstlisting}
/var/www/Examples/01simple.php
/var/www/Examples/02types.php
...
\end{lstlisting}

可以通过如下链接来访问对应的示例文件来测试PHPExcel的功能:

\begin{compactitem}
\item \url{http://example.com/Tests/01simple.php}
\item \url{http://example.com/Tests/02types.php}
\end{compactitem}




\subsection{Requirement}

PHPExcel需要zip、xml和gd2(可选)扩展的支持,并且要求在所有的文件中统一设置UTF-8编码。

如果要处理的文件编码不是UTF-8,那么可以使用PHP提供的iconv()或mb\_convert\_encoding()函数转换后再使用PHPExcel进行操作。


从Excel2007和OOCalc开始在内部使用Zip压缩,因此PHPExcel的ZipArchive类要求启用PHP的zip扩展,同时PHPExcel还内置了PCLZip库来作为ZipArchive类的替代。

在调用Excel2007 Writer的同名方法之前,可以通过如下的设置来启用PCLZip:

\begin{lstlisting}[language=PHP]
<?php
PHPExcel_Settings::setZipClass(PHPExcel_Settings::PCLZIP);
\end{lstlisting}

如果需要继续使用ZipArchive类,可以使用如下的设置来启用ZipArchive类:

\begin{lstlisting}[language=PHP]
<?php
PHPExcel_Settings::setZipClass(PHPExcel_Settings::ZIPARCHIVE);
\end{lstlisting}

\subsection{Intergration}

如果需要在Yii中使用PHPExcel,可以在protected/config/main.php添加如下的配置项:


\begin{lstlisting}[language=PHP]
<?php
// application components
'components'=>array(
   'excel'=>array(
        'class'=>'application.extensions.PHPExcel',
   ),
)
\end{lstlisting}

接下来需要修改protected/extensions/PHPExcel/Autoloader.php中的Register()方法:

\begin{lstlisting}[language=PHP]
<?php
// ...
public static function Register() {
    $functions = spl_autoload_function();
    foreach($functions as $function) {
        spl_autoload_unregister($function);
        $functions = array_merge(array(array('PHPExcel_Autoloader','Load')),$functions);
        foreach($functions as $function){
            $x = spl_autoload_register($function);
            return x;
        }
    }
}
\end{lstlisting}


在PHP5.3以上的版本中,可以直接使用如下的代码来引入PHPExcel库,例如:

\begin{lstlisting}[language=PHP]
<?php
// ...
public static function Register() {
    return spl_autoload_register(array('PHPExcel_Autoloader','Load'),false,true);
}
\end{lstlisting}

在Yii项目使用PHPExcel的示例如下:


\begin{lstlisting}[language=PHP]
<?php
public function actionExcel() {
    $objPHPExcel = new PHPExcel();

    $objPHPExcel->setActiveSheetIndex(0)
        ->setCellValue('A1', 'Hello')
        ->setCellValue('B2', 'world!')
        ->setCellValue('C1', 'Hello')
        ->setCellValue('D2', 'world!');

    $objPHPExcel->setActiveSheetIndex(0)
        ->setCellValue('A4', 'Miscellaneous glyphs')
        ->setCellValue('A5', 'eaeuaeioueiuyaouc');

    $objPHPExcel->getActiveSheet()->setTitle('Simple');

    $objPHPExcel->setActiveSheetIndex(0);

    ob_end_clean();
    ob_start();

    header('Content-Type: application/vnd.ms-excel');
    header('Content-Disposition: attachment;filename="test.xls"');
    header('Cache-Control: max-age=0');
    $objWriter = PHPExcel_IOFactory::createWriter($objPHPExcel, 'Excel5');
    $objWriter->save('php://output');
}
\end{lstlisting}


\subsection{Resource Type}


\subsection{Build-in Module}



\subsection{Runtime Configure}


\subsection{Predefined Constructs}




