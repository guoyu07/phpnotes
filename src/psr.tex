\part{PSR}


\chapter{Overview}




PSR 是 PHP Standard Recommendations 的简写,由 PHP FIG 组织制定的 PHP 规范,是 PHP 开发的实践标准。

用户可以使用PHP\_CodeSniffer 来检查代码是否符合PSR代码规范。例如,可以通过这些工具来实时地自动修正程序语法使其符合标准。 

\begin{compactitem}
\item PHP Coding Standards Fixer
\item php.tools(sublime-phpfmt)
\end{compactitem}


也可以手动运行 phpcs 命令来输出相应的错误以及如何修正的方法。同样地,phpcs也可以用在 git hook 中,这样如果分支代码不符合选择的代码标准则无法提交。

\begin{lstlisting}[language=bash]
$ phpcs -sw --standard=PSR2 file.php
\end{lstlisting}





\section{Introduction}

FIG 是 Framework Interoperability Group(框架可互用性小组)的缩写,由几位开源框架的开发者成立于 2009 年,从那开始也选取了很多其他成员进来(包括但不限于 Laravel, Joomla, Drupal, Composer, Phalcon, Slim, Symfony, Zend Framework 等),虽然不是「官方」组织,但也代表了大部分的 PHP 社区。

FIG项目的目的在于:通过框架作者或者框架的代表之间讨论,以最低程度的限制,制定一个协作标准,各个框架遵循统一的编码规范,避免各家自行发展的风格阻碍了 PHP 的发展,解决这个程序设计师由来已久的困扰。

目前已表决通过了的6套PSR标准,废弃了1套标准(自动加载规范),已经得到大部分 PHP 框架的支持和认可。

\begin{compactitem}
\item 基础编码规范
\item 编码风格规范
\item 日志接口规范
\item 自动加载规范
\item 缓存接口规范
\item HTTP 消息接口规范
\end{compactitem}

正在起草中的规范包括:

\begin{compactitem}
\item PHPDoc 标准
\item Huggable 接口
\item 项目安全问题公示
\item 项目安全上报方法
\item 服务容器接口
\item 全量编码风格规范
\item 超媒体链接
\end{compactitem}


\chapter{PSR-1}


\section{Overview}

\begin{compactitem}
\item PHP代码文件 必须 以 \texttt{<?php} 或 \texttt{<?=} 标签开始;

\item PHP代码文件 必须 以 不带 BOM 的 UTF-8 编码;

\item PHP代码中 应该 只定义类、函数、常量等声明,或其他会产生 副作用 的操作(例如生成文件输出以及修改 .ini 配置文件等),二者只能选其一;

\item 命名空间以及类 必须 符合 PSR 的自动加载规范:PSR-4 中的一个;

\item 类的命名 必须 遵循 StudlyCaps 大写开头的驼峰命名规范;

\item 类中的常量所有字母都 必须 大写,单词间用下划线分隔;

\item 方法名称 必须 符合 camelCase 式的小写开头驼峰命名规范。

\end{compactitem}





\section{File}

\subsection{PHP Tag}


PHP代码 必须 使用 \texttt{<?php ?>} 长标签 或 \texttt{<?= ?>} 短输出标签; 一定不可 使用其它自定义标签。

\subsection{PHP Charset}

PHP代码 必须 且只可使用 不带BOM的UTF-8 编码。


\subsection{Side Effect}

一份 PHP 文件中 应该 要不就只定义新的声明,如类、函数或常量等不产生 副作用 的操作,要不就只书写会产生 副作用 的逻辑操作,但 不该 同时具有两者。

「副作用」(side effects) 一词的意思是,仅仅通过包含文件,不直接声明类、函数和常量等,而执行的逻辑操作。

「副作用」包含却不仅限于:

\begin{compactitem}
\item 生成输出
\item 直接的 require 或 include
\item 连接外部服务
\item 修改 ini 配置
\item 抛出错误或异常
\item 修改全局或静态变量
\item 读或写文件等
\end{compactitem}

以下是一个 反例,一份包含「函数声明」以及产生「副作用」的代码:

\begin{lstlisting}[language=PHP]
<?php
// 「副作用」:修改 ini 配置
ini_set('error_reporting', E_ALL);

// 「副作用」:引入文件
include "file.php";

// 「副作用」:生成输出
echo "<html>\n";

// 声明函数
function foo()
{
    // 函数主体部分
}
\end{lstlisting}

下面是一个范例,一份只包含声明不产生「副作用」的代码:

\begin{lstlisting}[language=PHP]
<?php
// 声明函数
function foo()
{
    // 函数主体部分
}

// 条件声明 **不** 属于「副作用」
if (! function_exists('bar')) {
    function bar()
    {
        // 函数主体部分
    }
}
\end{lstlisting}


\section{Namespace}

命名空间以及类的命名必须遵循 PSR-4。

根据规范,每个类都独立为一个文件,且命名空间至少有一个层次:顶级的组织名称(vendor name)。

类的命名 必须 遵循 StudlyCaps 大写开头的驼峰命名规范。

PHP 5.3 及以后版本的代码 必须 使用正式的命名空间。


\begin{lstlisting}[language=PHP]
<?php
// PHP 5.3及以后版本的写法
namespace Vendor\Model;

class Foo
{
}
\end{lstlisting}

5.2.x 及之前的版本 应该 使用伪命名空间的写法,约定俗成使用顶级的组织名称(vendor name)如 Vendor\_ 为类前缀。

\begin{lstlisting}[language=PHP]
<?php
// 5.2.x及之前版本的写法
class Vendor_Model_Foo
{
}
\end{lstlisting}

\section{Class Member}

此处的「类」指代所有的类、接口以及可复用代码块(traits)。

\subsection{Constants}

类的常量中所有字母都 必须 大写,词间以下划线分隔。




\begin{lstlisting}[language=PHP]
<?php
namespace Vendor\Model;

class Foo
{
    const VERSION = '1.0';
    const DATE_APPROVED = '2012-06-01';
}
\end{lstlisting}


\subsection{Property}

类的属性命名 可以 遵循:

\begin{compactitem}
\item 大写开头的驼峰式 (\$StudlyCaps)
\item 小写开头的驼峰式 (\$camelCase)
\item 下划线分隔式 (\$under\_score)
\end{compactitem}

本规范不做强制要求,但无论遵循哪种命名方式,都 应该 在一定的范围内保持一致。这个范围可以是整个团队、整个包、整个类或整个方法。

\subsection{Method}


方法名称 必须 符合 camelCase() 式的小写开头驼峰命名规范。

\chapter{PSR-2}

PSR-2的代码 必须 符合 PSR-1 中的所有规范。

\section{Overview}

\begin{compactitem}
\item 代码 必须 遵循 PSR-1 中的编码规范 。

\item 代码 必须 使用4个空格符而不是「Tab 键」进行缩进。

\item 每行的字符数 应该 软性保持在 80 个之内,理论上 一定不可 多于 120 个,但 一定不可 有硬性限制。

\item 每个 namespace 命名空间声明语句和 use 声明语句块后面,必须 插入一个空白行。

\item 类的开始花括号(\{) 必须 写在函数声明后自成一行,结束花括号(\})也 必须 写在函数主体后自成一行。

\item 方法的开始花括号(\{) 必须 写在函数声明后自成一行,结束花括号(\})也 必须 写在函数主体后自成一行。

\item 类的属性和方法 必须 添加访问修饰符(private、protected 以及 public),abstract 以及 final 必须 声明在访问修饰符之前,而 static 必须 声明在访问修饰符之后。

\item 控制结构的关键字后 必须 要有一个空格符,而调用方法或函数时则 一定不可 有。

\item 控制结构的开始花括号(\{) 必须 写在声明的同一行,而结束花括号(\}) 必须 写在主体后自成一行。

\item 控制结构的开始左括号后和结束右括号前,都 一定不可 有空格符。

\end{compactitem}



\begin{lstlisting}[language=PHP]
<?php
namespace Vendor\Package;

use FooInterface;
use BarClass as Bar;
use OtherVendor\OtherPackage\BazClass;

class Foo extends Bar implements FooInterface
{
    public function sampleFunction($a, $b = null)
    {
        if ($a === $b) {
            bar();
        } elseif ($a > $b) {
            $foo->bar($arg1);
        } else {
            BazClass::bar($arg2, $arg3);
        }
    }

    final public static function bar()
    {
        // 方法的内容
    }
}
\end{lstlisting}

\section{File}

\begin{compactitem}
\item 所有PHP文件 必须 使用 Unix LF (linefeed) 作为行的结束符。
\item 所有PHP文件 必须 以一个空白行作为结束。
\item 纯PHP代码文件 必须 省略最后的 \texttt{?>} 结束标签。
\end{compactitem}

\section{Line}

\begin{compactitem}
\item 行的长度 一定不可 有硬性的约束。

\item 软性的长度约束 必须 要限制在 120 个字符以内,若超过此长度,带代码规范检查的编辑器 必须 要发出警告,不过 一定不可 发出错误提示。

\item 每行 不该 多于80个字符,大于80字符的行 应该 折成多行。

\item 非空行后 一定不可 有多余的空格符。

\item 空行 可以 使得阅读代码更加方便以及有助于代码的分块。

\item 每行 一定不可 存在多于一条语句。
\end{compactitem}

\section{Indent}

代码 必须 使用4个空格符的缩进,一定不可 用 tab键。

使用空格而不是「tab键缩进」的好处在于, 避免在比较代码差异、打补丁、重阅代码以及注释时产生混淆。 并且,使用空格缩进,让对齐变得更方便。

\section{Keyword}

\begin{compactitem}
\item PHP所有 关键字 必须 全部小写。
\item 常量 true 、false 和 null 也 必须 全部小写。
\end{compactitem}


\section{Namespace}

\begin{compactitem}
\item namespace 声明后 必须 插入一个空白行。

\item 所有 use 必须 在 namespace 后声明。

\item 每条 use 声明语句 必须 只有一个 use 关键词。

\item use 声明语句块后 必须 要有一个空白行。

\end{compactitem}


\begin{lstlisting}[language=PHP]
<?php
namespace Vendor\Package;

use FooClass;
use BarClass as Bar;
use OtherVendor\OtherPackage\BazClass;

// ... 更多的 PHP 代码在这里 ...
\end{lstlisting}

\section{Class Member}

此处的「类」泛指所有的「class类」、「接口」以及「traits 可复用代码块」。

\subsection{Extend}

关键词 extends 和 implements 必须 写在类名称的同一行。

类的开始花括号 必须 独占一行,结束花括号也 必须 在类主体后独占一行。

\begin{lstlisting}[language=PHP]
<?php
namespace Vendor\Package;

use FooClass;
use BarClass as Bar;
use OtherVendor\OtherPackage\BazClass;

class ClassName extends ParentClass implements \ArrayAccess, \Countable
{
    // 这里面是常量、属性、类方法
}
\end{lstlisting}

\subsection{Inherit}


implements 的继承列表也 可以 分成多行,这样的话,每个继承接口名称都 必须 分开独立成行,包括第一个。

\begin{lstlisting}[language=PHP]
<?php
namespace Vendor\Package;

use FooClass;
use BarClass as Bar;
use OtherVendor\OtherPackage\BazClass;

class ClassName extends ParentClass implements
    \ArrayAccess,
    \Countable,
    \Serializable
{
    // 这里面是常量、属性、类方法
}
\end{lstlisting}

\subsection{Property}

\begin{compactitem}
\item 每个属性都 必须 添加访问修饰符。

\item 一定不可 使用关键字 var 声明一个属性。

\item 每条语句 一定不可 定义超过一个属性。

\item 不该 使用下划线作为前缀,来区分属性是 protected 或 private。

\end{compactitem}




\begin{lstlisting}[language=PHP]
<?php
namespace Vendor\Package;

class ClassName
{
    public $foo = null;
}
\end{lstlisting}


\subsection{Method}

\begin{compactitem}
\item 所有方法都 必须 添加访问修饰符。

\item 不该 使用下划线作为前缀,来区分方法是 protected 或 private。

\item 方法名称后 一定不可 有空格符,其开始花括号 必须 独占一行,结束花括号也 必须 在方法主体后单独成一行。参数左括号后和右括号前 一定不可 有空格。
\end{compactitem}


一个标准的方法声明可参照以下范例,留意其括号、逗号、空格以及花括号的位置。

\begin{lstlisting}[language=PHP]
<?php
namespace Vendor\Package;

class ClassName
{
    public function fooBarBaz($arg1, &$arg2, $arg3 = [])
    {
        // method body
    }
}
\end{lstlisting}

\subsection{Parameter}

参数列表中,每个逗号后面 必须 要有一个空格,而逗号前面 一定不可 有空格。

有默认值的参数,必须 放到参数列表的末尾。


\begin{lstlisting}[language=PHP]
<?php
namespace Vendor\Package;

class ClassName
{
    public function foo($arg1, &$arg2, $arg3 = [])
    {
        // method body
    }
}
\end{lstlisting}


参数列表 可以 分列成多行,这样,包括第一个参数在内的每个参数都 必须 单独成行。

拆分成多行的参数列表后,结束括号以及方法开始花括号 必须 写在同一行,中间用一个空格分隔。

\begin{lstlisting}[language=PHP]
<?php
namespace Vendor\Package;

class ClassName
{
    public function aVeryLongMethodName(
        ClassTypeHint $arg1,
        &$arg2,
        array $arg3 = []
    ) {
        // 方法的内容
    }
}
\end{lstlisting}


\subsection{Declare}

需要添加 abstract 或 final 声明时,必须 写在访问修饰符前,而 static 则 必须 写在其后。

\begin{lstlisting}[language=PHP]
<?php
namespace Vendor\Package;

abstract class ClassName
{
    protected static $foo;

    abstract protected function zim();

    final public static function bar()
    {
        // method body
    }
}
\end{lstlisting}


\subsection{Method}

方法及函数调用时,方法名或函数名与参数左括号之间 一定不可 有空格,参数右括号前也 一定不可 有空格。每个参数前 一定不可 有空格,但其后 必须 有一个空格。




\begin{lstlisting}[language=PHP]
<?php
bar();
$foo->bar($arg1);
Foo::bar($arg2, $arg3);
\end{lstlisting}

参数 可以 分列成多行,此时包括第一个参数在内的每个参数都 必须 单独成行。

\begin{lstlisting}[language=PHP]
<?php
$foo->bar(
    $longArgument,
    $longerArgument,
    $muchLongerArgument
);
\end{lstlisting}

\section{Control Structure}


控制结构的基本规范如下:

\begin{compactitem}
\item 控制结构关键词后 必须 有一个空格。
\item 左括号 ( 后 一定不可 有空格。
\item 右括号 ) 前也 一定不可 有空格。
\item 右括号 ) 与开始花括号 \{ 间 必须 有一个空格。
\item 结构体主体 必须 要有一次缩进。
\item 结束花括号 \} 必须 在结构体主体后单独成行。
\end{compactitem}

每个结构体的主体都 必须 被包含在成对的花括号之中, 这能让结构体更加结构话,以及减少加入新行时,出错的可能性。


\subsection{if/elseif/else}

标准的 if 结构如下代码所示,请留意「括号」、「空格」以及「花括号」的位置, 注意 else 和 elseif 都与前面的结束花括号在同一行。


\begin{lstlisting}[language=PHP]
<?php
if ($expr1) {
    // if body
} elseif ($expr2) {
    // elseif body
} else {
    // else body;
}
\end{lstlisting}

应该 使用关键词 elseif 代替所有 else if ,以使得所有的控制关键字都像是单独的一个词。

\subsection{switch/case}

标准的 switch 结构如下代码所示,留意括号、空格以及花括号的位置。 case 语句 必须 相对 switch 进行一次缩进,而 break 语句以及 case 内的其它语句都 必须 相对 case 进行一次缩进。

如果存在非空的 case 直穿语句,主体里 必须 有类似 // no break 的注释。




\begin{lstlisting}[language=PHP]
<?php
switch ($expr) {
    case 0:
        echo 'First case, with a break';
        break;
    case 1:
        echo 'Second case, which falls through';
        // no break
    case 2:
    case 3:
    case 4:
        echo 'Third case, return instead of break';
        return;
    default:
        echo 'Default case';
        break;
}
\end{lstlisting}


\subsection{while/do~while}

一个规范的 while 语句应该如下所示,注意其「括号」、「空格」以及「花括号」的位置。

\begin{lstlisting}[language=PHP]
<?php
while ($expr) {
    // structure body
}
\end{lstlisting}

标准的 do while 语句如下所示,同样的,注意其「括号」、「空格」以及「花括号」的位置。



\begin{lstlisting}[language=PHP]
<?php
do {
    // structure body;
} while ($expr);
\end{lstlisting}


\subsection{for}

标准的 for 语句如下所示,注意其「括号」、「空格」以及「花括号」的位置。



\begin{lstlisting}[language=PHP]
<?php
for ($i = 0; $i < 10; $i++) {
    // for body
}
\end{lstlisting}


\subsection{foreach}

标准的 foreach 语句如下所示,注意其「括号」、「空格」以及「花括号」的位置。



\begin{lstlisting}[language=PHP]
<?php
foreach ($iterable as $key => $value) {
    // foreach body
}
\end{lstlisting}

\subsection{try/catch}

标准的 try catch 语句如下所示,注意其「括号」、「空格」以及「花括号」的位置。

\begin{lstlisting}[language=PHP]
<?php
try {
    // try body
} catch (FirstExceptionType $e) {
    // catch body
} catch (OtherExceptionType $e) {
    // catch body
}
\end{lstlisting}

\section{Closure}

\begin{compactitem}
\item 闭包声明时,关键词 function 后以及关键词 use 的前后都 必须 要有一个空格。

\item 开始花括号 必须 写在声明的同一行,结束花括号 必须 紧跟主体结束的下一行。

\item 参数列表和变量列表的左括号后以及右括号前,一定不可 有空格。

\item 参数和变量列表中,逗号前 一定不可 有空格,而逗号后 必须 要有空格。

\item 闭包中有默认值的参数 必须 放到列表的后面。
\end{compactitem}

标准的闭包声明语句如下所示,注意其「括号」、「空格」以及「花括号」的位置。





\begin{lstlisting}[language=PHP]
<?php
$closureWithArgs = function ($arg1, $arg2) {
    // body
};

$closureWithArgsAndVars = function ($arg1, $arg2) use ($var1, $var2) {
    // body
};
\end{lstlisting}

参数列表以及变量列表 可以 分成多行,这样,包括第一个在内的每个参数或变量都 必须 单独成行,而列表的右括号与闭包的开始花括号 必须 放在同一行。

以下几个例子,包含了参数和变量列表被分成多行的多情况。

\begin{lstlisting}[language=PHP]
<?php
$longArgs_noVars = function (
    $longArgument,
    $longerArgument,
    $muchLongerArgument
) {
   // body
};

$noArgs_longVars = function () use (
    $longVar1,
    $longerVar2,
    $muchLongerVar3
) {
   // body
};

$longArgs_longVars = function (
    $longArgument,
    $longerArgument,
    $muchLongerArgument
) use (
    $longVar1,
    $longerVar2,
    $muchLongerVar3
) {
   // body
};

$longArgs_shortVars = function (
    $longArgument,
    $longerArgument,
    $muchLongerArgument
) use ($var1) {
   // body
};

$shortArgs_longVars = function ($arg) use (
    $longVar1,
    $longerVar2,
    $muchLongerVar3
) {
   // body
};
\end{lstlisting}

注意,闭包被直接用作函数或方法调用的参数时,以上规则仍然适用。

\begin{lstlisting}[language=PHP]
<?php
$foo->bar(
    $arg1,
    function ($arg2) use ($var1) {
        // body
    },
    $arg3
);
\end{lstlisting}

\chapter{PSR-3}

本规范的主要目的,是为了让日志类库以简单通用的方式,通过接收一个 Psr\textbackslash Log\textbackslash LoggerInterface 对象来记录日志信息。 

框架以及CMS内容管理系统如有需要,可以 对此接口进行扩展,但需遵循本规范, 这才能保证在使用第三方的类库文件时,日志接口仍能正常对接。


\section{Overview}

LoggerInterface 接口对外定义了八个方法,分别用来记录 RFC 5424 中定义的八个等级的日志:debug、 info、 notice、 warning、 error、 critical、 alert 以及 emergency 。

第九个方法——log,其第一个参数为记录的等级。可使用一个预先定义的等级常量作为参数来调用此方法,必须 与直接调用以上八个方法具有相同的效果。如果传入的等级常量参数没有预先定义,则 必须 抛出 Psr\textbackslash Log\textbackslash InvalidArgumentException 类型的异常。在不确定的情况下,使用者 不该 使用未支持的等级常量来调用此方法。

\subsection{Logging}


以上每个方法都接受一个字符串类型或者是有 \_\_toString() 方法的对象作为记录信息参数,这样,实现者就能把它当成字符串来处理,否则实现者 必须 自己把它转换成字符串。

记录信息参数 可以 携带占位符,实现者 可以 根据上下文将其它替换成相应的值。其中,占位符 必须 与上下文数组中的键名保持一致。

占位符的名称 必须 由一个左花括号 \{ 以及一个右括号 \} 包含。但花括号与名称之间 一定不可有空格符。

占位符的名称 应该 只由 A-Z、a-z、0-9、下划线 \_、以及英文的句号 . 组成,其它字符作为将来占位符规范的保留。

实现者 可以 通过对占位符采用不同的转义和转换策略,来生成最终的日志。 而使用者在不知道上下文的前提下,不该 提前转义占位符。

以下是一个占位符使用的例子:


\begin{lstlisting}[language=PHP]
/**
* 用上下文信息替换记录信息中的占位符
*/
function interpolate($message, array $context = array())
{
  // 构建一个花括号包含的键名的替换数组
  $replace = array();
  foreach ($context as $key => $val) {
      $replace['{' . $key . '}'] = $val;
  }

  // 替换记录信息中的占位符,最后返回修改后的记录信息。
  return strtr($message, $replace);
}

// 含有带花括号占位符的记录信息。
$message = "User {username} created";

// 带有替换信息的上下文数组,键名为占位符名称,键值为替换值。
$context = array('username' => 'bolivar');

// 输出 "Username bolivar created"
echo interpolate($message, $context);
\end{lstlisting}

\subsection{Context}

\begin{compactitem}
\item 每个记录函数都接受一个上下文数组参数,用来装载字符串类型无法表示的信息。它 可以 装载任何信息,所以实现者 必须 确保能正确处理其装载的信息,对于其装载的数据, 一定不可 抛出异常,或产生PHP出错、警告或提醒信息(error、warning、notice)。

\item 如需通过上下文参数传入了一个 Exception 对象,必须 以 exception 作为键名。 记录异常信息是很普遍的,所以如果它能够在记录类库的底层实现,就能够让实现者从异常信息中抽丝剥茧。 当然,实现者在使用它时,必须 确保键名为 exception 的键值是否真的是一个 Exception,毕竟它 可以 装载任何信息。
\end{compactitem}

\subsection{Helper}

\begin{compactitem}
\item Psr\textbackslash Log\textbackslash AbstractLogger 类使得只需继承它和实现其中的 log 方法,就能够很轻易地实现 LoggerInterface 接口,而另外八个方法就能够把记录信息和上下文信息传给它。

\item 同样地,使用 Psr\textbackslash Log\textbackslash LoggerTrait 也只需实现其中的 log 方法。不过,需要特别注意的是,在 traits 可复用代码块还不能实现接口前,还需要 implement LoggerInterface。

\item 在没有可用的日志记录器时,Psr\textbackslash Log\textbackslash NullLogger 接口 可以 为使用者提供一个备用的日志「黑洞」。不过,当上下文的构建非常消耗资源时,带条件检查的日志记录或许是更好的办法。

\item Psr\textbackslash Log\textbackslash LoggerAwareInterface 接口仅包括一个 setLogger(LoggerInterface \$logger) 方法,框架可以使用它实现自动连接任意的日志记录实例。

\item Psr\textbackslash Log\textbackslash LoggerAwareTrait trait可复用代码块可以在任何的类里面使用,只需通过它提供的 \$this->logger,就可以轻松地实现等同的接口。

\item Psr\textbackslash Log\textbackslash LogLevel 类装载了八个记录等级常量。

\end{compactitem}

\section{Package}

上述的接口、类和相关的异常类,以及一系列的实现检测文件,都包含在 psr/log 文件包中。

\section{Interface}

\subsection{Psr\textbackslash Log\textbackslash LoggerInterface}


\begin{lstlisting}[language=PHP]
<?php

namespace Psr\Log;

/**
 * 日志记录实例
 *
 * 日志信息变量 —— message,**必须** 是一个字符串或是实现了 __toString() 方法的对象。
 *
 * 日志信息变量中 **可以** 包含格式如 “{foo}” (代表 foo) 的占位符,
 * 它将会由上下文数组中键名为「foo」的键值替代。
 *
 * 上下文数组可以携带任意的数据,唯一的限制是,当它携带的是一个 exception 对象时,它的键名 **必须** 是 "exception"。
 *
 * 详情可参阅: https://github.com/PizzaLiu/PHP-FIG/blob/master/PSR-3-logger-interface-cn.md
 */
interface LoggerInterface
{
    /**
     * 系统不可用
     *
     * @param string $message
     * @param array $context
     * @return null
     */
    public function emergency($message, array $context = array());

    /**
     *  **必须** 立刻采取行动
     *
     * 例如:在整个网站都垮掉了、数据库不可用了或者其他的情况下, **应该** 发送一条警报短信把你叫醒。
     *
     * @param string $message
     * @param array $context
     * @return null
     */
    public function alert($message, array $context = array());

    /**
     * 紧急情况
     *
     * 例如:程序组件不可用或者出现非预期的异常。
     *
     * @param string $message
     * @param array $context
     * @return null
     */
    public function critical($message, array $context = array());

    /**
     * 运行时出现的错误,不需要立刻采取行动,但必须记录下来以备检测。
     *
     * @param string $message
     * @param array $context
     * @return null
     */
    public function error($message, array $context = array());

    /**
     * 出现非错误性的异常。
     *
     * 例如:使用了被弃用的API、错误地使用了API或者非预想的不必要错误。
     *
     * @param string $message
     * @param array $context
     * @return null
     */
    public function warning($message, array $context = array());

    /**
     * 一般性重要的事件。
     *
     * @param string $message
     * @param array $context
     * @return null
     */
    public function notice($message, array $context = array());

    /**
     * 重要事件
     *
     * 例如:用户登录和SQL记录。
     *
     * @param string $message
     * @param array $context
     * @return null
     */
    public function info($message, array $context = array());

    /**
     * debug 详情
     *
     * @param string $message
     * @param array $context
     * @return null
     */
    public function debug($message, array $context = array());

    /**
     * 任意等级的日志记录
     *
     * @param mixed $level
     * @param string $message
     * @param array $context
     * @return null
     */
    public function log($level, $message, array $context = array());
}
\end{lstlisting}

\subsection{Psr\textbackslash Log\textbackslash LoggerAwareInterface}



\begin{lstlisting}[language=PHP]
<?php

namespace Psr\Log;

/**
 * logger-aware 定义实例
 */
interface LoggerAwareInterface
{
    /**
     * 设置一个日志记录实例
     *
     * @param LoggerInterface $logger
     * @return null
     */
    public function setLogger(LoggerInterface $logger);
}
\end{lstlisting}


\subsection{Psr\textbackslash Log\textbackslash LogLevel}


\begin{lstlisting}[language=PHP]
<?php

namespace Psr\Log;

/**
 * 日志等级常量定义
 */
class LogLevel
{
    const EMERGENCY = 'emergency';
    const ALERT     = 'alert';
    const CRITICAL  = 'critical';
    const ERROR     = 'error';
    const WARNING   = 'warning';
    const NOTICE    = 'notice';
    const INFO      = 'info';
    const DEBUG     = 'debug';
}
\end{lstlisting}

\chapter{PSR-4}


\section{Overview}


PSR-4 是关于由文件路径 自动载入 对应类的相关规范, 本规范是可互操作的,可以作为任一自动载入规范的补充,其中包括 PSR-0,此外, 本 PSR 还包括自动载入的类对应的文件存放路径规范。

PSR-4 提供了一种命名空间的推荐使用方式,它提供了一个标准的文件、类和命名空间的使用惯例,可以让代码随插即用。

此处的「类」泛指所有的「Class类」、「接口」、「traits 可复用代码块」以及其它类似结构。

一个完整的类名需具有以下结构:

\begin{lstlisting}[language=PHP]
\<命名空间>(\<子命名空间>)*\<类名>
\end{lstlisting}

\begin{compactitem}
\item 完整的类名 必须 要有一个顶级命名空间,被称为 "vendor namespace";

\item 完整的类名 可以 有一个或多个子命名空间;

\item 完整的类名 必须 有一个最终的类名;

\item 完整的类名中任意一部分中的下滑线都是没有特殊含义的;

\item 完整的类名 可以 由任意大小写字母组成;

\item 所有类名都 必须 是大小写敏感的。

\end{compactitem}

当根据完整的类名载入相应的文件时:

\begin{compactitem}
\item 完整的类名中,去掉最前面的命名空间分隔符,前面连续的一个或多个命名空间和子命名空间,作为「命名空间前缀」,其必须与至少一个「文件基目录」相对应;

\item 紧接命名空间前缀后的子命名空间 必须 与相应的「文件基目录」相匹配,其中的命名空间分隔符将作为目录分隔符。

\item 末尾的类名 必须 与对应的以 .php 为后缀的文件同名。

\item 自动加载器(autoloader)的实现 一定不可 抛出异常、一定不可 触发任一级别的错误信息以及 不应该 有返回值。

\end{compactitem}

下表展示了符合规范完整类名、命名空间前缀和文件基目录所对应的文件路径。


\begin{longtable}{|m{60pt}|m{40pt}|m{40pt}|m{100pt}|}
%head
\multicolumn{4}{r}{}
\tabularnewline\hline
完整类名&命名空间前缀&文件基目录&文件路径
\endhead
%endhead

%firsthead
\caption{PSR-4示例}\\
\hline
完整类名&命名空间前缀&文件基目录&文件路径
\endfirsthead
%endfirsthead

%foot
\multicolumn{4}{r}{}
\endfoot
%endfoot

%lastfoot
\endlastfoot
%endlastfoot
\hline
\textbackslash Acme\textbackslash Log\textbackslash Writer\textbackslash File\_Writer&Acme\textbackslash Log\textbackslash Writer&./acme-log-writer/lib/&./acme-log-writer/lib/File\_Writer.php\\
\hline
\textbackslash Aura\textbackslash Web\textbackslash Response\textbackslash Status&Aura\textbackslash Web&/path/to/aura-web/src/&/path/to/aura-web/src/Response/Status.php\\
\hline
\textbackslash Symfony\textbackslash Core\textbackslash Request&Symfony\textbackslash Core&./vendor/Symfony/Core/&./vendor/Symfony/Core/Request.php\\
\hline
\textbackslash Zend\textbackslash Acl	&Zend&/usr/includes/Zend/&/usr/includes/Zend/Acl.php\\
\hline
\end{longtable}

\section{Closure}




\begin{lstlisting}[language=PHP]
<?php
/**
 * An example of a project-specific implementation.
 *
 * After registering this autoload function with SPL, the following line
 * would cause the function to attempt to load the \Foo\Bar\Baz\Qux class
 * from /path/to/project/src/Baz/Qux.php:
 *
 *      new \Foo\Bar\Baz\Qux;
 *
 * @param string $class The fully-qualified class name.
 * @return void
 */
spl_autoload_register(function ($class) {

    // project-specific namespace prefix
    $prefix = 'Foo\\Bar\\';

    // base directory for the namespace prefix
    $base_dir = __DIR__ . '/src/';

    // does the class use the namespace prefix?
    $len = strlen($prefix);
    if (strncmp($prefix, $class, $len) !== 0) {
        // no, move to the next registered autoloader
        return;
    }

    // get the relative class name
    $relative_class = substr($class, $len);

    // replace the namespace prefix with the base directory, replace namespace
    // separators with directory separators in the relative class name, append
    // with .php
    $file = $base_dir . str_replace('\\', '/', $relative_class) . '.php';

    // if the file exists, require it
    if (file_exists($file)) {
        require $file;
    }
});
\end{lstlisting}

\section{Class}




\begin{lstlisting}[language=PHP]
<?php
namespace Example;

/**
 * An example of a general-purpose implementation that includes the optional
 * functionality of allowing multiple base directories for a single namespace
 * prefix.
 *
 * Given a foo-bar package of classes in the file system at the following
 * paths ...
 *
 *     /path/to/packages/foo-bar/
 *         src/
 *             Baz.php             # Foo\Bar\Baz
 *             Qux/
 *                 Quux.php        # Foo\Bar\Qux\Quux
 *         tests/
 *             BazTest.php         # Foo\Bar\BazTest
 *             Qux/
 *                 QuuxTest.php    # Foo\Bar\Qux\QuuxTest
 *
 * ... add the path to the class files for the \Foo\Bar\ namespace prefix
 * as follows:
 *
 *      <?php
 *      // instantiate the loader
 *      $loader = new \Example\Psr4AutoloaderClass;
 *
 *      // register the autoloader
 *      $loader->register();
 *
 *      // register the base directories for the namespace prefix
 *      $loader->addNamespace('Foo\Bar', '/path/to/packages/foo-bar/src');
 *      $loader->addNamespace('Foo\Bar', '/path/to/packages/foo-bar/tests');
 *
 * The following line would cause the autoloader to attempt to load the
 * \Foo\Bar\Qux\Quux class from /path/to/packages/foo-bar/src/Qux/Quux.php:
 *
 *      <?php
 *      new \Foo\Bar\Qux\Quux;
 *
 * The following line would cause the autoloader to attempt to load the
 * \Foo\Bar\Qux\QuuxTest class from /path/to/packages/foo-bar/tests/Qux/QuuxTest.php:
 *
 *      <?php
 *      new \Foo\Bar\Qux\QuuxTest;
 */
class Psr4AutoloaderClass
{
    /**
     * An associative array where the key is a namespace prefix and the value
     * is an array of base directories for classes in that namespace.
     *
     * @var array
     */
    protected $prefixes = array();

    /**
     * Register loader with SPL autoloader stack.
     *
     * @return void
     */
    public function register()
    {
        spl_autoload_register(array($this, 'loadClass'));
    }

    /**
     * Adds a base directory for a namespace prefix.
     *
     * @param string $prefix The namespace prefix.
     * @param string $base_dir A base directory for class files in the
     * namespace.
     * @param bool $prepend If true, prepend the base directory to the stack
     * instead of appending it; this causes it to be searched first rather
     * than last.
     * @return void
     */
    public function addNamespace($prefix, $base_dir, $prepend = false)
    {
        // normalize namespace prefix
        $prefix = trim($prefix, '\\') . '\\';

        // normalize the base directory with a trailing separator
        $base_dir = rtrim($base_dir, DIRECTORY_SEPARATOR) . '/';

        // initialize the namespace prefix array
        if (isset($this->prefixes[$prefix]) === false) {
            $this->prefixes[$prefix] = array();
        }

        // retain the base directory for the namespace prefix
        if ($prepend) {
            array_unshift($this->prefixes[$prefix], $base_dir);
        } else {
            array_push($this->prefixes[$prefix], $base_dir);
        }
    }

    /**
     * Loads the class file for a given class name.
     *
     * @param string $class The fully-qualified class name.
     * @return mixed The mapped file name on success, or boolean false on
     * failure.
     */
    public function loadClass($class)
    {
        // the current namespace prefix
        $prefix = $class;

        // work backwards through the namespace names of the fully-qualified
        // class name to find a mapped file name
        while (false !== $pos = strrpos($prefix, '\\')) {

            // retain the trailing namespace separator in the prefix
            $prefix = substr($class, 0, $pos + 1);

            // the rest is the relative class name
            $relative_class = substr($class, $pos + 1);

            // try to load a mapped file for the prefix and relative class
            $mapped_file = $this->loadMappedFile($prefix, $relative_class);
            if ($mapped_file) {
                return $mapped_file;
            }

            // remove the trailing namespace separator for the next iteration
            // of strrpos()
            $prefix = rtrim($prefix, '\\');
        }

        // never found a mapped file
        return false;
    }

    /**
     * Load the mapped file for a namespace prefix and relative class.
     *
     * @param string $prefix The namespace prefix.
     * @param string $relative_class The relative class name.
     * @return mixed Boolean false if no mapped file can be loaded, or the
     * name of the mapped file that was loaded.
     */
    protected function loadMappedFile($prefix, $relative_class)
    {
        // are there any base directories for this namespace prefix?
        if (isset($this->prefixes[$prefix]) === false) {
            return false;
        }

        // look through base directories for this namespace prefix
        foreach ($this->prefixes[$prefix] as $base_dir) {

            // replace the namespace prefix with the base directory,
            // replace namespace separators with directory separators
            // in the relative class name, append with .php
            $file = $base_dir
                  . str_replace('\\', '/', $relative_class)
                  . '.php';

            // if the mapped file exists, require it
            if ($this->requireFile($file)) {
                // yes, we're done
                return $file;
            }
        }

        // never found it
        return false;
    }

    /**
     * If a file exists, require it from the file system.
     *
     * @param string $file The file to require.
     * @return bool True if the file exists, false if not.
     */
    protected function requireFile($file)
    {
        if (file_exists($file)) {
            require $file;
            return true;
        }
        return false;
    }
}
\end{lstlisting}

\section{Unit Test}




\begin{lstlisting}[language=PHP]
<?php
namespace Example\Tests;

class MockPsr4AutoloaderClass extends Psr4AutoloaderClass
{
    protected $files = array();

    public function setFiles(array $files)
    {
        $this->files = $files;
    }

    protected function requireFile($file)
    {
        return in_array($file, $this->files);
    }
}

class Psr4AutoloaderClassTest extends \PHPUnit_Framework_TestCase
{
    protected $loader;

    protected function setUp()
    {
        $this->loader = new MockPsr4AutoloaderClass;

        $this->loader->setFiles(array(
            '/vendor/foo.bar/src/ClassName.php',
            '/vendor/foo.bar/src/DoomClassName.php',
            '/vendor/foo.bar/tests/ClassNameTest.php',
            '/vendor/foo.bardoom/src/ClassName.php',
            '/vendor/foo.bar.baz.dib/src/ClassName.php',
            '/vendor/foo.bar.baz.dib.zim.gir/src/ClassName.php',
        ));

        $this->loader->addNamespace(
            'Foo\Bar',
            '/vendor/foo.bar/src'
        );

        $this->loader->addNamespace(
            'Foo\Bar',
            '/vendor/foo.bar/tests'
        );

        $this->loader->addNamespace(
            'Foo\BarDoom',
            '/vendor/foo.bardoom/src'
        );

        $this->loader->addNamespace(
            'Foo\Bar\Baz\Dib',
            '/vendor/foo.bar.baz.dib/src'
        );

        $this->loader->addNamespace(
            'Foo\Bar\Baz\Dib\Zim\Gir',
            '/vendor/foo.bar.baz.dib.zim.gir/src'
        );
    }

    public function testExistingFile()
    {
        $actual = $this->loader->loadClass('Foo\Bar\ClassName');
        $expect = '/vendor/foo.bar/src/ClassName.php';
        $this->assertSame($expect, $actual);

        $actual = $this->loader->loadClass('Foo\Bar\ClassNameTest');
        $expect = '/vendor/foo.bar/tests/ClassNameTest.php';
        $this->assertSame($expect, $actual);
    }

    public function testMissingFile()
    {
        $actual = $this->loader->loadClass('No_Vendor\No_Package\NoClass');
        $this->assertFalse($actual);
    }

    public function testDeepFile()
    {
        $actual = $this->loader->loadClass('Foo\Bar\Baz\Dib\Zim\Gir\ClassName');
        $expect = '/vendor/foo.bar.baz.dib.zim.gir/src/ClassName.php';
        $this->assertSame($expect, $actual);
    }

    public function testConfusion()
    {
        $actual = $this->loader->loadClass('Foo\Bar\DoomClassName');
        $expect = '/vendor/foo.bar/src/DoomClassName.php';
        $this->assertSame($expect, $actual);

        $actual = $this->loader->loadClass('Foo\BarDoom\ClassName');
        $expect = '/vendor/foo.bardoom/src/ClassName.php';
        $this->assertSame($expect, $actual);
    }
}
\end{lstlisting}

\chapter{PSR-6}

PSR-6的目标是创建一套通用的接口规范,能够让开发人员整合到现有框架和系统,而不需要去 开发框架专属的适配器类。


\section{Overview}


缓存是提升应用性能的常用手段,为框架中最通用的功能,每个框架也都推出专属的、功能多 样的缓存库。这些差别使得开发人员不得不学习多种系统,而很多可能是他们并不需要的功能。 此外,缓存库的开发者同样面临着一个窘境,是只支持有限数量的几个框架还是创建一堆庞 大的适配器类。

一个通用的缓存系统接口可以解决掉这些问题。库和框架的开发人员能够知道缓存系统会按照他们所 预期的方式工作,缓存系统的开发人员只需要实现单一的接口,而不用去开发各种各样的适配器。


\section{Definition}


\subsection{Calling Library}

调用类库 (Calling Library) - 调用者,使用缓存服务的类库,这个类库调用缓存服务,调用的 是此缓存接口规范的具体「实现类库」,调用者不需要知道任何「缓存服务」的具体实现。




\subsection{Implementing Library}

实现类库 (Implementing Library) - 此类库是对「缓存接口规范」的具体实现,封装起来的缓存服务,供「调用类库」使用。

实现类库 必须 提供 PHP 类来实现 Cache\textbackslash CacheItemPoolInterface 和 Cache\textbackslash CacheItemInterface 接口。 实现类库 必须 支持最小的TTL 功能,秒级别的精准度。

\subsection{Expiration}

过期时间 (Expiration) - 定义准确的过期时间点,一般为缓存存储发生的时间点加上 TTL 时 间值,也可以指定一个 DateTime 对象。

假如一个缓存项的 TTL 设置为 300 秒,保存于 1:30:00 ,那么缓存项的过期时间为 1:35:00。

实现类库 可以 让缓存项提前过期,但是 必须 在到达过期时间时立即把缓存项标示为 过期。如果调用类库在保存一个缓存项的时候未设置「过期时间」、或者设置了 null 作为过期 时间(或者 TTL 设置为 null),实现类库 可以 使用默认自行配置的一个时间。如果没 有默认时间,实现类库 必须把存储时间当做 永久性 存储,或者按照底层驱动能支持的 最长时间作为保持时间。

\subsection{TTL}

生存时间值 (TTL - Time To Live) - 定义了缓存可以存活的时间,以秒为单位的整数值。



\subsection{Key}

长度大于 1 的字串,用作缓存项在缓存系统里的唯一标识符。

\begin{compactitem}
\item 实现类库 必须 支持「键」规则 A-Z,a-z,0-9,\_和 . 任何顺序的 UTF-8 编码,长度 小于 64 位。
\item 实现类库 可以 支持更多的编码或者更长的长度,不过 必须 支持至少以上指定 的编码和长度。
\item 实现类库可自行实现对「键」的转义,但是 必须 保证能够无损的返回「键」字串。
\end{compactitem}


以下 的字串作为系统保留: \{\}()/\textbackslash @:,一定不可 作为「键」的命名支持。


\subsection{Hit}

命中 (Hit) - 一个缓存的命中,指的是当调用类库使用「键」在请求一个缓存项的时候,在缓存 池里能找到对应的缓存项,并且此缓存项还未过期,并且此数据不会因为任何原因出现错误。

调用类 库 应该 确保先验证下 isHit() 有命中后才调用 get() 获取数据。

\subsection{Miss}

未命中 (Miss) - 一个缓存未命中,是完全的上面描述的「命中」的相反。指的是当调用类库使用「键」在请求一个缓存项的时候,在缓存池里未能找到对应的缓存项,或者此缓存项已经过期,或者此数据因为任何原因出现错误。

一个过期的缓存项,必须 被当做 未命中 来对待。


\subsection{Deferred}


延迟 (Deferred) - 一个延迟的缓存,指的是这个缓存项可能不会立刻被存储到物理缓存池里。

一个 缓存池对象 可以 对一个指定延迟的缓存项进行延迟存储,这样做的好处是可以利用一些缓存服务器提供 的批量插入功能。缓存池 必须 能对所有延迟缓存最终能持久化,并且不会丢失。可以 在调用类库还未发起保存请求之前就做持久化。

当调用类库调用 commit() 方法时,所有的延迟缓存都 必须 做持久化。实现类库 可以 自行决定使用什么逻辑来触发数据持久化,如对象的 析构方法 (destructor) 内、调用 save() 时持久化、倒计时保存或者触及最大数量时保存等。

当请求一个延迟 缓存项时,必须 返回一个延迟,未持久化的缓存项对象。

\section{Data Type}

实现类库 必须 支持所有的可序列化的 PHP 数据类型,包含:

\begin{compactitem}
\item 字符串 - 任何大小的 PHP 兼容字符串
\item 整数 - PHP 支持的低于 64 位的有符号整数值
\item 浮点数 - 所有的有符号浮点数
\item 布尔 - true 和 false.
\item Null - null 值
\item 数组 - 各种形式的 PHP 数组
\item 对象(Object) - 所有的支持无损序列化和反序列化的对象,例如\texttt{\$o == unserialize(serialize(\$o))} 。

对象 可以 使用 PHP 的 Serializable 接口,\_\_sleep() 或者 \_\_wakeup() 魔术方法,或者在合适的情况下,使用其他类似的语言特性。
\end{compactitem}

所有存进实现类库的数据,都 必须 能做到原封不动的取出。连类型也 必须 是完全一致,如果 存进缓存的是字符串 5,取出来的却是整数值 5 的话,可以算作严重的错误。实现类库 可以 使用 PHP 的「serialize()/unserialize() 方法」作为底层实现,不过不强迫这样做。对于它们的兼容性,以能支持所有数据类型作为基准线。

实在无法「完整取出」存入的数据的话,实现类库 必须 把「缓存丢失」标示作为返回,而不是损坏了的数据。

\section{Concepts}

\subsection{Pool}

缓存池包含缓存系统里所有缓存数据的集合。缓存池逻辑上是所有缓存项存储的仓库,所有存储进去的数据, 都能从缓存池里取出来,所有的对缓存的操作,都发生在缓存池子里。

\subsection{Items}

一条缓存项在缓存池里代表了一对「键/值」对应的数据,「键」被视为每一个缓存项主键,是缓存项的 唯一标识符,必须 是不可变更的,当然,「值」可以 任意变更。

\section{Error Handling}

缓存对应用性能起着至关重要的作用,但是,无论在任何情况下,缓存 一定不可 作为应用程序不 可或缺的核心功能。

缓存系统里的错误 一定不可 导致应用程序故障,所以,实现类库 一定不可 抛出任何除了 此接口规范定义的以外的异常,并且 必须 捕捉包括底层存储驱动抛出的异常,不让其冒泡至超 出缓存系统内。

实现类库 应该 对此类错误进行记录,或者以任何形式通知管理员。

调用类库发起删除缓存项的请求,或者清空整个缓冲池子的请求,「键」不存在的话 必须 不能 当成是有错误发生。后置条件是一样的,如果取数据时,「键」不存在的话 必须 不能当成是有错误发生。

\section{Interface}

\subsection{CacheItemInterface}


CacheItemInterface 定义了缓存系统里的一个缓存项。每一个缓存项 必须 有一个「键」与之相 关联,此「键」通常是通过 Cache\textbackslash CacheItemPoolInterface 来设置。

Cache\textbackslash CacheItemInterface 对象把缓存项的存储进行了封装,每一个 Cache\textbackslash CacheItemInterface 由一个 Cache\textbackslash CacheItemPoolInterface 对象生成,CacheItemPoolInterface 负责一些必须的设置,并且给对象设置具有 唯一性 的「键」。

Cache\textbackslash CacheItemInterface 对象 必须 能够存储和取出任何类型的定义有效的 PHP 数值。

调用类库 一定不可 擅自初始化「CacheItemInterface」对象,「缓存项」只能使用「CacheItemPoolInterface」对象的 getItem() 方法来获取。调用类库 一定不可 假设 由一个实现类库创建的「缓存项」能被另一个实现类库完全兼容。

\begin{lstlisting}[language=PHP]
namespace Psr\Cache;

/**
 * CacheItemInterface 定了缓存系统里对缓存项操作的接口
 */
interface CacheItemInterface
{
    /**
     * 返回当前缓存项的「键」
     * 
     * 「键」由实现类库来加载,并且高层的调用者(如:CacheItemPoolInterface)
     *  **应该** 能使用此方法来获取到「键」的信息。
     *
     * @return string
     *   当前缓存项的「键」
     */
    public function getKey();

    /**
     * 凭借此缓存项的「键」从缓存系统里面取出缓存项。
     *
     * 取出的数据 **必须** 跟使用 `set()` 存进去的数据是一模一样的。
     *
     * 如果 `isHit()` 返回 false 的话,此方法必须返回 `null`,需要注意的是 `null` 
     * 本来就是一个合法的缓存数据,所以你 **应该** 使用 `isHit()` 方法来辨别到底是
     * "返回 null 数据" 还是 "缓存里没有此数据"。
     *
     * @return mixed
     *   此缓存项的「键」对应的「值」,如果找不到的话,返回 `null`
     */
    public function get();

    /**
     * 确认缓存项的检查是否命中。
     * 
     * 注意: 调用此方法和调用 `get()` 时 **一定不可** 有先后顺序之分。
     *
     * @return bool
     *   如果缓冲池里有命中的话,返回 `true`,反之返回 `false`
     */
    public function isHit();

    /**
     * 为此缓存项设置「值」。
     *
     * 参数 $value 可以是所有能被 PHP 序列化的数据,序列化的逻辑
     * 需要在实现类库里书写。
     *
     * @param mixed $value
     *   将被存储的可序列化的数据。
     *
     * @return static
     *   返回当前对象。
     */
    public function set($value);

    /**
     * 设置缓存项的准确过期时间点。
     *
     * @param \DateTimeInterface $expiration
     * 
     *   过期的准确时间点,过了这个时间点后,缓存项就 **必须** 被认为是过期了的。
     *   如果明确的传参 `null` 的话,**可以** 使用一个默认的时间。
     *   如果没有设置的话,缓存 **应该** 存储到底层实现的最大允许时间。
     *
     * @return static
     *   返回当前对象。
     */
    public function expiresAt($expiration);

    /**
     * 设置缓存项的过期时间。
     *
     * @param int|\DateInterval $time
     *   以秒为单位的过期时长,过了这段时间后,缓存项就 **必须** 被认为是过期了的。
     *   如果明确的传参 `null` 的话,**可以** 使用一个默认的时间。
     *   如果没有设置的话,缓存 **应该** 存储到底层实现的最大允许时间。
     *
     * @return static
     *   返回当前对象
     */
    public function expiresAfter($time);

}
\end{lstlisting}

\subsection{CacheItemPoolInterface}

Cache\textbackslash CacheItemPoolInterface 的主要目的是从调用类库接收「键」,然后返回对应的 Cache\textbackslash CacheItemInterface 对象。

此接口也是作为主要的,与整个缓存集合交互的方式。所有的配置和初始化由实现类库自行实现。




\begin{lstlisting}[language=PHP]
namespace Psr\Cache;

/**
 * CacheItemPoolInterface 生成 CacheItemInterface 对象
 */
interface CacheItemPoolInterface
{
    /**
     * 返回「键」对应的一个缓存项。
     *
     * 此方法 **必须** 返回一个 CacheItemInterface 对象,即使是找不到对应的缓存项
     * 也 **一定不可** 返回 `null`。
     *
     * @param string $key
     *   用来搜索缓存项的「键」。
     *
     * @throws InvalidArgumentException
     *   如果 $key 不是合法的值,\Psr\Cache\InvalidArgumentException 异常会被抛出。
     *
     * @return CacheItemInterface
     *   对应的缓存项。
     */
    public function getItem($key);

    /**
     * 返回一个可供遍历的缓存项集合。
     *
     * @param array $keys
     *   由一个或者多个「键」组成的数组。
     *
     * @throws InvalidArgumentException
     *   如果 $keys 里面有哪个「键」不是合法,\Psr\Cache\InvalidArgumentException 异常
     *   会被抛出。
     *   
     * @return array|\Traversable
     *   返回一个可供遍历的缓存项集合,集合里每个元素的标识符由「键」组成,即使即使是找不到对
     *   的缓存项,也要返回一个「CacheItemInterface」对象到对应的「键」中。
     *   如果传参的数组为空,也需要返回一个空的可遍历的集合。
     */
    public function getItems(array $keys = array());

    /**
     * 检查缓存系统中是否有「键」对应的缓存项。
     *
     * 注意: 此方法应该调用 `CacheItemInterface::isHit()` 来做检查操作,而不是
     * `CacheItemInterface::get()`
     *
     * @param string $key
     *   用来搜索缓存项的「键」。
     *
     * @throws InvalidArgumentException
     *   如果 $key 不是合法的值,\Psr\Cache\InvalidArgumentException 异常会被抛出。
     *
     * @return bool
     *   如果存在「键」对应的缓存项即返回 true,否则 false
     */
    public function hasItem($key);

    /**
     * 清空缓冲池
     *
     * @return bool
     *   成功返回 true,有错误发生返回 false
     */
    public function clear();

    /**
     * 从缓冲池里移除某个缓存项
     *
     * @param string $key
     *   用来搜索缓存项的「键」。
     *
     * @throws InvalidArgumentException
     *   如果 $key 不是合法的值,\Psr\Cache\InvalidArgumentException 异常会被抛出。
     *
     * @return bool
     *   成功返回 true,有错误发生返回 false
     */
    public function deleteItem($key);

    /**
     * 从缓冲池里移除多个缓存项
     *
     * @param array $keys
     *   由一个或者多个「键」组成的数组。
     *   
     * @throws InvalidArgumentException
     *   如果 $keys 里面有哪个「键」不是合法,\Psr\Cache\InvalidArgumentException 异常
     *   会被抛出。
     *
     * @return bool
     *   成功返回 true,有错误发生返回 false
     */
    public function deleteItems(array $keys);

    /**
     * 立刻为「CacheItemInterface」对象做数据持久化。
     *
     * @param CacheItemInterface $item
     *   将要被存储的缓存项
     *
     * @return bool
     *   成功返回 true,有错误发生返回 false
     */
    public function save(CacheItemInterface $item);

    /**
     * 稍后为「CacheItemInterface」对象做数据持久化。
     *
     * @param CacheItemInterface $item
     *   将要被存储的缓存项
     *
     * @return bool
     *   成功返回 true,有错误发生返回 false
     */
    public function saveDeferred(CacheItemInterface $item);

    /**
     * 提交所有的正在队列里等待的请求到数据持久层,配合 `saveDeferred()` 使用
     *
     * @return bool
     *  成功返回 true,有错误发生返回 false
     */
    public function commit();
}
\end{lstlisting}



\subsection{CacheException}

此异常用于缓存系统发生的所有严重错误,包括但不限制于 缓存系统配置,如连接到缓存服务器出错、错 误的用户身份认证等。

所有的实现类库抛出的异常都 必须 实现此接口。


\begin{lstlisting}[language=PHP]
namespace Psr\Cache;

/**
 * 被所有的实现类库抛出的异常继承的「异常接口」
 */
interface CacheException
{
}
\end{lstlisting}

\subsection{InvalidArgumentException}


\begin{lstlisting}[language=PHP]
namespace Psr\Cache;

/**
 * 传参错误抛出的异常接口
 *
 * 当一个错误或者非法的传参发生时,**必须** 抛出一个继承了
 * Psr\Cache\InvalidArgumentException 的异常
 */
interface InvalidArgumentException extends CacheException
{
}
\end{lstlisting}


\chapter{PSR-7}

PSR-7描述了 RFC 7230 和 RFC 7231 HTTP 消息传递的接口,还有 RFC 3986 里对 HTTP 消息的 URIs 使用。


\section{Overview}

\subsection{HTTP Message}


HTTP 消息是 Web 技术发展的基础。浏览器或 HTTP 客户端如 curl 生成发送 HTTP 请求消息到 Web 服务器,Web 服务器响应 HTTP 请求。服务端的代码接受 HTTP 请求消息后返回 HTTP 响应消息。

通常 HTTP 消息对于终端用户来说是不可见的,但是作为 Web 开发者,我们需要知道 HTTP 机制,如何发起、构建、取用还有操纵 HTTP 消息,知道这些原理,以助我们刚好的完成开发任务,无论这个任务是发起一个 HTTP 请求,或者处理传入的请求。


\subsection{HTTP Request}


每一个 HTTP 请求都有专属的格式:



\begin{lstlisting}[language=PHP]
POST /path HTTP/1.1
Host: example.com

foo=bar&baz=bat
\end{lstlisting}


按照顺序,第一行的各个字段意义为: HTTP 请求方法、请求的目标地址(通常是一个绝对路径的 URI 或 者路径),HTTP 协议。

接下来是 HTTP 头信息,在这个例子中:目的主机。接下来是空行,然后是消息内容。

\subsection{HTTP Response}

HTTP 返回消息有类似的结构:



\begin{lstlisting}[language=PHP]
HTTP/1.1 200 OK
Content-Type: text/plain

这是返回的消息内容
\end{lstlisting}

按照顺序,第一行为状态行,包括 HTTP 协议版本,HTTP 状态码,描述文本。

和 HTTP 请求类似的,接下来是 HTTP 头信息,在这个例子中:内容类型。接下来是空行,然后是消息内容。

PSR-7文档探讨的是 HTTP 请求消息接口,和构建 HTTP 消息需要的元素数据定义。

\section{Specification}



\subsection{Message}

一个 HTTP 消息,指定是一个从客户端发往服务器端的请求,或者是服务器端返回客户端的响应。对应的两 个消息接口:Psr\textbackslash Http\textbackslash Message\textbackslash RequestInterface 和 Psr\textbackslash Http\textbackslash Message\textbackslash ResponseInterface。

这个两个接口都继承于 Psr\textbackslash Http\textbackslash Message\textbackslash MessageInterface。虽然你 可以 实现 Psr\textbackslash Http\textbackslash Message\textbackslash MessageInterface 接口,但是,最重要的,你 必须 实现 Psr\textbackslash Http\textbackslash Message\textbackslash RequestInterface 和 Psr\textbackslash Http\textbackslash Message\textbackslash ResponseInterface 接口。

从现在开始,为了行文的方便,我们提到这些接口的时候,都会把命名空间 Psr\textbackslash Http\textbackslash Message 去除掉。

\subsection{Header}

大小写不敏感的字段名字

HTTP 消息包含大小写不敏感头信息。使用 MessageInterface 接口来设置和获取头信息,大小写 不敏感的定义在于,如果你设置了一个 Foo 的头信息,foo 的值会被重写,你也可以通过 foo 来 拿到 FoO 头对应的值。

\begin{lstlisting}[language=PHP]
$message = $message->withHeader('foo', 'bar');

echo $message->getHeaderLine('foo');
// 输出:bar

echo $message->getHeaderLine('FOO');
// 输出:bar

$message = $message->withHeader('fOO', 'baz');
echo $message->getHeaderLine('foo');
// 输出:baz
\end{lstlisting}

虽然头信息可以用大小写不敏感的方式取出,但是接口实现类 必须 保持自己的大小写规范,特别是用 getHeaders() 方法输出的内容。

因为一些非标准的 HTTP 应用程序,可能会依赖于大小写敏感的头信息,所有在此我们把主宰 HTTP 大小写的权利开放出来,以适用不同的场景。

为了适用一个 HTTP 「键」可以对应多条数据的情况,我们使用字符串配合数组来实现,你可以从一个 MessageInterface 取出数组或字符串,使用 getHeaderLine(\$name) 方法可以获取通过逗号分割的不区分大小写的字符串形式的所有值。也可以通过 getHeader(\$name) 获取数组形式头信息的所有值。



\begin{lstlisting}[language=PHP]
$message = $message
    ->withHeader('foo', 'bar')
    ->withAddedHeader('foo', 'baz');

$header = $message->getHeaderLine('foo');
// $header 包含: 'bar, baz'

$header = $message->getHeader('foo');
// ['bar', 'baz']
\end{lstlisting}

注意,并不是所有的头信息都可以适用逗号分割(例如 Set-Cookie),当处理这种头信息时候, MessageInterace 的继承类 应该 使用 getHeader(\$name) 方法来获取这种多值的情况。

在请求中,Host 头信息通常和 URI 的 host 信息,还有建立起 TCP 连接使用的 Host 信息一致。 然而,HTTP 标准规范允许主机 host 信息与其他两个不一样。

在构建请求的时候,如果 host 头信息未提供的话,实现类库 必须 尝试着从 URI 中提取 host 信息。

RequestInterface::withUri() 会默认的,从传参的 UriInterface 实例中提取 host , 并替代请求中原有的 host 信息。

你可以提供传参第二个参数为 true 来保证返回的消息实例中,原有的 host 头信息不会被替代掉。

以下表格说明了当 withUri() 的第二个参数被设置为 true 的时,返回的消息实例中调用 getHeaderLine('Host') 方法会返回的内容:


\begin{longtable}{|m{60pt}|m{80pt}|m{80pt}|m{40pt}|}
%head
\multicolumn{4}{r}{}
\tabularnewline\hline
请求 Host 头信息&请求 URI 中的 Host 信息&	传参进去 URI 的 Host&	结果
\endhead
%endhead

%firsthead
\caption{Host信息示例}\\
\hline
请求 Host 头信息&请求 URI 中的 Host 信息&	传参进去 URI 的 Host&	结果
\endfirsthead
%endfirsthead

%foot
\multicolumn{4}{r}{}
\endfoot
%endfoot

%lastfoot
\endlastfoot
%endlastfoot
\hline
\texttt{''}	&\texttt{''}&\texttt{''}&\texttt{''}\\
\hline
\texttt{''}	&\texttt{foo.com}	&\texttt{''}&\texttt{foo.com}\\
\hline
\texttt{''}	&\texttt{foo.com}	&\texttt{bar.com}	&\texttt{foo.com}\\
\hline
\texttt{foo.com}&\texttt{''}	&\texttt{bar.com}	&\texttt{foo.com}\\
\hline
\texttt{foo.com}&\texttt{bar.com}&\texttt{baz.com}&\texttt{foo.com}\\
\hline
\end{longtable}

\begin{compactitem}
\item 请求 Host 头信息 - 当前请求的 Host 头信息
\item 请求 URI 中的 Host 信息 - 当前请求 URI 中的 Host 信息
\item 传参进去 URI 的 Host - 通过 withUri() 传参进入的 URI 中的 host 信息
\end{compactitem}

\subsection{Stream}

HTTP 消息包含开始的一行、头信息、还有消息的内容。HTTP 的消息内容有时候可以很小,有时候确是 非常巨大。尝试使用字符串的形式来展示消息内容,会消耗大量的内存,使用数据流的形式来读取消息 可以解决此问题。

StreamInterface 接口用来隐藏具体的数据流读写实现。在一些情况下,消息 类型的读取方式为字符串是能容许的,可以使用 php://memory 或者 php://temp。

StreamInterface 暴露出来几个接口,这些接口允许你读取、写入,还有高效的遍历内容。

数据流使用这个三个接口来阐明对他们的操作能力:isReadable()、isWritable() 和 isSeekable()。这些方法可以让数据流的操作者得知数据流能否能提供他们想要的功能。

每一个数据流的实例,都会有多种功能:可以只读、可以只写、可以读和写,可以随机读取,可以按顺序读取等。

最终,StreamInterface 定义了一个 \_\_toString() 的方法,用来一次性以字符串的形式输出 所有消息内容。

与请求和响应的接口不同的是,StreamInterface 并不强调不可修改性。因为在 PHP 的实现内,基 本上没有办法保证不可修改性,因为指针的指向,内容的变更等状态,都是不可控的。

作为读取者,可以 调用只读的方法来返回数据流,以最大程度上保证数据流的不可修改性。使用者要时刻明确的知道数据 流的可修改性,建议把数据流附加到消息实例中,来强迫不可修改的特性。

\section{Request Target}


Per RFC 7230, request messages contain a "request-target" as the second segment of the request line. The request target can be one of the following forms:

\begin{compactitem}
\item origin-form, which consists of the path, and, if present, the query string; this is often referred to as a relative URL. Messages as transmitted over TCP typically are of origin-form; scheme and authority data are usually only present via CGI variables.
\item absolute-form, which consists of the scheme, authority ("[user-info@]host[:port]", where items in brackets are optional), path (if present), query string (if present), and fragment (if present). This is often referred to as an absolute URI, and is the only form to specify a URI as detailed in RFC 3986. This form is commonly used when making requests to HTTP proxies.
\item authority-form, which consists of the authority only. This is typically used in CONNECT requests only, to establish a connection between an HTTP client and a proxy server.
\item asterisk-form, which consists solely of the string *, and which is used with the OPTIONS method to determine the general capabilities of a web server.
\end{compactitem}

Aside from these request-targets, there is often an 'effective URL' which is separate from the request target. The effective URL is not transmitted within an HTTP message, but it is used to determine the protocol (http/https), port and hostname for making the request.

The effective URL is represented by UriInterface. UriInterface models HTTP and HTTPS URIs as specified in RFC 3986 (the primary use case). The interface provides methods for interacting with the various URI parts, which will obviate the need for repeated parsing of the URI. It also specifies a \_\_toString() method for casting the modeled URI to its string representation.

When retrieving the request-target with getRequestTarget(), by default this method will use the URI object and extract all the necessary components to construct the origin-form. The origin-form is by far the most common request-target.

If it's desired by an end-user to use one of the other three forms, or if the user wants to explicitly override the request-target, it is possible to do so with withRequestTarget().

Calling this method does not affect the URI, as it is returned from getUri().

For example, a user may want to make an asterisk-form request to a server:

\begin{lstlisting}[language=PHP]
$request = $request
    ->withMethod('OPTIONS')
    ->withRequestTarget('*')
    ->withUri(new Uri('https://example.org/'));
\end{lstlisting}

This example may ultimately result in an HTTP request that looks like this:

\begin{lstlisting}[language=PHP]
OPTIONS * HTTP/1.1
\end{lstlisting}


But the HTTP client will be able to use the effective URL (from getUri()), to determine the protocol, hostname and TCP port.

An HTTP client MUST ignore the values of Uri::getPath() and Uri::getQuery(), and instead use the value returned by getRequestTarget(), which defaults to concatenating these two values.

Clients that choose to not implement 1 or more of the 4 request-target forms, MUST still use getRequestTarget(). These clients MUST reject request-targets they do not support, and MUST NOT fall back on the values from getUri().

RequestInterface provides methods for retrieving the request-target or creating a new instance with the provided request-target. By default, if no request-target is specifically composed in the instance, getRequestTarget() will return the origin-form of the composed URI (or "/" if no URI is composed). withRequestTarget(\$requestTarget) creates a new instance with the specified request target, and thus allows developers to create request messages that represent the other three request-target forms (absolute-form, authority-form, and asterisk-form). When used, the composed URI instance can still be of use, particularly in clients, where it may be used to create the connection to the server.

\section{Server-side Requests}


RequestInterface provides the general representation of an HTTP request message. However, server-side requests need additional treatment, due to the nature of the server-side environment. Server-side processing needs to take into account Common Gateway Interface (CGI), and, more specifically, PHP's abstraction and extension of CGI via its Server APIs (SAPI). PHP has provided simplification around input marshaling via superglobals such as:

\begin{compactitem}
\item \$\_COOKIE, which deserializes and provides simplified access for HTTP cookies.
\item \$\_GET, which deserializes and provides simplified access for query string arguments.
\item \$\_POST, which deserializes and provides simplified access for urlencoded parameters submitted via HTTP POST; generically, it can be considered the results of parsing the message body.
\item \$\_FILES, which provides serialized metadata around file uploads.
\item \$\_SERVER, which provides access to CGI/SAPI environment variables, which commonly include the request method, the request scheme, the request URI, and headers.
\end{compactitem}

ServerRequestInterface extends RequestInterface to provide an abstraction around these various superglobals. This practice helps reduce coupling to the superglobals by consumers, and encourages and promotes the ability to test request consumers.

The server request provides one additional property, "attributes", to allow consumers the ability to introspect, decompose, and match the request against application-specific rules (such as path matching, scheme matching, host matching, etc.). As such, the server request can also provide messaging between multiple request consumers.


\section{Upload Files}

ServerRequestInterface specifies a method for retrieving a tree of upload files in a normalized structure, with each leaf an instance of UploadedFileInterface.

The \$\_FILES superglobal has some well-known problems when dealing with arrays of file inputs. As an example, if you have a form that submits an array of files — e.g., the input name "files", submitting files[0] and files[1] — PHP will represent this as:


\begin{lstlisting}[language=PHP]
array(
    'files' => array(
        'name' => array(
            0 => 'file0.txt',
            1 => 'file1.html',
        ),
        'type' => array(
            0 => 'text/plain',
            1 => 'text/html',
        ),
        /* etc. */
    ),
)
\end{lstlisting}

instead of the expected:

\begin{lstlisting}[language=PHP]
array(
    'files' => array(
        0 => array(
            'name' => 'file0.txt',
            'type' => 'text/plain',
            /* etc. */
        ),
        1 => array(
            'name' => 'file1.html',
            'type' => 'text/html',
            /* etc. */
        ),
    ),
)
\end{lstlisting}

The result is that consumers need to know this language implementation detail, and write code for gathering the data for a given upload.

Additionally, scenarios exist where \$\_FILES is not populated when file uploads occur:

\begin{compactitem}
\item When the HTTP method is not POST.
\item When unit testing.
\item When operating under a non-SAPI environment, such as ReactPHP.
\end{compactitem}

In such cases, the data will need to be seeded differently. As examples:

\begin{compactitem}
\item A process might parse the message body to discover the file uploads. In such cases, the implementation may choose not to write the file uploads to the file system, but instead wrap them in a stream in order to reduce memory, I/O, and storage overhead.
\item In unit testing scenarios, developers need to be able to stub and/or mock the file upload metadata in order to validate and verify different scenarios.
\end{compactitem}

getUploadedFiles() provides the normalized structure for consumers. Implementations are expected to:

\begin{compactitem}
\item Aggregate all information for a given file upload, and use it to populate a Psr\textbackslash Http\textbackslash Message\textbackslash UploadedFileInterface instance.
\item Re-create the submitted tree structure, with each leaf being the appropriate Psr\textbackslash Http\textbackslash Message\textbackslash UploadedFileInterface instance for the given location in the tree.
\end{compactitem}

The tree structure referenced should mimic the naming structure in which files were submitted.

In the simplest example, this might be a single named form element submitted as:

\begin{lstlisting}[language=PHP]
<input type="file" name="avatar" />
\end{lstlisting}

In this case, the structure in \$\_FILES would look like:

\begin{lstlisting}[language=PHP]
array(
    'avatar' => array(
        'tmp_name' => 'phpUxcOty',
        'name' => 'my-avatar.png',
        'size' => 90996,
        'type' => 'image/png',
        'error' => 0,
    ),
)
\end{lstlisting}


The normalized form returned by getUploadedFiles() would be:



\begin{lstlisting}[language=PHP]
array(
    'avatar' => /* UploadedFileInterface instance */
)

\end{lstlisting}

In the case of an input using array notation for the name:



\begin{lstlisting}[language=PHP]
<input type="file" name="my-form[details][avatar]" />
\end{lstlisting}

\$\_FILES ends up looking like this:



\begin{lstlisting}[language=PHP]
array(
    'my-form' => array(
        'details' => array(
            'avatar' => array(
                'tmp_name' => 'phpUxcOty',
                'name' => 'my-avatar.png',
                'size' => 90996,
                'type' => 'image/png',
                'error' => 0,
            ),
        ),
    ),
)
\end{lstlisting}

And the corresponding tree returned by getUploadedFiles() should be:



\begin{lstlisting}[language=PHP]
array(
    'my-form' => array(
        'details' => array(
            'avatar' => /* UploadedFileInterface instance */
        ),
    ),
)
\end{lstlisting}

In some cases, you may specify an array of files:



\begin{lstlisting}[language=PHP]
Upload an avatar: <input type="file" name="my-form[details][avatars][]" />
Upload an avatar: <input type="file" name="my-form[details][avatars][]" />
\end{lstlisting}

(As an example, JavaScript controls might spawn additional file upload inputs to allow uploading multiple files at once.)

In such a case, the specification implementation must aggregate all information related to the file at the given index. The reason is because \$\_FILES deviates from its normal structure in such cases:

\begin{lstlisting}[language=PHP]
array(
    'my-form' => array(
        'details' => array(
            'avatars' => array(
                'tmp_name' => array(
                    0 => '...',
                    1 => '...',
                    2 => '...',
                ),
                'name' => array(
                    0 => '...',
                    1 => '...',
                    2 => '...',
                ),
                'size' => array(
                    0 => '...',
                    1 => '...',
                    2 => '...',
                ),
                'type' => array(
                    0 => '...',
                    1 => '...',
                    2 => '...',
                ),
                'error' => array(
                    0 => '...',
                    1 => '...',
                    2 => '...',
                ),
            ),
        ),
    ),
)
\end{lstlisting}

The above \$\_FILES array would correspond to the following structure as returned by getUploadedFiles():



\begin{lstlisting}[language=PHP]
array(
    'my-form' => array(
        'details' => array(
            'avatars' => array(
                0 => /* UploadedFileInterface instance */,
                1 => /* UploadedFileInterface instance */,
                2 => /* UploadedFileInterface instance */,
            ),
        ),
    ),
)
\end{lstlisting}

Consumers would access index 1 of the nested array using:


\begin{lstlisting}[language=PHP]
$request->getUploadedFiles()['my-form']['details']['avatars'][1];
\end{lstlisting}

Because the uploaded files data is derivative (derived from \$\_FILES or the request body), a mutator method, withUploadedFiles(), is also present in the interface, allowing delegation of the normalization to another process.

In the case of the original examples, consumption resembles the following:



\begin{lstlisting}[language=PHP]
$file0 = $request->getUploadedFiles()['files'][0];
$file1 = $request->getUploadedFiles()['files'][1];

printf(
    "Received the files %s and %s",
    $file0->getClientFilename(),
    $file1->getClientFilename()
);

// "Received the files file0.txt and file1.html"
\end{lstlisting}

This proposal also recognizes that implementations may operate in non-SAPI environments. As such, UploadedFileInterface provides methods for ensuring operations will work regardless of environment. In particular:

\begin{compactitem}
\item moveTo(\$targetPath) is provided as a safe and recommended alternative to calling move\_uploaded\_file() directly on the temporary upload file. Implementations will detect the correct operation to use based on environment.
\item getStream() will return a StreamInterface instance. In non-SAPI environments, one proposed possibility is to parse individual upload files into php://temp streams instead of directly to files; in such cases, no upload file is present. getStream() is therefore guaranteed to work regardless of environment.
\end{compactitem}

As examples:



\begin{lstlisting}[language=PHP]
// Move a file to an upload directory
$filename = sprintf(
    '%s.%s',
    create_uuid(),
    pathinfo($file0->getClientFilename(), PATHINFO_EXTENSION)
);
$file0->moveTo(DATA_DIR . '/' . $filename);

// Stream a file to Amazon S3.
// Assume $s3wrapper is a PHP stream that will write to S3, and that
// Psr7StreamWrapper is a class that will decorate a StreamInterface as a PHP
// StreamWrapper.
$stream = new Psr7StreamWrapper($file1->getStream());
stream_copy_to_stream($stream, $s3wrapper);
\end{lstlisting}

\section{Package}

上面讨论的接口和类库已经整合成为扩展包: psr/http-message。


\section{Interface}

\subsection{Psr\textbackslash Http\textbackslash MessageInterface}




\begin{lstlisting}[language=PHP]
<?php
namespace Psr\Http\Message;

/**
 * 
 * HTTP 消息值得是客户端发起的「请求」和服务器端返回的「响应」,此接口
 * 定义了他们通用的方法。
 * 
 * HTTP 消息是被视为无法修改的,所有能修改状态的方法,都 **必须** 有一套
 * 机制,在内部保持好原有的内容,然后把修改状态后的信息返回。
 *
 * @see http://www.ietf.org/rfc/rfc7230.txt
 * @see http://www.ietf.org/rfc/rfc7231.txt
 */
interface MessageInterface
{
    /**
     * 获取字符串形式的 HTTP 协议版本信息
     *
     * 字符串必须包含 HTTP 版本数字,如:"1.1", "1.0"。
     *
     * @return string HTTP 协议版本
     */
    public function getProtocolVersion();

    /**
     * 返回指定 HTTP 版本号的消息实例。
     *
     * 传参的版本号必须 **只** 包含 HTTP 版本数字,如:"1.1", "1.0"。
     *
     * 此方法在实现的时候,**必须** 保留原有的不可修改的 HTTP 消息对象,然后返回
     * 一个新的带有传参进去的 HTTP 版本的实例
     *
     * @param string $version HTTP 版本信息
     * @return self
     */
    public function withProtocolVersion($version);

    /**
     * 获取所有的头信息
     *
     * 返回的二维数组中,第一维数组的「键」代表单条头信息的名字,「值」是
     * 以数据形式返回的,见以下实例:
     *
     *     // 把「值」的数据当成字串打印出来
     *     foreach ($message->getHeaders() as $name => $values) {
     *         echo $name . ': ' . implode(', ', $values);
     *     }
     *
     *     // 迭代的循环二维数组
     *     foreach ($message->getHeaders() as $name => $values) {
     *         foreach ($values as $value) {
     *             header(sprintf('%s: %s', $name, $value), false);
     *         }
     *     }
     *
     * 虽然头信息是没有大小写之分,但是使用 `getHeaders()` 会返回保留了原本
     * 大小写形式的内容。
     *
     * @return string[][] 返回一个两维数组,第一维数组的「键」 **必须** 为单条头信息的
     *     名称,对应的是由字串组成的数组,请注意,对应的「值」 **必须** 是数组形式的。
     */
    public function getHeaders();

    /**
     * 检查是否头信息中包含有此名称的值,不区分大小写
     *
     * @param string $name 不区分大小写的头信息名称
     * @return bool 找到返回 true,未找到返回 false
     */
    public function hasHeader($name);

    /**
     * 根据给定的名称,获取一条头信息,不区分大小写,以数组形式返回
     *
     * 此方法以数组形式返回对应名称的头信息。
     *
     * 如果没有对应的头信息,**必须** 返回一个空数组。
     *
     * @param string $name 不区分大小写的头部字段名称。
     * @return string[] 返回头信息中,对应名称的,由字符串组成的数组值,如果没有对应
     *  的内容,**必须** 返回空数组。
     */
    public function getHeader($name);

    /**
     * 根据给定的名称,获取一条头信息,不区分大小写,以逗号分隔的形式返回
     * 
     * 此方法返回所有对应的头信息,并将其使用逗号分隔的方法拼接起来。
     *
     * 注意:不是所有的头信息都可使用逗号分隔的方法来拼接,对于那些头信息,请使用
     * `getHeader()` 方法来获取。
     * 
     * 如果没有对应的头信息,此方法 **必须** 返回一个空字符串。
     *
     * @param string $name 不区分大小写的头部字段名称。
     * @return string 返回头信息中,对应名称的,由逗号分隔组成的字串,如果没有对应
     *  的内容,**必须** 返回空字符串。
     */
    public function getHeaderLine($name);

    /**
     * 返回指定头信息「键/值」对的消息实例。
     *
     * 虽然头信息是不区分大小写的,但是此方法必须保留其传参时的大小写状态,并能够在
     * 调用 `getHeaders()` 的时候被取出。
     *
     * 此方法在实现的时候,**必须** 保留原有的不可修改的 HTTP 消息对象,然后返回
     * 一个新的带有传参进去头信息的实例
     *
     * @param string $name Case-insensitive header field name.
     * @param string|string[] $value Header value(s).
     * @return self
     * @throws \InvalidArgumentException for invalid header names or values.
     */
    public function withHeader($name, $value);

    /**
     * 返回一个头信息增量的 HTTP 消息实例。
     *
     * 原有的头信息会被保留,新的值会作为增量加上,如果头信息不存在的话,会被加上。
     *
     * 此方法在实现的时候,**必须** 保留原有的不可修改的 HTTP 消息对象,然后返回
     * 一个新的修改过的 HTTP 消息实例。
     *
     * @param string $name 不区分大小写的头部字段名称。
     * @param string|string[] $value 头信息对应的值。
     * @return self
     * @throws \InvalidArgumentException 头信息字段名称非法时会被抛出。
     * @throws \InvalidArgumentException 头信息的值非法的时候,会被抛出。
     */
    public function withAddedHeader($name, $value);

    /**
     * 返回被移除掉指定头信息的 HTTP 消息实例。
     *
     * 头信息字段在解析的时候,**必须** 保证是不区分大小写的。
     *
     * 此方法在实现的时候,**必须** 保留原有的不可修改的 HTTP 消息对象,然后返回
     * 一个新的修改过的 HTTP 消息实例。
     *
     * @param string $name 不区分大小写的头部字段名称。
     * @return self
     */
    public function withoutHeader($name);

    /**
     * 获取 HTTP 消息的内容。
     *
     * @return StreamInterface 以数据流的形式返回。
     */
    public function getBody();

    /**
     * 返回拼接了内容的 HTTP 消息实例。
     *
     * 内容 **必须** 是 StreamInterface 接口的实例。
     *
     * 此方法在实现的时候,**必须** 保留原有的不可修改的 HTTP 消息对象,然后返回
     * 一个新的修改过的 HTTP 消息实例。
     *
     * @param StreamInterface $body 数据流形式的内容。
     * @return self
     * @throws \InvalidArgumentException 当消息内容不正确的时候。
     */
    public function withBody(StreamInterface $body);
}
\end{lstlisting}

\subsection{Psr\textbackslash Http\textbackslash Message\textbackslash RequestInterface}




\begin{lstlisting}[language=PHP]
<?php
namespace Psr\Http\Message;

/**
 * 代表客户端请求的 HTTP 消息对象。
 *
 * 根据规范,每一个 HTTP 请求都包含以下信息:
 *
 * - HTTP 协议版本号 (Protocol version)
 * - HTTP 请求方法 (HTTP method)
 * - URI
 * - 头信息 (Headers)
 * - 消息内容 (Message body)
 *
 * 在构造 HTTP 请求对象的时候,实现类库 **必须** 从给出的 URI 中去提取 HOST 信息。
 *
 * HTTP 请求是被视为无法修改的,所有能修改状态的方法,都 **必须** 有一套机制,在内部保
 * 持好原有的内容,然后把修改状态后的,新的 HTTP 请求实例返回。
 */
interface RequestInterface extends MessageInterface
{
    /**
     * 获取消息请求的目标。
     *
     * 在大部分情况下,此方法会返回完整的 URI,除非 `withRequestTarget()` 被设置过。
     *
     * 如果没有提供 URI,并且没有提供任何的请求目标,此方法 **必须** 返回 "/"。
     *
     * @return string
     */
    public function getRequestTarget();

    /**
     * 返回一个指定目标的请求实例。
     *
     * 此方法在实现的时候,**必须** 保留原有的不可修改的 HTTP 请求实例,然后返回
     * 一个新的修改过的 HTTP 请求实例。
     *
     * @see 关于请求目标的各种允许的格式,请见 http://tools.ietf.org/html/rfc7230#section-2.7 
     * 
     * @param mixed $requestTarget
     * @return self
     */
    public function withRequestTarget($requestTarget);

    /**
     * 获取当前请求使用的 HTTP 方法
     *
     * @return string HTTP 方法字符串
     */
    public function getMethod();

    /**
     * 返回更改了请求方法的消息实例。
     *
     * 虽然,在大部分情况下,HTTP 请求方法都是使用大写字母来标示的,但是,实现类库 **一定不可**
     * 修改用户传参的大小格式。
     * 
     * 此方法在实现的时候,**必须** 保留原有的不可修改的 HTTP 请求实例,然后返回
     * 一个新的修改过的 HTTP 请求实例。
     *
     * @param string $method 大小写敏感的方法名
     * @return self
     * @throws \InvalidArgumentException 当非法的 HTTP 方法名传入时会抛出异常。
     */
    public function withMethod($method);

    /**
     * 获取 URI 实例。
     *
     * 此方法必须返回 `UriInterface` 的 URI 实例。
     *
     * @see http://tools.ietf.org/html/rfc3986#section-4.3
     * @return UriInterface 返回与当前请求相关 `UriInterface` 类型的 URI 实例。
     */
    public function getUri();

    /**
     * 返回修改了 URI 的消息实例。
     *
     * 当传入的 `URI` 包含有 `HOST` 信息时,**必须** 更新 `HOST` 头信息,如果 `URI` 
     * 实例没有附带 `HOST` 信息,任何之前存在的 `HOST` 信息 **必须** 作为候补,应用
     * 更改到返回的消息实例里。
     * 
     * 你可以通过传入第二个参数来,来干预方法的处理,当 `$preserveHost` 设置为 `true` 
     * 的时候,会保留原来的 `HOST` 信息。
     * 
     * 此方法在实现的时候,**必须** 保留原有的不可修改的 HTTP 请求实例,然后返回
     * 一个新的修改过的 HTTP 请求实例。
     *
     * @see http://tools.ietf.org/html/rfc3986#section-4.3
     * @param UriInterface $uri `UriInterface` 类型的 URI 实例
     * @param bool $preserveHost 是否保留原有的 HOST 头信息
     * @return self
     */
    public function withUri(UriInterface $uri, $preserveHost = false);
}
\end{lstlisting}

\subsection{Psr\textbackslash Http\textbackslash Message\textbackslash ServerRequestInterface}



\begin{lstlisting}[language=PHP]
<?php
namespace Psr\Http\Message;

/**
 * Representation of an incoming, server-side HTTP request.
 *
 * Per the HTTP specification, this interface includes properties for
 * each of the following:
 *
 * - Protocol version
 * - HTTP method
 * - URI
 * - Headers
 * - Message body
 *
 * Additionally, it encapsulates all data as it has arrived to the
 * application from the CGI and/or PHP environment, including:
 *
 * - The values represented in $_SERVER.
 * - Any cookies provided (generally via $_COOKIE)
 * - Query string arguments (generally via $_GET, or as parsed via parse_str())
 * - Upload files, if any (as represented by $_FILES)
 * - Deserialized body parameters (generally from $_POST)
 *
 * $_SERVER values MUST be treated as immutable, as they represent application
 * state at the time of request; as such, no methods are provided to allow
 * modification of those values. The other values provide such methods, as they
 * can be restored from $_SERVER or the request body, and may need treatment
 * during the application (e.g., body parameters may be deserialized based on
 * content type).
 *
 * Additionally, this interface recognizes the utility of introspecting a
 * request to derive and match additional parameters (e.g., via URI path
 * matching, decrypting cookie values, deserializing non-form-encoded body
 * content, matching authorization headers to users, etc). These parameters
 * are stored in an "attributes" property.
 *
 * HTTP 请求是被视为无法修改的,所有能修改状态的方法,都 **必须** 有一套机制,在内部保
 * 持好原有的内容,然后把修改状态后的,新的 HTTP 请求实例返回。
 */
interface ServerRequestInterface extends RequestInterface
{
    /**
     * Retrieve server parameters.
     *
     * Retrieves data related to the incoming request environment,
     * typically derived from PHP's $_SERVER superglobal. The data IS NOT
     * REQUIRED to originate from $_SERVER.
     *
     * @return array
     */
    public function getServerParams();

    /**
     * Retrieve cookies.
     *
     * Retrieves cookies sent by the client to the server.
     *
     * The data MUST be compatible with the structure of the $_COOKIE
     * superglobal.
     *
     * @return array
     */
    public function getCookieParams();

    /**
     * Return an instance with the specified cookies.
     *
     * The data IS NOT REQUIRED to come from the $_COOKIE superglobal, but MUST
     * be compatible with the structure of $_COOKIE. Typically, this data will
     * be injected at instantiation.
     *
     * This method MUST NOT update the related Cookie header of the request
     * instance, nor related values in the server params.
     *
     * 此方法在实现的时候,**必须** 保留原有的不可修改的 HTTP 消息实例,然后返回
     * 一个新的修改过的 HTTP 消息实例。
     * 
     * @param array $cookies Array of key/value pairs representing cookies.
     * @return self
     */
    public function withCookieParams(array $cookies);

    /**
     * Retrieve query string arguments.
     *
     * Retrieves the deserialized query string arguments, if any.
     *
     * Note: the query params might not be in sync with the URI or server
     * params. If you need to ensure you are only getting the original
     * values, you may need to parse the query string from `getUri()->getQuery()`
     * or from the `QUERY_STRING` server param.
     *
     * @return array
     */
    public function getQueryParams();

    /**
     * Return an instance with the specified query string arguments.
     *
     * These values SHOULD remain immutable over the course of the incoming
     * request. They MAY be injected during instantiation, such as from PHP's
     * $_GET superglobal, or MAY be derived from some other value such as the
     * URI. In cases where the arguments are parsed from the URI, the data
     * MUST be compatible with what PHP's parse_str() would return for
     * purposes of how duplicate query parameters are handled, and how nested
     * sets are handled.
     *
     * Setting query string arguments MUST NOT change the URI stored by the
     * request, nor the values in the server params.
     *
     * 此方法在实现的时候,**必须** 保留原有的不可修改的 HTTP 消息实例,然后返回
     * 一个新的修改过的 HTTP 消息实例。
     *
     * @param array $query Array of query string arguments, typically from
     *     $_GET.
     * @return self
     */
    public function withQueryParams(array $query);

    /**
     * Retrieve normalized file upload data.
     *
     * This method returns upload metadata in a normalized tree, with each leaf
     * an instance of Psr\Http\Message\UploadedFileInterface.
     *
     * These values MAY be prepared from $_FILES or the message body during
     * instantiation, or MAY be injected via withUploadedFiles().
     *
     * @return array An array tree of UploadedFileInterface instances; an empty
     *     array MUST be returned if no data is present.
     */
    public function getUploadedFiles();

    /**
     * Create a new instance with the specified uploaded files.
     *
     * 此方法在实现的时候,**必须** 保留原有的不可修改的 HTTP 消息实例,然后返回
     * 一个新的修改过的 HTTP 消息实例。
     *
     * @param array An array tree of UploadedFileInterface instances.
     * @return self
     * @throws \InvalidArgumentException if an invalid structure is provided.
     */
    public function withUploadedFiles(array $uploadedFiles);

    /**
     * Retrieve any parameters provided in the request body.
     *
     * If the request Content-Type is either application/x-www-form-urlencoded
     * or multipart/form-data, and the request method is POST, this method MUST
     * return the contents of $_POST.
     *
     * Otherwise, this method may return any results of deserializing
     * the request body content; as parsing returns structured content, the
     * potential types MUST be arrays or objects only. A null value indicates
     * the absence of body content.
     *
     * @return null|array|object The deserialized body parameters, if any.
     *     These will typically be an array or object.
     */
    public function getParsedBody();

    /**
     * Return an instance with the specified body parameters.
     *
     * These MAY be injected during instantiation.
     *
     * If the request Content-Type is either application/x-www-form-urlencoded
     * or multipart/form-data, and the request method is POST, use this method
     * ONLY to inject the contents of $_POST.
     *
     * The data IS NOT REQUIRED to come from $_POST, but MUST be the results of
     * deserializing the request body content. Deserialization/parsing returns
     * structured data, and, as such, this method ONLY accepts arrays or objects,
     * or a null value if nothing was available to parse.
     *
     * As an example, if content negotiation determines that the request data
     * is a JSON payload, this method could be used to create a request
     * instance with the deserialized parameters.
     *
     * 此方法在实现的时候,**必须** 保留原有的不可修改的 HTTP 消息实例,然后返回
     * 一个新的修改过的 HTTP 消息实例。
     *
     * @param null|array|object $data The deserialized body data. This will
     *     typically be in an array or object.
     * @return self
     * @throws \InvalidArgumentException if an unsupported argument type is
     *     provided.
     */
    public function withParsedBody($data);

    /**
     * Retrieve attributes derived from the request.
     *
     * The request "attributes" may be used to allow injection of any
     * parameters derived from the request: e.g., the results of path
     * match operations; the results of decrypting cookies; the results of
     * deserializing non-form-encoded message bodies; etc. Attributes
     * will be application and request specific, and CAN be mutable.
     *
     * @return mixed[] Attributes derived from the request.
     */
    public function getAttributes();

    /**
     * Retrieve a single derived request attribute.
     *
     * Retrieves a single derived request attribute as described in
     * getAttributes(). If the attribute has not been previously set, returns
     * the default value as provided.
     *
     * This method obviates the need for a hasAttribute() method, as it allows
     * specifying a default value to return if the attribute is not found.
     *
     * @see getAttributes()
     * @param string $name The attribute name.
     * @param mixed $default Default value to return if the attribute does not exist.
     * @return mixed
     */
    public function getAttribute($name, $default = null);

    /**
     * Return an instance with the specified derived request attribute.
     *
     * This method allows setting a single derived request attribute as
     * described in getAttributes().
     *
     * 此方法在实现的时候,**必须** 保留原有的不可修改的 HTTP 消息实例,然后返回
     * 一个新的修改过的 HTTP 消息实例。
     *
     * @see getAttributes()
     * @param string $name The attribute name.
     * @param mixed $value The value of the attribute.
     * @return self
     */
    public function withAttribute($name, $value);

    /**
     * Return an instance that removes the specified derived request attribute.
     *
     * This method allows removing a single derived request attribute as
     * described in getAttributes().
     *
     * 此方法在实现的时候,**必须** 保留原有的不可修改的 HTTP 消息实例,然后返回
     * 一个新的修改过的 HTTP 消息实例。
     *
     * @see getAttributes()
     * @param string $name The attribute name.
     * @return self
     */
    public function withoutAttribute($name);
}
\end{lstlisting}

\subsection{Psr\textbackslash Http\textbackslash Message\textbackslash ResponseInterface}


\begin{lstlisting}[language=PHP]
<?php
namespace Psr\Http\Message;

/**
 * Representation of an outgoing, server-side response.
 *
 * Per the HTTP specification, this interface includes properties for
 * each of the following:
 *
 * - Protocol version
 * - Status code and reason phrase
 * - Headers
 * - Message body
 *
 * Responses are considered immutable; all methods that might change state MUST
 * be implemented such that they retain the internal state of the current
 * message and return an instance that contains the changed state.
 * 
 * HTTP 响应是被视为无法修改的,所有能修改状态的方法,都 **必须** 有一套机制,在内部保
 * 持好原有的内容,然后把修改状态后的,新的 HTTP 响应实例返回。
 */
interface ResponseInterface extends MessageInterface
{
    /**
     * Gets the response status code.
     *
     * The status code is a 3-digit integer result code of the server's attempt
     * to understand and satisfy the request.
     *
     * @return int Status code.
     */
    public function getStatusCode();

    /**
     * Return an instance with the specified status code and, optionally, reason phrase.
     *
     * If no reason phrase is specified, implementations MAY choose to default
     * to the RFC 7231 or IANA recommended reason phrase for the response's
     * status code.
     *
     * 此方法在实现的时候,**必须** 保留原有的不可修改的 HTTP 消息实例,然后返回
     * 一个新的修改过的 HTTP 消息实例。
     *
     * @see http://tools.ietf.org/html/rfc7231#section-6
     * @see http://www.iana.org/assignments/http-status-codes/http-status-codes.xhtml
     * @param int $code The 3-digit integer result code to set.
     * @param string $reasonPhrase The reason phrase to use with the
     *     provided status code; if none is provided, implementations MAY
     *     use the defaults as suggested in the HTTP specification.
     * @return self
     * @throws \InvalidArgumentException For invalid status code arguments.
     */
    public function withStatus($code, $reasonPhrase = '');

    /**
     * Gets the response reason phrase associated with the status code.
     *
     * Because a reason phrase is not a required element in a response
     * status line, the reason phrase value MAY be empty. Implementations MAY
     * choose to return the default RFC 7231 recommended reason phrase (or those
     * listed in the IANA HTTP Status Code Registry) for the response's
     * status code.
     *
     * @see http://tools.ietf.org/html/rfc7231#section-6
     * @see http://www.iana.org/assignments/http-status-codes/http-status-codes.xhtml
     * @return string Reason phrase; must return an empty string if none present.
     */
    public function getReasonPhrase();
}
\end{lstlisting}

\subsection{Psr\textbackslash Http\textbackslash Message\textbackslash StreamInterface}


\begin{lstlisting}[language=PHP]
<?php
namespace Psr\Http\Message;

/**
 * Describes a data stream.
 *
 * Typically, an instance will wrap a PHP stream; this interface provides
 * a wrapper around the most common operations, including serialization of
 * the entire stream to a string.
 */
interface StreamInterface
{
    /**
     * Reads all data from the stream into a string, from the beginning to end.
     *
     * This method MUST attempt to seek to the beginning of the stream before
     * reading data and read the stream until the end is reached.
     *
     * Warning: This could attempt to load a large amount of data into memory.
     *
     * This method MUST NOT raise an exception in order to conform with PHP's
     * string casting operations.
     *
     * @see http://php.net/manual/en/language.oop5.magic.php#object.tostring
     * @return string
     */
    public function __toString();

    /**
     * Closes the stream and any underlying resources.
     *
     * @return void
     */
    public function close();

    /**
     * Separates any underlying resources from the stream.
     *
     * After the stream has been detached, the stream is in an unusable state.
     *
     * @return resource|null Underlying PHP stream, if any
     */
    public function detach();

    /**
     * Get the size of the stream if known.
     *
     * @return int|null Returns the size in bytes if known, or null if unknown.
     */
    public function getSize();

    /**
     * Returns the current position of the file read/write pointer
     *
     * @return int Position of the file pointer
     * @throws \RuntimeException on error.
     */
    public function tell();

    /**
     * Returns true if the stream is at the end of the stream.
     *
     * @return bool
     */
    public function eof();

    /**
     * Returns whether or not the stream is seekable.
     *
     * @return bool
     */
    public function isSeekable();

    /**
     * Seek to a position in the stream.
     *
     * @see http://www.php.net/manual/en/function.fseek.php
     * @param int $offset Stream offset
     * @param int $whence Specifies how the cursor position will be calculated
     *     based on the seek offset. Valid values are identical to the built-in
     *     PHP $whence values for `fseek()`.  SEEK_SET: Set position equal to
     *     offset bytes SEEK_CUR: Set position to current location plus offset
     *     SEEK_END: Set position to end-of-stream plus offset.
     * @throws \RuntimeException on failure.
     */
    public function seek($offset, $whence = SEEK_SET);

    /**
     * Seek to the beginning of the stream.
     *
     * If the stream is not seekable, this method will raise an exception;
     * otherwise, it will perform a seek(0).
     *
     * @see seek()
     * @see http://www.php.net/manual/en/function.fseek.php
     * @throws \RuntimeException on failure.
     */
    public function rewind();

    /**
     * Returns whether or not the stream is writable.
     *
     * @return bool
     */
    public function isWritable();

    /**
     * Write data to the stream.
     *
     * @param string $string The string that is to be written.
     * @return int Returns the number of bytes written to the stream.
     * @throws \RuntimeException on failure.
     */
    public function write($string);

    /**
     * Returns whether or not the stream is readable.
     *
     * @return bool
     */
    public function isReadable();

    /**
     * Read data from the stream.
     *
     * @param int $length Read up to $length bytes from the object and return
     *     them. Fewer than $length bytes may be returned if underlying stream
     *     call returns fewer bytes.
     * @return string Returns the data read from the stream, or an empty string
     *     if no bytes are available.
     * @throws \RuntimeException if an error occurs.
     */
    public function read($length);

    /**
     * Returns the remaining contents in a string
     *
     * @return string
     * @throws \RuntimeException if unable to read.
     * @throws \RuntimeException if error occurs while reading.
     */
    public function getContents();

    /**
     * Get stream metadata as an associative array or retrieve a specific key.
     *
     * The keys returned are identical to the keys returned from PHP's
     * stream_get_meta_data() function.
     *
     * @see http://php.net/manual/en/function.stream-get-meta-data.php
     * @param string $key Specific metadata to retrieve.
     * @return array|mixed|null Returns an associative array if no key is
     *     provided. Returns a specific key value if a key is provided and the
     *     value is found, or null if the key is not found.
     */
    public function getMetadata($key = null);
}
\end{lstlisting}

\subsection{Psr\textbackslash Http\textbackslash Message\textbackslash UriInterface}


\begin{lstlisting}[language=PHP]
<?php
namespace Psr\Http\Message;

/**
 * URI 数据对象。
 *
 * 此接口按照 RFC 3986 来构建 HTTP URI,提供了一些通用的操作,你可以自由的对此接口
 * 进行扩展。你可以使用此 URI 接口来做 HTTP 相关的操作,也可以使用此接口做任何 URI 
 * 相关的操作。
 *
 * 此接口的实例化对象被视为无法修改的,所有能修改状态的方法,都 **必须** 有一套机制,在内部保
 * 持好原有的内容,然后把修改状态后的,新的实例返回。
 *
 * @see http://tools.ietf.org/html/rfc3986 (URI 通用标准规范)
 */
interface UriInterface
{
    /**
     * 从 URI 中取出 scheme。
     *
     * 如果不存在 Scheme,此方法 **必须** 返回空字符串。
     *
     * 返回的数据 **必须** 是小写字母,遵照  RFC 3986 规范 3.1 章节。
     *
     * 最后部分的 ":" 字串不属于 Scheme,**一定不可** 作为返回数据的一部分。
     *
     * @see https://tools.ietf.org/html/rfc3986#section-3.1
     * @return string URI scheme 的值
     */
    public function getScheme();

    /**
     * 返回 URI 授权信息。
     *
     * 如果没有 URI 信息的话,**必须** 返回一个空数组。
     *
     * URI 的授权信息语法是:
     *
     * <pre>
     * [user-info@]host[:port]
     * </pre>
     *
     * 如果端口部分没有设置,或者端口不是标准端口,**一定不可** 包含在返回值内。
     *
     * @see https://tools.ietf.org/html/rfc3986#section-3.2
     * @return string URI 授权信息,格式为:"[user-info@]host[:port]" 
     */
    public function getAuthority();

    /**
     * 从 URI 中获取用户信息。
     *
     * 如果不存在用户信息,此方法 **必须** 返回一个空字符串。
     *
     * 用户信息后面跟着的 "@" 字符,不是用户信息里面的一部分,**一定不可** 在返回值里
     * 出现。
     *
     * @return string URI 的用户信息,格式:"username[:password]" 
     */
    public function getUserInfo();

    /**
     * 从 URI 信息中获取 HOST 值。
     *
     * 如果 URI 中没有此值,**必须** 返回空字符串。
     *
     * 返回的数据 **必须** 是小写字母,遵照  RFC 3986 规范 3.2.2 章节。
     *
     * @see http://tools.ietf.org/html/rfc3986#section-3.2.2
     * @return string URI 信息中的 HOST 值。
     */
    public function getHost();

    /**
     * 从 URI 信息中获取端口信息。
     *
     * 如果端口信息是与当前 Scheme 的标准端口不匹配的话,就使用整数值的格式返回,如果是一
     * 样的话,**必须** 返回 `null` 值。
     * 
     * 如果存在端口信息,都是不存在 scheme 信息的话,**必须** 返回 `null` 值。
     * 
     * 如果不存在端口数据,但是 scheme 数据存在的话,**可以** 返回 scheme 对应的
     * 标准端口,但是 **应该** 返回 `null`。
     * 
     * @return null|int 从 URI 信息中的端口信息。
     */
    public function getPort();

    /**
     * 从 URI 信息中获取路径。
     *
     *  The path can either be empty or absolute (starting with a slash) or
     * rootless (not starting with a slash). Implementations MUST support all
     * three syntaxes.
     *
     * Normally, the empty path "" and absolute path "/" are considered equal as
     * defined in RFC 7230 Section 2.7.3. But this method MUST NOT automatically
     * do this normalization because in contexts with a trimmed base path, e.g.
     * the front controller, this difference becomes significant. It's the task
     * of the user to handle both "" and "/".
     *
     * The value returned MUST be percent-encoded, but MUST NOT double-encode
     * any characters. To determine what characters to encode, please refer to
     * RFC 3986, Sections 2 and 3.3.
     *
     * As an example, if the value should include a slash ("/") not intended as
     * delimiter between path segments, that value MUST be passed in encoded
     * form (e.g., "%2F") to the instance.
     *
     * @see https://tools.ietf.org/html/rfc3986#section-2
     * @see https://tools.ietf.org/html/rfc3986#section-3.3
     * @return string The URI path.
     */
    public function getPath();

    /**
     * Retrieve the query string of the URI.
     *
     * If no query string is present, this method MUST return an empty string.
     *
     * The leading "?" character is not part of the query and MUST NOT be
     * added.
     *
     * The value returned MUST be percent-encoded, but MUST NOT double-encode
     * any characters. To determine what characters to encode, please refer to
     * RFC 3986, Sections 2 and 3.4.
     *
     * As an example, if a value in a key/value pair of the query string should
     * include an ampersand ("&") not intended as a delimiter between values,
     * that value MUST be passed in encoded form (e.g., "%26") to the instance.
     *
     * @see https://tools.ietf.org/html/rfc3986#section-2
     * @see https://tools.ietf.org/html/rfc3986#section-3.4
     * @return string The URI query string.
     */
    public function getQuery();

    /**
     * Retrieve the fragment component of the URI.
     *
     * If no fragment is present, this method MUST return an empty string.
     *
     * The leading "#" character is not part of the fragment and MUST NOT be
     * added.
     *
     * The value returned MUST be percent-encoded, but MUST NOT double-encode
     * any characters. To determine what characters to encode, please refer to
     * RFC 3986, Sections 2 and 3.5.
     *
     * @see https://tools.ietf.org/html/rfc3986#section-2
     * @see https://tools.ietf.org/html/rfc3986#section-3.5
     * @return string The URI fragment.
     */
    public function getFragment();

    /**
     * Return an instance with the specified scheme.
     *
     * This method MUST retain the state of the current instance, and return
     * an instance that contains the specified scheme.
     *
     * Implementations MUST support the schemes "http" and "https" case
     * insensitively, and MAY accommodate other schemes if required.
     *
     * An empty scheme is equivalent to removing the scheme.
     *
     * @param string $scheme The scheme to use with the new instance.
     * @return self A new instance with the specified scheme.
     * @throws \InvalidArgumentException for invalid schemes.
     * @throws \InvalidArgumentException for unsupported schemes.
     */
    public function withScheme($scheme);

    /**
     * Return an instance with the specified user information.
     *
     * This method MUST retain the state of the current instance, and return
     * an instance that contains the specified user information.
     *
     * Password is optional, but the user information MUST include the
     * user; an empty string for the user is equivalent to removing user
     * information.
     *
     * @param string $user The user name to use for authority.
     * @param null|string $password The password associated with $user.
     * @return self A new instance with the specified user information.
     */
    public function withUserInfo($user, $password = null);

    /**
     * Return an instance with the specified host.
     *
     * This method MUST retain the state of the current instance, and return
     * an instance that contains the specified host.
     *
     * An empty host value is equivalent to removing the host.
     *
     * @param string $host The hostname to use with the new instance.
     * @return self A new instance with the specified host.
     * @throws \InvalidArgumentException for invalid hostnames.
     */
    public function withHost($host);

    /**
     * Return an instance with the specified port.
     *
     * This method MUST retain the state of the current instance, and return
     * an instance that contains the specified port.
     *
     * Implementations MUST raise an exception for ports outside the
     * established TCP and UDP port ranges.
     *
     * A null value provided for the port is equivalent to removing the port
     * information.
     *
     * @param null|int $port The port to use with the new instance; a null value
     *     removes the port information.
     * @return self A new instance with the specified port.
     * @throws \InvalidArgumentException for invalid ports.
     */
    public function withPort($port);

    /**
     * Return an instance with the specified path.
     *
     * This method MUST retain the state of the current instance, and return
     * an instance that contains the specified path.
     *
     * The path can either be empty or absolute (starting with a slash) or
     * rootless (not starting with a slash). Implementations MUST support all
     * three syntaxes.
     *
     * If an HTTP path is intended to be host-relative rather than path-relative
     * then it must begin with a slash ("/"). HTTP paths not starting with a slash
     * are assumed to be relative to some base path known to the application or
     * consumer.
     *
     * Users can provide both encoded and decoded path characters.
     * Implementations ensure the correct encoding as outlined in getPath().
     *
     * @param string $path The path to use with the new instance.
     * @return self A new instance with the specified path.
     * @throws \InvalidArgumentException for invalid paths.
     */
    public function withPath($path);

    /**
     * Return an instance with the specified query string.
     *
     * This method MUST retain the state of the current instance, and return
     * an instance that contains the specified query string.
     *
     * Users can provide both encoded and decoded query characters.
     * Implementations ensure the correct encoding as outlined in getQuery().
     *
     * An empty query string value is equivalent to removing the query string.
     *
     * @param string $query The query string to use with the new instance.
     * @return self A new instance with the specified query string.
     * @throws \InvalidArgumentException for invalid query strings.
     */
    public function withQuery($query);

    /**
     * Return an instance with the specified URI fragment.
     *
     * This method MUST retain the state of the current instance, and return
     * an instance that contains the specified URI fragment.
     *
     * Users can provide both encoded and decoded fragment characters.
     * Implementations ensure the correct encoding as outlined in getFragment().
     *
     * An empty fragment value is equivalent to removing the fragment.
     *
     * @param string $fragment The fragment to use with the new instance.
     * @return self A new instance with the specified fragment.
     */
    public function withFragment($fragment);

    /**
     * Return the string representation as a URI reference.
     *
     * Depending on which components of the URI are present, the resulting
     * string is either a full URI or relative reference according to RFC 3986,
     * Section 4.1. The method concatenates the various components of the URI,
     * using the appropriate delimiters:
     *
     * - If a scheme is present, it MUST be suffixed by ":".
     * - If an authority is present, it MUST be prefixed by "//".
     * - The path can be concatenated without delimiters. But there are two
     *   cases where the path has to be adjusted to make the URI reference
     *   valid as PHP does not allow to throw an exception in __toString():
     *     - If the path is rootless and an authority is present, the path MUST
     *       be prefixed by "/".
     *     - If the path is starting with more than one "/" and no authority is
     *       present, the starting slashes MUST be reduced to one.
     * - If a query is present, it MUST be prefixed by "?".
     * - If a fragment is present, it MUST be prefixed by "#".
     *
     * @see http://tools.ietf.org/html/rfc3986#section-4.1
     * @return string
     */
    public function __toString();
}
\end{lstlisting}


\subsection{Psr\textbackslash Http\textbackslash Message\textbackslash UploadedFileInterface}



\begin{lstlisting}[language=PHP]
<?php
namespace Psr\Http\Message;

/**
 * 通过 HTTP 请求上传的一个文件内容。
 *
 * 此接口的实例是被视为无法修改的,所有能修改状态的方法,都 **必须** 有一套机制,在内部保
 * 持好原有的内容,然后把修改状态后的,新的实例返回。
 */
interface UploadedFileInterface
{
    /**
     * 获取上传文件的数据流。
     *
     * 此方法必须返回一个 `StreamInterface` 实例,此方法的目的在于允许 PHP 对获取到的数
     * 据流直接操作,如 stream_copy_to_stream() 。
     *
     * 如果在调用此方法之前调用了 `moveTo()` 方法,此方法 **必须** 抛出异常。
     *
     * @return StreamInterface 上传文件的数据流
     * @throws \RuntimeException 没有数据流的情形下。
     * @throws \RuntimeException 无法创建数据流。
     */
    public function getStream();

    /**
     * 把上传的文件移动到新目录。
     *
     * 此方法保证能同时在 `SAPI` 和 `non-SAPI` 环境下使用。实现类库 **必须** 判断
     * 当前处在什么环境下,并且使用合适的方法来处理,如 move_uploaded_file(), rename()
     * 或者数据流操作。
     *
     * $targetPath 可以是相对路径,也可以是绝对路径,使用 rename() 解析起来应该是一样的。
     *
     * 当这一次完成后,原来的文件 **必须** 会被移除。
     * 
     * 如果此方法被调用多次,一次以后的其他调用,都要抛出异常。
     *
     * 如果在 SAPI 环境下的话,$_FILES 内有值,当使用  moveTo(), is_uploaded_file()
     * 和 move_uploaded_file() 方法来移动文件时 **应该** 确保权限和上传状态的准确性。
     * 
     * 如果你希望操作数据流的话,请使用 `getStream()` 方法,因为在 SAPI 场景下,无法
     * 保证书写入数据流目标。
     * 
     * @see http://php.net/is_uploaded_file
     * @see http://php.net/move_uploaded_file
     * @param string $targetPath 目标文件路径。
     * @throws \InvalidArgumentException 参数有问题时抛出异常。
     * @throws \RuntimeException 发生任何错误,都抛出此异常。
     * @throws \RuntimeException 多次运行,也抛出此异常。
     */
    public function moveTo($targetPath);

    /**
     * 获取文件大小。
     *
     * 实现类库 **应该** 优先使用 $_FILES 里的 `size` 数值。
     * 
     * @return int|null 以 bytes 为单位,或者 null 未知的情况下。
     */
    public function getSize();

    /**
     * 获取上传文件时出现的错误。
     *
     * 返回值 **必须** 是 PHP 的 UPLOAD_ERR_XXX 常量。
     *
     * 如果文件上传成功,此方法 **必须** 返回 UPLOAD_ERR_OK。
     *
     * 实现类库 **必须** 返回 $_FILES 数组中的 `error` 值。
     * 
     * @see http://php.net/manual/en/features.file-upload.errors.php
     * @return int PHP 的 UPLOAD_ERR_XXX 常量。
     */
    public function getError();

    /**
     * 获取客户端上传的文件的名称。
     * 
     * 永远不要信任此方法返回的数据,客户端有可能发送了一个恶意的文件名来攻击你的程序。
     * 
     * 实现类库 **应该** 返回存储在 $_FILES 数组中 `name` 的值。
     *
     * @return string|null 用户上传的名字,或者 null 如果没有此值。
     */
    public function getClientFilename();

    /**
     * 客户端提交的文件类型。
     * 
     * 永远不要信任此方法返回的数据,客户端有可能发送了一个恶意的文件类型名称来攻击你的程序。
     *
     * 实现类库 **应该** 返回存储在 $_FILES 数组中 `type` 的值。
     *
     * @return string|null 用户上传的类型,或者 null 如果没有此值。
     */
    public function getClientMediaType();
}
\end{lstlisting}

\chapter{PSR-8}


\section{Overview}

This standard establishes a common way for objects to express mutual appreciation and support by hugging. This allows objects to support each other in a constructive fashion, furthering cooperation between different PHP projects.

This specification defines two interfaces, \textbackslash Psr\textbackslash Hug\textbackslash Huggable and \textbackslash Psr\textbackslash Hug\textbackslash GroupHuggable.

\subsection{Huggable Objects}

\begin{compactenum}
\item A Huggable object expresses affection and support for another object by invoking its hug() method, passing \$this as the first parameter.

\item An object whose hug() method is invoked MUST hug() the calling object back at least once.

\item Two objects that are engaged in a hug MAY continue to hug each other back for any number of iterations. However, every huggable object MUST have a termination condition that will prevent an infinite loop. For example, an object MAY be configured to only allow up to 3 mutual hugs, after which it will break the hug chain and return.

\item An object MAY take additional actions, including modifying state, when hugged. A common example is to increment an internal happiness or satisfaction counter.

\end{compactenum}

\subsection{GroupHuggable objects}

\begin{compactenum}
\item An object may optionally implement GroupHuggable to indicate that it is able to support and affirm multiple objects at once.
\end{compactenum}

\section{Interface}


\subsection{Psr\textbackslash Hug\textbackslash HuggableInterface}



\begin{lstlisting}[language=PHP]
namespace Psr\Hug;

/**
 * Defines a huggable object.
 *
 * A huggable object expresses mutual affection with another huggable object.
 */
interface Huggable
{

    /**
     * Hugs this object.
     *
     * All hugs are mutual. An object that is hugged MUST in turn hug the other
     * object back by calling hug() on the first parameter. All objects MUST
     * implement a mechanism to prevent an infinite loop of hugging.
     *
     * @param Huggable $h
     *   The object that is hugging this object.
     */
    public function hug(Huggable $h);
}
\end{lstlisting}


\subsection{Psr\textbackslash Hug\textbackslash GroupHuggable}


\begin{lstlisting}[language=PHP]
namespace Psr\Hug;

/**
 * Defines a huggable object.
 *
 * A huggable object expresses mutual affection with another huggable object.
 */
interface GroupHuggable extends Huggable
{

  /**
   * Hugs a series of huggable objects.
   *
   * When called, this object MUST invoke the hug() method of every object
   * provided. The order of the collection is not significant, and this object
   * MAY hug each of the objects in any order provided that all are hugged.
   *
   * @param $huggables
   *   An array or iterator of objects implementing the Huggable interface.
   */
  public function groupHug($huggables);
}
\end{lstlisting}


\chapter{PSR-9}

\section{Overview}


There are two aspects with dealing with security issues: One is the process by which security issues are reported and fixed in projects, the other is how the general public is informed about the issues and any remedies available. While PSR-9 addresses the former, this PSR, ie. PSR-10, deals with the later. So the goal of PSR-10 is to define how security issues are disclosed to the public and what format such disclosures should follow. Especially today where PHP developers are sharing code across projects more than ever, this PSR aims to ease the challenges in keeping an overview of security issues in all dependencies and the steps required to address them.

The goal of this PSR is to give project leads a clearly defined approach to enabling end users to discover security disclosures using a clearly defined structured format for these disclosures.

\section{Disclosure Discovery}

Every project MUST provide a link to its security vulnerability database in an obvious place. Ideally this should be on the root page of the main domain of the given project. This MAY be a sub-domain in case it is a sub-project of a larger initiative. If the project has a dedicated page for its disclosure process discovery then this is also considered a good place for this link. The link MAY use the custom link relation php-vuln-disclosures, ie. for example <link rel="php-vuln-disclosures" href="http://example.org/disclosures"/>.

Note that projects MAY choose to host their disclosure files on a domain other than their main project page. It is RECOMMENDED to not store the disclosures in a VCS as this can lead to the confusions about which branch is the relevant branch. If a VCS is used then additional steps SHOULD be taken to clearly document to users which branch contains all vulnerabilities for all versions. If necessary projects MAY however split vulnerability disclosure files by major version number. In this case again this SHOULD be clearly documented.


\section{Disclosure Format}

The disclosure format is based on Atom [1], which in turn is based on XML. It leverages the "The Common Vulnerability Reporting Framework (CVRF) v1.1" [2]. Specifically it leverages its dictionary [3] as its base terminology.

TODO: Should we also provide a JSON serialization to lower the bar for projects. Aggregation services can then spring up to provide an Atom representation of these disclosures in JSON format.

The Atom extensions [4] allow a structured description of the vulnerability to enable automated tools to determine if installed is likely affected by the vulnerability. However human readability is considered highly important and as such not the full CVRF is used.

TODO: Review the Atom format and the supplied XSD

Note that for each vulnerability only a single entry MUST be created. In case any information changes the original file MUST be updated along with the last update field.

Any disclosure uses entryType using the following tags from the Atom namespace (required tags are labeled with "MUST"):

\begin{compactitem}
\item title (short description of the vulnerability and affected versions, MUST)
\item summary (description of the vulnerability)
\item author (contact information, MUST)
\item published (initial publication date, MUST)
\item updated (date of the last update)
\item link (to reference more information)
\item id (project specific vulnerability id)
\end{compactitem}

In addition the following tags are added:


\begin{compactitem}
\item reported (initial report date)
\item reportedBy (contact information for the persons or entity that initially reported the vulnerability)
\item resolvedBy (contact information for the persons or entity that resolved the vulnerability)
\item name (name of the product, MUST)
\item cve (unique CVE ID)
\item cwe (unique CWE ID)
\item severity (low, medium high)
\item affected (version(s) using composer syntax [5])
\item status (open, in progress, disputed, completed, MUST)
\item remediation (textual description for how to fix an affected system)
\item remediationType (workaround, mitigation, vendor fix, none available, will not fix)
\item remediationLink (URL to give additional information for remediation)
\end{compactitem}


\chapter{PSR-10}

\section{Overview}

There are two aspects with dealing with security issues: One is the process by which security issues are reported and fixed in projects, the other is how the general public is informed about the issues and any remedies available. While PSR-10 addresses the later, this PSR, ie. PSR-9, deals with the former. So the goal of PSR-9 is to define the process by which security researchers and report security vulnerabilities to projects. It is important that when security vulnerabilities are found that researchers have an easy channel to the projects in question allowing them to disclose the issue to a controlled group of people.

The goal of this PSR is to give researchers, project leads, upstream project leads and end users a defined and structured process for disclosing security vulnerabilities.



\section{Security Disclosure Process Discovery}

Every project MUST provide a link to its security disclosure process in an obvious place. Ideally this should be on the root page the main domain of the given project. This MAY be a sub-domain in case it is a sub-project of a larger initiative. The link MAY use the custom link relation php-vuln-reporting, ie. for example <link rel="php-vuln-reporting" href="http://example.org/security"/>.

Projects SHOULD ideally make the location prominent itself by either creating a dedicated sub-domain like http://security.example.org or by making it a top level directory like http://example.org/security. Alternatively projects MAY also simply reference this document, ie. PSR-9. By referencing PSR-9 a project basically states that they follow the default procedures as noted in the section "Default Procedures" towards the end of this document. Projects MUST list the variables noted at the start of that section in this reference (ie. project name, project domain, etc.). Projects MAY choose to list any part of the procedures that is not a MUST which they choose to omit.

Note that projects MAY not have a dedicated domain. For example a project hosted on Github, Bitbucket or other service should still ensure that the process is referenced on the landing page, ie. for example http://github.com/example/somelib should ensure that the default branch has a README file which references the procedures used so that it is automatically displayed.

If necessary projects MAY have different disclosure process for different major version number. In this case one URL MUST be provided for each major version. In the case a major version is no longer receiving security fixes, instead of an URL a project MAY opt to instead simply note that the version is no longer receiving security fixes.


\section{Security Disclosure Process}


Every project MUST provide an email address in their security disclosure process description as the contact email address. Projects SHALL NOT use contact forms.

\section{Default Procedures}

\begin{compactitem}
\item [project name] denotes the name on which the project uses to identify itself.
\item [project domain] denotes the main (sub)domain on which the project relies.
\end{compactitem}

If not specified otherwise, the contact email address is security@[project domain].

\chapter{PSR-11}

\section{Overview}

This document describes a common interface for dependency injection containers.

The goal set by ContainerInterface is to standardize how frameworks and libraries make use of a container to obtain objects and parameters (called entries in the rest of this document).

The word implementor in this document is to be interpreted as someone implementing the ContainerInterface in a dependency injection-related library or framework. Users of dependency injections containers (DIC) are referred to as user.

\section{Specification}


\subsection{Basics}

The Psr\textbackslash Container\textbackslash ContainerInterface exposes two methods : get and has.

\begin{compactitem}
\item \texttt{get} takes one mandatory parameter: an entry identifier. It MUST be a string. A call to get can return anything (a mixed value), or throws an exception if the identifier is not known to the container. Two successive calls to get with the same identifier SHOULD return the same value. However, depending on the implementor design and/or user configuration, different values might be returned, so user SHOULD NOT rely on getting the same value on 2 successive calls. While ContainerInterface only defines one mandatory parameter in get(), implementations MAY accept additional optional parameters.

\item \texttt{has} takes one unique parameter: an entry identifier. It MUST return true if an entry identifier is known to the container and false if it is not. has(\$id) returning true does not mean that get(\$id) will not throw an exception. It does however mean that get(\$id) will not throw a NotFoundException.

\end{compactitem}


\subsection{Exceptions}

Exceptions directly thrown by the container MUST implement the Psr\textbackslash Container\textbackslash Exception\textbackslash ContainerExceptionInterface.

A call to the get method with a non-existing id SHOULD throw a Psr\textbackslash Container\textbackslash Exception\textbackslash NotFoundExceptionInterface.

Users SHOULD NOT pass a container into an object so that the object can retrieve its own dependencies. This means the container is used as a Service Locator which is a pattern that is generally discouraged.

\section{Delegate}

This section describes an additional feature that MAY be added to a container. Containers are not required to implement the delegate lookup to respect the ContainerInterface.

The goal of the delegate lookup feature is to allow several containers to share entries. Containers implementing this feature can perform dependency lookups in other containers.

Containers implementing this feature will offer a greater lever of interoperability with other containers. Implementation of this feature is therefore RECOMMENDED.

A container implementing this feature:

\begin{compactitem}
\item MUST implement the ContainerInterface
\item MUST provide a way to register a delegate container (using a constructor parameter, or a setter, or any possible way). The delegate container MUST implement the ContainerInterface.
\end{compactitem}

When a container is configured to use a delegate container for dependencies:

\begin{compactitem}
\item Calls to the \texttt{get} method should only return an entry if the entry is part of the container. If the entry is not part of the container, an exception should be thrown (as requested by the ContainerInterface).
\item Calls to the \texttt{has} method should only return true if the entry is part of the container. If the entry is not part of the container, false should be returned.
\item If the fetched entry has dependencies, instead of performing the dependency lookup in the container, the lookup is performed on the delegate container.

\end{compactitem}

By default, the lookup SHOULD be performed on the delegate container only, not on the container itself.

It is however allowed for containers to provide exception cases for special entries, and a way to lookup into the same container (or another container) instead of the delegate container.


\section{Package}


The interfaces and classes described as well as relevant exception are provided as part of the psr/container package. (still to-be-created)


\section{Interface}



\subsection{Psr\textbackslash Container\textbackslash ContainerInterface}



\begin{lstlisting}[language=PHP]
<?php
namespace Psr\Container;

use Psr\Container\Exception\ContainerExceptionInterface;
use Psr\Container\Exception\NotFoundExceptionInterface;

/**
 * Describes the interface of a container that exposes methods to read its entries.
 */
interface ContainerInterface
{
    /**
     * Finds an entry of the container by its identifier and returns it.
     *
     * @param string $id Identifier of the entry to look for.
     *
     * @throws NotFoundExceptionInterface  No entry was found for this identifier.
     * @throws ContainerExceptionInterface Error while retrieving the entry.
     *
     * @return mixed Entry.
     */
    public function get($id);

    /**
     * Returns true if the container can return an entry for the given identifier.
     * Returns false otherwise.
     *
     * `has($id)` returning true does not mean that `get($id)` will not throw an exception.
     * It does however mean that `get($id)` will not throw a `NotFoundException`.
     *
     * @param string $id Identifier of the entry to look for.
     *
     * @return boolean
     */
    public function has($id);
}
\end{lstlisting}


\subsection{Psr\textbackslash Container\textbackslash Exception\textbackslash ContainerExceptionInterface}




\begin{lstlisting}[language=PHP]
<?php
namespace Psr\Container\Exception;

/**
 * Base interface representing a generic exception in a container.
 */
interface ContainerExceptionInterface
{
}
\end{lstlisting}

\subsection{Psr\textbackslash Container\textbackslash Exception\textbackslash NotFoundExceptionInterface}



\begin{lstlisting}[language=PHP]
<?php
namespace Psr\Container\Exception;

/**
 * No entry was found in the container.
 */
interface NotFoundExceptionInterface extends ContainerExceptionInterface
{
}
\end{lstlisting}


\chapter{PSR-12}


\section{Overview}


This specification extends, expands and replaces PSR-2, the coding style guide and requires adherance to PSR-1, the basic coding standard.

Like PSR-2, the intent of this specification is to reduce cognitive friction when scanning code from different authors. It does so by enumerating a shared set of rules and expectations about how to format PHP code. This PSR seeks to provide a set way that coding style tools can implement, projects can declare adherence to and developers can easily relate to between different projects. When various authors collaborate across multiple projects, it helps to have one set of guidelines to be used among all those projects. Thus, the benefit of this guide is not in the rules themselves but the sharing of those rules.

PSR-2 was accepted in 2012 and since then a number of changes have been made to PHP which have implications for coding style guidelines. Whilst PSR-2 is very comprehensive of PHP functionality that existed at the time of writing, new functionality is very open to interpretation. This PSR therefore seeks to clarify the content of PSR-2 in a more modern context with new functionality available, and make the errata to PSR-2 binding.



Throughout this document, any instructions MAY be ignored if they do not exist in versions of PHP supported by your project.

This example encompasses some of the rules below as a quick overview:




\begin{lstlisting}[language=PHP]
<?php
declare(strict_types=1);

namespace Vendor\Package;

use Vendor\Package\{ClassA as A, ClassB, ClassC as C};
use Vendor\Package\Namespace\ClassD as D;

use function Vendor\Package\{functionA, functionB, functionC};
use const Vendor\Package\{ConstantA, ConstantB, ConstantC};

class Foo extends Bar implements FooInterface
{
    public function sampleFunction(int $a, int $b = null): array
    {
        if ($a === $b) {
            bar();
        } elseif ($a > $b) {
            $foo->bar($arg1);
        } else {
            BazClass::bar($arg2, $arg3);
        }
    }

    final public static function bar()
    {
        // method body
    }
}
\end{lstlisting}


\section{Basics}


Code MUST follow all rules outlined in PSR-1.

The term 'StudlyCaps' in PSR-1 MUST be interpreted as PascalCase where the first letter of each word is capitalised including the very first letter.


\section{Files}

All PHP files MUST use the Unix LF (linefeed) line ending.

All PHP files MUST end with a single line, containing only a single newline (LF) character.

The closing \texttt{?>} tag MUST be omitted from files containing only PHP.

\section{Lines}


There MUST NOT be a hard limit on line length.

The soft limit on line length MUST be 120 characters; automated style checkers MUST warn but MUST NOT error at the soft limit.

Lines SHOULD NOT be longer than 80 characters; lines longer than that SHOULD be split into multiple subsequent lines of no more than 80 characters each.

There MUST NOT be trailing whitespace at the end of lines.

Blank lines MAY be added to improve readability and to indicate related blocks of code except where explictly forbidden.

There MUST NOT be more than one statement per line.


\section{Intenting}

Code MUST use an indent of 4 spaces for each indent level, and MUST NOT use tabs for indenting.



\section{Keywords}

PHP keywords MUST be in lower case.

The PHP types and keywords array, int, true, object, float, false, mixed, bool, null, numeric, string, void and resource MUST be in lower case.

Short form of type keywords MUST be used in both code and documentation blocks i.e. bool instead of boolean, int instead of integer etc.

\section{Declaration}

The header of a PHP file may consist of a number of different blocks. If present, each of the blocks below MUST be separated by a single blank line, and MUST NOT contain a blank line. Each block MUST be in the order listed below, although blocks that are not relevant may be omitted.

\begin{compactitem}
\item File-level docblock.
\item One or more declare statements.
\item The namespace declaration of the file.
\item One or more class-based use statements.
\item One or more function-based use statements.
\item One or more constant-based use statements.
\item The remainder of the code in the file.
\end{compactitem}

When a file contains a mix of HTML and PHP, any of the above sections may still be used. If so, they MUST be present at the top of the file, even if the remainder of the code consists a closing PHP tag and then a mixture of HTML and PHP.

When the opening <?php tag is on the first line of the file, it MUST be on its own line with no other statements unless it is a file containing markup outside of PHP opening and closing tags.

The following example illustrates a complete list of all blocks:



\begin{lstlisting}[language=PHP]
<?php

/**
 * This file contains an example of coding styles.
 */

declare(strict_types=1);

namespace Vendor\Package;

use Vendor\Package\{ClassA as A, ClassB, ClassC as C};
use Vendor\Package\Namespace\ClassD as D;
use Vendor\Package\AnotherNamespace\ClassE as E;

use function Vendor\Package\{functionA, functionB, functionC};
use function Another\Vendor\function D;

use const Vendor\Package\{CONSTANT_A, CONSTANT_B, CONSTANT_C};
use const Another\Vendor\CONSTANT_D;

/**
 * FooBar is an example class.
 */
class FooBar
{
    // ... additional PHP code ...
}
\end{lstlisting}


Compound namespaces with a depth of more than two MUST not be used. Therefore the following is the maximum compounding depth allowed:





\begin{lstlisting}[language=PHP]
<?php

use Vendor\Package\Namespace\{
    SubnamespaceOne\ClassA,
    SubnamespaceOne\ClassB,
    SubnamespaceTwo\ClassY,
    ClassZ,
};
\end{lstlisting}



And the following would not be allowed:




\begin{lstlisting}[language=PHP]
<?php

use Vendor\Package\Namespace\{
    SubnamespaceOne\AnotherNamespace\ClassA,
    SubnamespaceOne\ClassB,
    ClassZ,
};
\end{lstlisting}


When wishing to declare strict types in files containing markup outside PHP opening and closing tags MUST, on the first line, include an opening php tag, the strict types declaration and closing tag.



\begin{lstlisting}[language=PHP]
<?php declare(strict_types=1); ?>
<html>
<body>
    <?php
        // ... additional PHP code ...
    ?>
</body>
</html>
\end{lstlisting}


Declare statements MUST contain no spaces and MUST look like declare(strict\_types=1);.

Block declare statements are allowed and MUST be formatted as below. Note position of braces and spacing:





\begin{lstlisting}[language=PHP]
declare(ticks=1) {
    //some code
}
\end{lstlisting}


\section{Classes}

The term "class" refers to all classes, interfaces, and traits.

Any closing brace must not be followed by any comment or statement on the same line.

When instantiating a new class, parenthesis MUST always be present even when there are no arguments passed to the constructor.


\begin{lstlisting}[language=PHP]
new Foo();
\end{lstlisting}


\section{Inherit}


The extends and implements keywords MUST be declared on the same line as the class name.

The opening brace for the class MUST go on its own line; the closing brace for the class MUST go on the next line after the body.

Opening braces MUST be on their own line and MUST NOT be preceded or followed by a blank line.

Closing braces MUST be on their own line and MUST NOT be preceded by a blank line.




\begin{lstlisting}[language=PHP]
<?php
namespace Vendor\Package;

use FooClass;
use BarClass as Bar;
use OtherVendor\OtherPackage\BazClass;

class ClassName extends ParentClass implements \ArrayAccess, \Countable
{
    // constants, properties, methods
}
\end{lstlisting}

Lists of implements and extends MAY be split across multiple lines, where each subsequent line is indented once. When doing so, the first item in the list MUST be on the next line, and there MUST be only one interface per line.




\begin{lstlisting}[language=PHP]
<?php
namespace Vendor\Package;

use FooClass;
use BarClass as Bar;
use OtherVendor\OtherPackage\BazClass;

class ClassName extends ParentClass implements
    \ArrayAccess,
    \Countable,
    \Serializable
{
    // constants, properties, methods
}
\end{lstlisting}


\section{Traits}

The use keyword used inside the classes to implement traits MUST be declared on the next line after the opening brace.




\begin{lstlisting}[language=PHP]
<?php
namespace Vendor\Package;

use Vendor\Package\FirstTrait;

class ClassName
{
    use FirstTrait;
}
\end{lstlisting}


Each individual Trait that is imported into a class MUST be included one-per-line, and each inclusion MUST have its own use statement.





\begin{lstlisting}[language=PHP]
<?php
namespace Vendor\Package;

use Vendor\Package\FirstTrait;
use Vendor\Package\SecondTrait;
use Vendor\Package\ThirdTrait;

class ClassName
{
    use FirstTrait;
    use SecondTrait;
    use ThirdTrait;
}
\end{lstlisting}



When the class has nothing after the use declaration, the class closing brace MUST be on the next line after the use declaration.





\begin{lstlisting}[language=PHP]
<?php
namespace Vendor\Package;

use Vendor\Package\FirstTrait;

class ClassName
{
    use FirstTrait;
}
\end{lstlisting}


Otherwise it MUST have a blank line after the use declaration.



\begin{lstlisting}[language=PHP]
<?php
namespace Vendor\Package;

use Vendor\Package\FirstTrait;

class ClassName
{
    use FirstTrait;

    private $property;
}
\end{lstlisting}


\section{Properties}

Visibility MUST be declared on all properties.

The var keyword MUST NOT be used to declare a property.

There MUST NOT be more than one property declared per statement.

Property names MUST NOT be prefixed with a single underscore to indicate protected or private visibility. That is, an underscore prefix explicitly has no meaning.

A property declaration looks like the following.


\begin{lstlisting}[language=PHP]
<?php
namespace Vendor\Package;

class ClassName
{
    public $foo = null;
}
\end{lstlisting}



\section{Methods}

Visibility MUST be declared on all methods.

Method names MUST NOT be prefixed with a single underscore to indicate protected or private visibility. That is, an underscore prefix explicitly has no meaning.

Method and function names MUST NOT be declared with a space after the method name. The opening brace MUST go on its own line, and the closing brace MUST go on the next line following the body. There MUST NOT be a space after the opening parenthesis, and there MUST NOT be a space before the closing parenthesis.

A method declaration looks like the following. Note the placement of parentheses, commas, spaces, and braces:





\begin{lstlisting}[language=PHP]
<?php
namespace Vendor\Package;

class ClassName
{
    public function fooBarBaz($arg1, &$arg2, $arg3 = [])
    {
        // method body
    }
}
\end{lstlisting}

A function declaration looks like the following. Note the placement of parentheses, commas, spaces, and braces:



\begin{lstlisting}[language=PHP]
<?php

function fooBarBaz($arg1, &$arg2, $arg3 = [])
{
    // function body
}
\end{lstlisting}


\section{Arguments}

In the argument list, there MUST NOT be a space before each comma, and there MUST be one space after each comma.

Method and function arguments with default values MUST go at the end of the argument list.

Method and function argument scalar type hints MUST be lowercase.




\begin{lstlisting}[language=PHP]
<?php
namespace Vendor\Package;

class ClassName
{
    public function foo(int $arg1, &$arg2, $arg3 = [])
    {
        // method body
    }
}
\end{lstlisting}

Argument lists MAY be split across multiple lines, where each subsequent line is indented once. When doing so, the first item in the list MUST be on the next line, and there MUST be only one argument per line.

When the argument list is split across multiple lines, the closing parenthesis and opening brace MUST be placed together on their own line with one space between them.




\begin{lstlisting}[language=PHP]
<?php
namespace Vendor\Package;

class ClassName
{
    public function aVeryLongMethodName(
        ClassTypeHint $arg1,
        &$arg2,
        array $arg3 = []
    ) {
        // method body
    }
}
\end{lstlisting}



When you have a return type declaration present there MUST be one space after the colon with followed by the type declaration. The colon and declaration MUST be on the same line as the argument list closing parentheses with no spaces between the two characters. The declaration keyword (e.g. string) MUST be lowercase.




\begin{lstlisting}[language=PHP]
<?php
declare(strict_types=1);

namespace Vendor\Package;

class ReturnTypeVariations
{
    public function functionName($arg1, $arg2): string
    {
        return 'foo';
    }
}
\end{lstlisting}


\section{Visibility}



When present, the abstract and final declarations MUST precede the visibility declaration.

When present, the static declaration MUST come after the visibility declaration.

\begin{lstlisting}[language=PHP]
<?php
namespace Vendor\Package;

abstract class ClassName
{
    protected static $foo;

    abstract protected function zim();

    final public static function bar()
    {
        // method body
    }
}
\end{lstlisting}

\section{Calling}

When making a method or function call, there MUST NOT be a space between the method or function name and the opening parenthesis, there MUST NOT be a space after the opening parenthesis, and there MUST NOT be a space before the closing parenthesis. In the argument list, there MUST NOT be a space before each comma, and there MUST be one space after each comma.



\begin{lstlisting}[language=PHP]
<?php

bar();
$foo->bar($arg1);
Foo::bar($arg2, $arg3);
\end{lstlisting}



Argument lists MAY be split across multiple lines, where each subsequent line is indented once. When doing so, the first item in the list MUST be on the next line, and there MUST be only one argument per line. A single argument being split across multiple lines (as might be the case with an anonymous function or array) does not constitute splitting the argument list itself.




\begin{lstlisting}[language=PHP]
<?php

$foo->bar(
    $longArgument,
    $longerArgument,
    $muchLongerArgument
);
\end{lstlisting}






\begin{lstlisting}[language=PHP]
<?php

somefunction($foo, $bar, [
  // ...
], $baz);

$app->get('/hello/{name}', function ($name) use ($app) {
    return 'Hello ' . $app->escape($name);
});
\end{lstlisting}

\section{Structure}

The general style rules for control structures are as follows:

\begin{compactitem}
\item There MUST be one space after the control structure keyword
\item There MUST NOT be a space after the opening parenthesis
\item There MUST NOT be a space before the closing parenthesis
\item There MUST be one space between the closing parenthesis and the opening brace
\item The structure body MUST be indented once
\item The closing brace MUST be on the next line after the body
\end{compactitem}

The body of each structure MUST be enclosed by braces. This standardizes how the structures look, and reduces the likelihood of introducing errors as new lines get added to the body.

\subsection{if/elseif/else}


An if structure looks like the following. Note the placement of parentheses, spaces, and braces; and that else and elseif are on the same line as the closing brace from the earlier body.





\begin{lstlisting}[language=PHP]
<?php

if ($expr1) {
    // if body
} elseif ($expr2) {
    // elseif body
} else {
    // else body;
}
\end{lstlisting}



The keyword elseif SHOULD be used instead of else if so that all control keywords look like single words.


\subsection{switch/case}

A switch structure looks like the following. Note the placement of parentheses, spaces, and braces. The case statement MUST be indented once from switch, and the break keyword (or other terminating keyword) MUST be indented at the same level as the case body. There MUST be a comment such as // no break when fall-through is intentional in a non-empty case body.




\begin{lstlisting}[language=PHP]
<?php

switch ($expr) {
    case 0:
        echo 'First case, with a break';
        break;
    case 1:
        echo 'Second case, which falls through';
        // no break
    case 2:
    case 3:
    case 4:
        echo 'Third case, return instead of break';
        return;
    default:
        echo 'Default case';
        break;
}
\end{lstlisting}


\subsection{while/do-while}

A while statement looks like the following. Note the placement of parentheses, spaces, and braces.



\begin{lstlisting}[language=PHP]
<?php

while ($expr) {
    // structure body
}
\end{lstlisting}

Similarly, a do while statement looks like the following. Note the placement of parentheses, spaces, and braces.






\begin{lstlisting}[language=PHP]
<?php

do {
    // structure body;
} while ($expr);
\end{lstlisting}



\subsection{for}

A for statement looks like the following. Note the placement of parentheses, spaces, and braces.




\begin{lstlisting}[language=PHP]
<?php

for ($i = 0; $i < 10; $i++) {
    // for body
}
\end{lstlisting}


\subsection{foreach}

A foreach statement looks like the following. Note the placement of parentheses, spaces, and braces.




\begin{lstlisting}[language=PHP]
<?php

foreach ($iterable as $key => $value) {
    // foreach body
}
\end{lstlisting}


\subsection{try/catch/finally}

A try-catch-finally block looks like the following. Note the placement of parentheses, spaces, and braces.




\begin{lstlisting}[language=PHP]
<?php

try {
    // try body
} catch (FirstThrowableType $e) {
    // catch body
} catch (OtherThrowableType $e) {
    // catch body
} finally {
    // finally body
}
\end{lstlisting}

\section{Operators}

All binary and ternary (but not unary) operators MUST be preceded and followed by at least one space. This includes all arithmetic, comparison, assignment, bitwise, logical (excluding ! which is unary), string concatenation, and type operators.

Other operators are left undefined.




\begin{lstlisting}[language=PHP]
<?php

if ($a === $b) {
    $foo = $bar ?? $a ?? $b;
} elseif ($a > $b) {
    $variable = $foo ? 'foo' : 'bar';
}
\end{lstlisting}

\section{Closures}

Closures MUST be declared with a space after the function keyword, and a space before and after the use keyword.

The opening brace MUST go on the same line, and the closing brace MUST go on the next line following the body.

There MUST NOT be a space after the opening parenthesis of the argument list or variable list, and there MUST NOT be a space before the closing parenthesis of the argument list or variable list.

In the argument list and variable list, there MUST NOT be a space before each comma, and there MUST be one space after each comma.

Closure arguments with default values MUST go at the end of the argument list.

A closure declaration looks like the following. Note the placement of parentheses, commas, spaces, and braces:



\begin{lstlisting}[language=PHP]
<?php

$closureWithArgs = function ($arg1, $arg2) {
    // body
};

$closureWithArgsAndVars = function ($arg1, $arg2) use ($var1, $var2) {
    // body
};
\end{lstlisting}


Argument lists and variable lists MAY be split across multiple lines, where each subsequent line is indented once. When doing so, the first item in the list MUST be on the next line, and there MUST be only one argument or variable per line.

When the ending list (whether of arguments or variables) is split across multiple lines, the closing parenthesis and opening brace MUST be placed together on their own line with one space between them.

The following are examples of closures with and without argument lists and variable lists split across multiple lines.

\begin{lstlisting}[language=PHP]
<?php

$longArgs_noVars = function (
    $longArgument,
    $longerArgument,
    $muchLongerArgument
) {
   // body
};

$noArgs_longVars = function () use (
    $longVar1,
    $longerVar2,
    $muchLongerVar3
) {
   // body
};

$longArgs_longVars = function (
    $longArgument,
    $longerArgument,
    $muchLongerArgument
) use (
    $longVar1,
    $longerVar2,
    $muchLongerVar3
) {
   // body
};

$longArgs_shortVars = function (
    $longArgument,
    $longerArgument,
    $muchLongerArgument
) use ($var1) {
   // body
};

$shortArgs_longVars = function ($arg) use (
    $longVar1,
    $longerVar2,
    $muchLongerVar3
) {
   // body
};
\end{lstlisting}

Note that the formatting rules also apply when the closure is used directly in a function or method call as an argument.




\begin{lstlisting}[language=PHP]
<?php

$foo->bar(
    $arg1,
    function ($arg2) use ($var1) {
        // body
    },
    $arg3
);
\end{lstlisting}


\section{Anonymous}

Anonymous Classes MUST follow the same guidelines and principles as closures in the above section.



\begin{lstlisting}[language=PHP]
<?php

$instance = new class {};
\end{lstlisting}

The opening parenthesis MAY be on the same line as the class keyword so long as the list of implements interfaces does not wrap. If the list of interfaces wraps, the parenthesis MUST be placed on the line immediately following the last interface.



\begin{lstlisting}[language=PHP]
<?php

// Parenthesis on the same line
$instance = new class extends \Foo implements \HandleableInterface {
    // Class content
};

// Parenthesis on the next line
$instance = new class extends \Foo implements
    \ArrayAccess,
    \Countable,
    \Serializable
{
    // Class content
};
\end{lstlisting}



\chapter{PSR-13}

\begin{lstlisting}[language=PHP]

\end{lstlisting}




\begin{lstlisting}[language=PHP]

\end{lstlisting}



\begin{lstlisting}[language=PHP]

\end{lstlisting}



\begin{lstlisting}[language=PHP]

\end{lstlisting}