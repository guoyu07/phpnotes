\part{Domain-Driven Design}

\chapter{Overview}




“领域驱动设计”,顾名思义,首先强调的是“领域”。这个词不是指技术上的任何东西,而是指“业务领域”,是说用领域的角度去分析和设计业务。
 
现实中的大部分开发者并不会直接接触业务,实际接触到的仅仅是架构师消化过的,而且企业开发中处处充满着管理的不规范性,再加上用户的很多需求本身就是不规范的,因此实际生产出的软件只能是怪胎,大多数软件只能满足一时之需,跟不上用户的成长和规范。

为了实施领域驱动设计,需要在向领域专家和专业书籍学习的同时,找出当前用户在业务不足之处,从而在规范和现实之间构建桥梁。

“领域驱动设计”的另一层含义是“驱动设计”。显然,“设计”只是一种手段和工具,是为“领域”服务的。“设计”的过程和质量必须通过“领域模型”来检验和验证,这也是测试驱动的本质。



尽可能不要让设计工具本身的缺陷扭曲了“领域”本身,否则就本末倒置了。例如,Banq批判的“基于关系数据库的业务设计”以及坚持“域模型不要让技术污染“。

在生活中也是如此,例如在骑车或者开车出行时,用户心中都有一辆真正属于自己的车,至于这个车是不是有轮子等都并不重要,关键是能让自己舒适地出行。显然”舒适地出行“才是真正的业务领域,而”轮子之类的“只是设计。

领域驱动设计、敏捷工程以及测试驱动是浑然一体的,三者紧密结合,形成一套新的软件开发方式,而且已经开始普及和成熟,未来可能专门开发出专门的MDD工具或DSL来支持领域开发。

模型驱动架构、开发和工程学以及丰富的模型驱动开发环境——例如NeXTStep——都是在80年代末出现的,现在模型已经无处不在,然而开发工具的数量还是很少,业界还在寻找让模型驱动开发成为主流的方法。

