\part{Defensive}


\chapter{Overview}


防御性编程(Defensive programming)是防御式设计的一种具体体现,它是为了保证对程序的不可预见的使用不会造成程序功能上的损坏。

防御性编程可以被看作是为了减少或消除墨菲定律效应的想法,主要用于可能被滥用,恶作剧或无意地造成灾难性影响的程序上。

潜在的软件缺陷可能会被黑客利用来进行代码注入,拒绝服务攻击或其他攻击,因此防御性编程又称为安全编程(Secure Programming)。

防御性编程与非防御性编程之间的区别在于,程序员不会对特定的函数调用或库的使用情况做假设。例如,下面的示例程序中,当输入超过1000个字符时,该函数将会崩溃。

\begin{lstlisting}[language=PHP]
int risky_programming(char *input) {
   char str[1000+1]; // one more for the null character
   // ...
   strcpy(str, input); // copy input
   // ...
}
\end{lstlisting}

一些新手程序员可能并不会觉得这是个问题,因为没有用户会输入这么长的字符串。

防御性编程不会允许这样的错误,因为这段程序包含一个已知的bug,一个可能会导致缓冲区溢出攻击的漏洞。下面这个例子是一个解决方案:

\begin{lstlisting}[language=PHP]
int secure_programming(char *input) {
   char str[1000];
   // ...
   strncpy(str, input, sizeof(str)); // copy input without exceeding the length of the destination
   str[sizeof(str)-1] = '\0'; // if strlen(input) == sizeof(str) then strcpy won't NUL terminate
   // ...
}
\end{lstlisting}